\chapter{Εισαγωγή}
\section{Περιγραφή του προβλήματος}

Το πρόβλημα που καλούμαστε να λύσουμε είναι η πρόβλεψη σφαλμάτων (bugs) σε έργα λογισμικού.
Ένας τρόπος αντιμετώπισης είναι η χρήση διάφορων μερτικών πάνω στα τμήματα που έχουν και που δεν έχουν σφάλμα. Στο συγκεκριμένο project μας δίνεται ένα σύνολο από κλάσεις με υπολογισμένες μετρικές και γνώση αν περιέχουν σφάλμα ή όχι (training set). Με την χρήση αυτών των δεδομένων θα δημιουργήσουμε διάφορα μοντέλα ταξινόμησης που θα ταξινομούν τις κλάσεις σε εσφαλμένες και μη.

Εδώ πρέπει να ορίσουμε με ποια κριτήρια θα κρίνουμε αν ένα μοντέλο είναι καλό ή όχι. Για αυτό θα λάβουμε υπόψιν 2 παράγοντες.
\begin{enumerate}
\item Θεωρούμε σημαντικό να βρεθούν όσο το δυνατόν περισσότερο bugs από αυτά που υπάρχουν (recall) καθώς η διόρθωση σε αρχικά στάδια ενός έργου λογισμικού είναι πολύ πιο φθηνή.
\item Θα διατεθούν πόροι για κάθε κλάση που θα θεωρήσει το μοντέλο ταξινόμησης μας εσφαλμένη ώστε είτε να επιδιορθωθεί το bug ή να εξακριβωθεί ότι δεν περιέχει κάποιο bug. Συνεπώς, είναι σημαντικό το ποσοστό των  σωστά προβλεμένων εσφαλμένων κλάσεων προς των συνολικών προβλεμένων εσφαλμέων κλάσεων (precision) να είναι υψηλό.
\end{enumerate}

Συνολικά λοιπόν θα δώσουμε βάση στην μετρική f-measure που συνδυάζει και το recall και το precision. Ακόμη, θα δώσουμε περισσότερη έμφαση στο recall σε σχέση με το precision γιατί θεωρούμε ότι είναι πιο οικονομικό να διορθωθούν πολλά bugs στην αρχή παρά στο λίγα στο τέλος. 
\newpage
\section{Αλγόριθμοι που χρησιμοποιήθηκαν}
Οι αλγόριθμοι ταξινόμησης που χρησιμοποιήσαμε ήταν :
\begin{enumerate}
  \item \textbf{Bayes:}
  \begin{itemize}
     \item NaiveBayes
     \item BayesNet
  \end{itemize}  
  \item \textbf{Trees:}
  \begin{itemize}
     \item J48
     \item RandomForest
  \end{itemize}
  \item \textbf{SVM:}  
  \begin{itemize}
     \item C-SVC (libsvm)
     \item SMO
  \end{itemize}
  \item \textbf{NNs:}
  \begin{itemize}
     \item KNN
     \item NN
  \end{itemize}
\end{enumerate}

\section{Εργαλεία που χρησιμοποιήθηκαν}

\begin{enumerate}
  \item \textbf{Weka}
  \begin{itemize}
     \item TODO
  \end{itemize}  
  \item \textbf{Python:}
\begin{itemize}
     \item TODO
\end{itemize}
\end{enumerate}