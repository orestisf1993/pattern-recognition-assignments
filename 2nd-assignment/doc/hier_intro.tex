\chapter{Ιεραρχικοί αλγόριθμοι}
Οι ιεραρχικοί αλγόριθμοι (hieratchical cluster analysis or HCA) είναι μία μεγάλη κατηγορία αλγορίθμων που χρησιμοποιούνται για να κάνουμε ομαδοποίηση δεδομένων .Ονομάζονται έτσι επειδή προσπαθούν να δημιουργήγουν μία ιεραρχία από συστάδες (clusters)
Οο στρατηγικές που χρησιμοπούνται είναι γενικά 2
\begin{itemize}
\item{η \textbf{Συνάνθροισης (Aglomerative)} Είναι μία μέθοδος 'bottom-up' δηλαδή : στην αρχή κάθε cluster αποτελείται από μία παρατήρηση και μετά ζευγάρια από cluster 'συναθροίζονται' και γίνονται ένα cluster όσο άνεβάινουμε την ιεραρχία  }
\item{}οι\textbf{Διαχωρισμού (Divisive)} Είναι ουσιαστικά το ανάποδο ('top down) όπου αρχικά όλες οι παρατηρήσεις είναι ένα cluster και έπειτα διαχωρίζοντε όσο κατεβαίνουμε την ιεραρχία.(Τέτοιος αλγόριθμος είναι το Ελαφρύτατου Συνδετικού Δεντρο (Minimum Spanning Tree)) 
end{itemize}

Εφαρμόστηκαν αλγόριθμοι με την μέθοδο της συνάθροισης για αυτό και θα αναλύσουμε κυρίως   