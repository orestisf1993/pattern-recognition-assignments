\section{Ιεραρχικοί αλγόριθμοι}
Οι ιεραρχικοί αλγόριθμοι (hierarchical cluster analysis or HCA) είναι μία μεγάλη κατηγορία αλγορίθμων που χρησιμοποιούνται για να κάνουμε ομαδοποίηση δεδομένων .Ονομάζονται έτσι επειδή προσπαθούν να δημιουργήσουν μία ιεραρχία από συστάδες (clusters).

Οι στρατηγικές που χρησιμοποιούνται είναι γενικά 2:
\begin{itemize}
\item{ \textbf{Συνάθροισης (Agglomerative)} Είναι μία μέθοδος 'bottom-up' δηλαδή : στην αρχή κάθε cluster αποτελείται από μία παρατήρηση και μετά ανά 2 'συναθροίζονται' και γίνονται ένα cluster όσο ανεβαίνουμε την ιεραρχία  }
\item{}\textbf{Διαχωρισμού (Divisive)} Είναι ουσιαστικά το ανάποδο 'top down' όπου αρχικά όλες οι παρατηρήσεις είναι ένα cluster και έπειτα διαχωρίζονται όσο κατεβαίνουνε την ιεραρχία.Τέτοιος αλγόριθμος είναι το Ελαφρύτατου Συνδετικού Δέντρο (Minimum Spanning Tree))
\end{itemize}

Εφαρμόστηκαν αλγόριθμοι με την μέθοδο της συνάθροισης για αυτό και θα αναλύσουμε κυρίως   αυτούς αν και υπάρχουν αρκετά κοινά σημεία στις 2 μεθόδους
Σε αυτούς τους Αλγορίθμους 2 είναι οι βασικές έννοιες που χρειάζονται .
Η πρώτη είναι η μετρική(metric) με την οποία θα μετράμε την απόσταση ανάμεσα σε 2 clusters και η δεύτερη είναι ο τρόπος με τον οποίος θα συνδέουμε τα κοντινότερα clusters (linkage)

Μερικές μετρικές που χρησιμοποιούνται σαν απόσταση είναι:
\begin{enumerate}
   \item \textbf{Ευκλείδεια }H κλασική ευκλείδεια απόσταση
\item \textbf{Mahalanobis }Απόσταση προσαρμοσμένη στην διασπορά
\item \textbf{Correlation}Απόσταση που εμπεριέχει την συσχέτιση των σημείων.
\item \textbf{Cosine}Απόσταση που εμπεριέχει το συνημίτονο της γωνίας των σημείων.
\end{enumerate}
Μπορούν να χρησιμοποιηθούν πολλές διαφορετικές μετρικές ανάλογα με το πρόβλημα και την περίσταση.

\noindent\begin{minipage}{\linewidth}
\centering
\captionsetup{type={figure}}
H απόσταση ανάμεσα στα   clusters  συνήθως οπτικοποιείται με έναν πίνακα ομοιότητας χρωματισμένο σαν θερμοκρασία, το λεγόμενο heatmap  όπως το παρακάτω :
\makebox[\linewidth]{%        to center the image
    \includegraphics[keepaspectratio=true,width=1.0\linewidth]{images/heat1}}
    \captionof{figure}{Πίνακας ομοιότητα από τα δεδομένα της άσκησης}\label{fig:heat1}
\end{minipage}

Οι πιο γνωστοί τρόποι για linkage είναι 
\begin{enumerate}
\item \textbf{Max or Complete }Η μεγαλύτερη απόσταση ανάμεσα σε 2 cluster
\item \textbf{Min or Single}H μικρότερη απόσταση ανάμεσα σε 2 clusters 
\item \textbf{Average }Ο μέσος όρος της απόστασης  ανάμεσα 2 clusters 
\item \textbf{Centroid } απόσταση ανάμεσα στα κέντρα των clusters.
\end{enumerate}
Εννοείται ότι η απόσταση εξαρτάται από την μετρική που έχουμε επιλέξει.
 Αφού έχουμε επιλέξει την μετρική της απόστασης και της σύνδεσης η διαδικασία που ακολουθείται για να δημιουργηθεί η ιεραρχία μπορεί να περογραφέι από τον παρακάτω ψευδοκώδικα:
 
\begin{algorithm}[H]
Calculate proximity\_Matrix\;
Define each point as cluster\;
\While{num\_clust bigger than 1}{
    link closest clusters\;
    update proximity\_Matrix\;
    num\_clust +=1
}
\end{algorithm}

\noindent\begin{minipage}{\linewidth}
\centering
\captionsetup{type={figure}}
Ένας σύνηθες τρόπος να το δούμε είναι η  χρήση ενός δένδρο-διαγράμματος
\makebox[\linewidth]{
  \includegraphics[keepaspectratio=true,width=1.0\linewidth]{images/hier_tree}}
\captionof{figure}{Παράδειγμα δεντρο-διαγράμματος}\label{fig:dentro1}
\end{minipage}
 
Τέλος αφού έχουμε την ιεραρχία μπορούμε να σπάμε τους πιο ασθενείς συνδέσμους
(πιο ψηλούς στο δένδρο-διάγραμμα )
και να δημιουργούμε μικρότερα clusters.
Το κάνουμε ώσπου να έχουμε τον επιθυμητό αριθμό από clusters
ή όταν όλες οι συνδέσεις είναι αρκετά ισχυρές.


