\chapter{Ανοιχτά θέματα}\label{chapter:openissues}
Στο πλαίσιο της συγκεκριμένης εργασίας προέκυψαν ορισμένα θέματα τα οποία θα μπορούσαμε να έχουμε διαχειριστεί διαφορετικά. Αυτά τα θέματα αφορούν τα dataset μας , το στάδιο της προ-επεξεργασίας, τους αλγορίθμους που χρησιμοποιήσαμε και διάφορες παραμέτρους μέσα σε αυτές καθώς και τις μετρικές για την αξιολόγηση των μοντέλων. Ορισμένα από τα θέματα αυτά είναι:

\begin{enumerate}
    \item Κατά την διάρκεια της προ-επεξεργασίας επιλέξαμε να βγάλουμε ορισμένα outliers τα οποία θεωρήθηκε ότι δεν προσδίδουν καμία χρήσιμη πληροφορία. Η επιλογή μας αυτή έγινε μετά από εξερεύνηση του dataset μας και θεωρούμε ότι έγινε μια αρκετά αντιπροσωπευτική επιλογή. Ωστόσο θα μπορούσαμε να έχουμε δοκιμάσει να απομακρύνουμε περισσότερα outliers. Στην περίπτωση που είχαμε ένα διαφορετικό αρχικό dataset η επιλογή αυτή θα ήταν διαφορετική.
    \item Μία ιδέα που θα μπορούσε να εφαρμοστεί είναι να απομακρυνθούν τα features τα οποία έχουν πολλά distinct values.
    \item Επίσης, κατά την διάρκεια της προ-επεξεργασίας δοκιμάσαμε να απομακρύνουμε τις λέξεις (feature) τα οποία έχουν πολύ χαμηλό ή πολύ υψηλό variance. Αυτή μας η προσπάθεια δεν οδήγησε σε καλύτερα αποτελέσματα και για αυτόν τον λόγο δεν χρησιμοποιήθηκε τελικά.
    \item Για την επιπλέον μείωση του αριθμού των λέξεων μας από τα τελικά dataset μπορούμε να χρησιμοποιήσουμε την τεχνική της Principal Component Analysis (PCA). Αν και δεν θα βελτίωνε τα αποτελέσματα της ομαδοποίησης μας θα παρουσίαζε ένα πιο μικρό σε μέγεθος τελικό dataset.
    \item Χρησιμοποιήσαμε αλγορίθμους Διαχωριστικούς και Ιεραρχικούς. Για τον υπολογισμό των αποστάσεων που απαιτούνται χρησιμοποιήσαμε στο Matlab όλες των ειδών τις αποστάσεις αλλά επικεντρωθήκαμε στις αποστάσεις τύπου cosine και correlation καθώς αυτές ήταν οι πιο αποτελεσματικές για το dataset μας.
    \item Θα μπορούσαμε να χρησιμοποιήσουμε και άλλους αλγορίθμους ομαδοποίησης όπως τους Πυκνωτικούς (DBSCAN), Particle Swarm Optimazation(PSO) ή Διαφορικούς(Differential Evaluation). Ακόμα θα μπορούσαμε να αναπτύξουμε σε μεγαλύτερο βαθμό τους Γενετικούς Αλγορίθμους (GA).
    \item Για την φάση της αξιολόγησης της ομαδοποίησης αν και υπολογίσαμε όλες τις μετρικές (SSE,Cohesion,Separation,Silhouette) δώσαμε μεγαλύτερη βαρύτητα στην μετρική Silhouette η οποία συνδυάζει τις $Seperation$ και $Cohesion$. Θα μπορούσαμε να έχουμε επιλέξει ένα άλλο βέλτιστο μοντέλο αν δίναμε βαρύτητα σε άλλη μετρική.
    \item Τέλος για καλύτερη ομαδοποίηση θα μπορούσαμε να χρησιμοποιήσουμε την τεχνική της μετά-επεξεργασίας (post-processing). Γενικά είναι μια καλή τεχνική για να ελαχιστοποιήσουμε το SSE να βρούμε περισσότερα clusters (επιλέγοντας μεγαλύτερο Κ). Εμείς όμως επειδή θέλουμε ο τελικός αριθμός των ομάδων να είναι ίσος με 8 μπορούμε να αλλάξουμε το συνολικό SSE υλοποιώντας διάφορες λειτουργίες πάνω στις ομάδες, όπως να διαχωρίζουμε ή να ενώνουμε ομάδες. Μία αρκετά διαδεδομένη τεχνική είναι να χρησιμοποιούμε φάσεις διαχωρισμού και συνένωσης. Κατά την διάρκεια του διαχωρισμού, οι ομάδες διαιρούνται, ενώ κατά την διάρκεια της συνένωσης οι ομάδες συνδυάζονται. Με αυτόν τον τρόπο είναι πιθανό να αποφύγουμε να κολλήσουμε σε κάποιο τοπικό ελάχιστο SSE και να επιτύχουμε την βέλτιστη ομαδοποίηση με τον επιθυμητό αριθμό ομάδων.
    
     Οι 2 στρατηγικές για να μειώσουμε το συνολικό SSE αυξάνοντας τον αριθμό των ομάδων είναι η εξής:
    \begin{itemize}
        \item \textbf{Διαχωρισμός Ομάδας} : Η ομάδα με το μεγαλύτερο SSE επιλέγεται συνήθως για να διασπαστεί. Ακόμα θα μπορούσαμε να διασπάσουμε μια ομάδα με την μεγαλύτερη τυπική απόκλιση για ένα συγκεκριμένο γνώρισμα.
        \item \textbf{Δημιουργία ενός νέου Κέντρου Ομάδας} : Συχνά επιλέγεται το σημείο που είναι πιο μακριά από το κέντρο μιας ομάδας.
    \end{itemize} 
    
    Οι 2 στρατηγικές για να μειώσουμε τον συνολικό αριθμό των ομάδων καθώς προσπαθούμε να διατηρήσουμε ελάχιστο το συνολικό SSE είναι οι εξής:
    
        \begin{itemize}
            \item \textbf{Διάλυση Ομάδας} : Αυτό επιτυγχάνεται διαγράφοντας το κέντρο που αντιστοιχεί σε μία ομάδα και αντιστοιχίζοντας τα σημεία σε άλλες ομάδες. Ιδανικά η ομάδα που διαλύεται είναι αυτή που αυξάνει το συνολικό SSE λιγότερο.
            \item \textbf{Συνένωση 2 Ομάδων} : Συνήθως επιλέγονται οι ομάδες με τα κοντινότερα κέντρα αν και ίσως μια καλύτερη τεχνική θα ήταν να επιλέξουμε να συνενώσουμε 2 ομάδες που οδηγούν στην ελάχιστη αύξηση του συνολικού SSE.
        \end{itemize} 
        
        Αυτές οι 2 τεχνικές είναι αυτές που χρησιμοποιούνται σε διάφορους ιεραρχικούς αλγορίθμους ομαδοποίησης. 
        
        Στην περίπτωση μας θα μπορούσαμε στην περίπτωση των διαχωριστικών αλγορίθμων να επιλέξουμε αρχικά ένα μεγαλύτερο αριθμό ομάδων για παράδειγμα Κ=12 και στην συνέχεια εφαρμόζοντας μία από τις 2 στρατηγικές μείωσης ομάδων να φτάσουμε στον τελικό αριθμό ομάδων κρατώντας το συνολικό SSE όσο το δυνατόν μικρότερο μπορούμε.
\end{enumerate}
