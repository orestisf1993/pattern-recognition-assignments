\section{Γενετικοί Αλγόριθμοι}
\subsection{Εισαγωγή}

Οι γενετικοί αλγόριθμοι είναι μια μεγάλη κατηγορία αλγορίθμων η οποία ανήκει στον κλάδο της επιστήμης των υπολογιστών και λύνει προβλήματα ελαχιστοποίησης. Ουσιαστικά πρόκειται για μια μέθοδο αναζήτησης βέλτιστων λύσεων σε συστήματα που μπορούν να περιγραφούν σαν μαθηματικά προβλήματα. Οι γενετικοί αλγόριθμοι είναι στενά συνδεδεμένοι με την επιστήμη της βιολογίας. Οι εξελικτικοί αλγόριθμοι βασίζονται στην αρχή της φυσικής εξέλιξης, δηλαδή της επιβίωσης των ισχυρότερων. Σε αντίθεση µε τις κλασικές μεθόδους αναζήτησης, δε χρησιμοποιούν μοναδικό σημείο αναζήτησης (single-point search),αλλά ένα πληθυσμό σημείων που ονομάζονται άτομα (individuals). Κάθε άτοκο αντιπροσωπεύει μια πιθανή λύση για το σχεδιαστικό πρόβλημα. Στους αλγόριθμους αυτούς, ο πληθυσμός εξελίσσεται προς συνεχώς καλύτερες περιοχές του χώρου αναζήτησης, χρησιµοποιώντας τεχνικές όπως είναι η επιλογή, η διασταύρωση και η μετάλλαξη.
Η μοντελοποίηση των Γενετικών αλγόριθμων είναι εμπνευσμένη από τη θεωρία της εξέλιξης των ειδών και βασίζεται στους μηχανισμούς της βιολογικής αναπαραγωγής που συναντώνται στη φύση. Οι ΓΑ βασίζουν τη λειτουργία τους στην εξελικτική διαδικασία και στηρίζουν την εύρεση λύσεων στην επιβίωση του ικανότερου, των χρωµοσωµικών ανακατατάξεων και των γονιδιακών αλλαγών. Κάθε μεταβολή που προκαλείται σε ένα πληθυσμό προσαρμόζεται καλύτερα ή χειρότερα στις συνθήκες του περιβάλλοντος. Αν προσαρμόζεται καλύτερα τότε οι πρόγονοι εξαφανίζονται και τη θέση τους παίρνουν οι επίγονοι. Αν η μεταβολή προσαρμόζεται χειρότερα, τότε οδηγεί σε θάνατο (απόρριψη). Η ορολογία για την περιγραφή των δομικών τους στοιχείων είναι δανεισμένη από το χώρο της γενετικής.
Συγκεκριμένα, οι ΓΑ αναφέρονται στην έννοια του πληθυσμού ο οποίος αποτελείται από άτομα (individuals) ή γενότυπους. Κάθε άτοκο αποτελείται από χρωµοσώµατα.
Στους ΓΑ αλγόριθμους τα άτοκα αποτελούνται από ένα χρωμόσωμα και οι δύο
έννοιες συνήθως ταυτίζονται.

\subsubsection{Επιλογή Χαρακτηριστικών και Γενετικοί Αλγόριθμοι}
Οι ΓΑ αποτελούν μια στοχαστική μέθοδο καθολικής βελτιστοποίησης µε
δυνατότητες παράλληλης επεξεργασίας ενός πληθυσμού υποψήφιων λύσεων (σημεία αναζήτησης). Χρησιμοποιούν πιθανοθεωρητικούς κανόνες αναζήτησης όχι µε την αυστηρή έννοια αφού η αναπαραγωγή των λύσεων εκτός της τυχαιότητας βασίζεται και στα αποτελέσματα της αξιολόγησης των λύσεων δηλαδή το πόσο καλή είναι µια λύση και µε βάση και αυτό το κριτήριο θα παραμένει και θα εξελιχθεί μέσα σε ένα πληθυσμό. Στο πρόβλημα της επιλογής χαρακτηριστικών τα μέλη του πληθυσμού είναι τα υποσύνολα χαρακτηριστικών (υποψήφιες λύσεις) που αναπαρίστανται από δυαδικά διανύσματα.
Η επεξεργασία των υποψηφίων λύσεων λαμβάνει χώρα σε διακριτές φάσεις που ονομάζονται γενιές.
Σε κάθε γενιά τα υποψήφια υποσύνολα χαρακτηριστικών αξιολογούνται βάση μιας αντικειμενικής συνάρτησης αξιολόγησης και οι πιο εύρωστες λύσεις επιλέγονται για αναπαραγωγή. Η αναπαραγωγή πραγματοποιείται διασταυρώνοντας χαρακτηριστικά (features) από διαφορετικούς υποσύνολα χαρακτηριστικών, ώστε να παραχθούν νέα υποσύνολα (children). Τα νέα υποσύνολα εισέρχονται στη συνέχεια στον πληθυσμό και η διαδικασία επαναλαμβάνεται ακολουθώντας ένα εξελικτικό μοτίβο (Δαρβινικό περιβάλλον).
Η "καθολικά βέλτιστη" μέθοδος βελτιστοποίησης και κατά συνέπεια μια μέθοδος επιλογής χαρακτηριστικών θα πρέπει να συνδυάζει τα δύο ακόλουθα θεμελιώδη χαρακτηριστικά επίδοσης:
\begin{enumerate}
\item \textbf{Αποτελεσματικότητα} :δηλαδή υψηλή αξιοπιστία εντοπισµού (ή προσέγγισης) του καθολικού ακροτάτου της συνάρτησης ή αντίστοιχα του καθολικά βέλτιστου υποσυνόλων χαρακτηριστικών.
\item \textbf{Αποδοτικότητα} :δηλαδή υψηλή ταχύτητα σύγκλισης (εγγυηµένος εντοπισµός του καθολικά βέλτιστου µε εύλογο πλήθος δοκιµών).
\end{enumerate}

Τα χαρακτηριστικά αυτά πολλές φορές είναι αντικρουόµενα αφού τεχνικές συστηµατικής αναζήτησης προσεγγίζουν το καθολικό βέλτιστο µε ακρίβεια, αλλά ταυτόχρονα απαιτούν και υψηλό υπολογιστικό κόστος, ενώ οι γρήγορες τεχνικές άµεσης αναζήτησης εγκλωβίζονται εύκολα σε τοπικά ακρότατα. Τα εξελικτικά σχήµατα βελτιστοποίησης και κατά συνέπεια και οι ΓΑ δείχνουν να υπερτερούν, στις περισσότερες κατηγορίες προβληµάτων βελτιστοποίησης. Μειονέκτηµά τους, είναι η ύπαρξη αρκετών παραµέτρων εισόδου, που επηρεάζουν σηµαντικά την επίδοση των αλγορίθµων (π.χ. µέγεθος πληθυσµού), ο ορισµός των οποίων απαιτεί πολλαπλούς πειραµατικούς ελέγχους µε διαφορετικές συνθήκες εκκίνησης.

Οι σημαντικότεροι λόγοι
χρησιµοποίησης των ΓΑ στο πρόβληµα της επιλογής χαρακτηριστικών είναι οι
ακόλουθοι:
\begin{enumerate}
\item Πραγµατοποιούν ταυτόχρονη εξερεύνηση σε πολλά διαφορετικά υποσύνολα (σηµεία του χώρου αναζήτησης) και εποµένως η εξαγωγή βέλτιστων λύσεων προκύπτει µέσα από ένα πλήθος διαφορετικών υποσυνόλων χαρακτηριστικών.
\item Για την εύρεση ενός καθολικά βέλτιστου υποσυνόλου χαρακτηριστικών οι ΓΑ ξεφεύγουν ως ένα βαθµό από τον εγκλωβισµό τους σε τοπικά βέλτιστα υποσύνολα χαρακτηριστικών. Αυτό γίνεται συνδυάζοντας υποσύνολα χαρακτηριστικών και δηµιουργώντας παράλληλα νέα µέσω των διαδικασιών αναπαραγωγής τουπληθυσµού. Έτσι προηγούµενες λύσεις που στην ουσία ενδεχοµένως να είχαν εγκλωβίσει τον αλγόριθµο σε περιοχές τοπικών βέλτιστων αποµακρύνονται.
\item Δουλεύουν µε µια κωδικοποίηση των δεδοµένων του προβλήµατος και όχι µε τα ίδια τα δεδοµένα.
\item Απαιτούν τη γνώση µόνο της αντικειµενικής συνάρτησης αξιολόγησης των υποσύνολων των χαρακτηριστικών.
\item Μπορούν να συνδυαστούν µε κάποια τοπική µέθοδος αναζήτησης αυξάνοντας την αποτελεσµατικότητα και την αποδοτικότητα τους σε δύσκολα υπολογιστικά προβλήµατα όπου η διάσταση του χώρου αναζήτησης είναι µεγάλη.
\end{enumerate}
\subsubsection{Δομή και Λειτουργία Γενετικού Αλγορίθμου}
Σε γενικές γραµµές η λειτουργία ενός ΓΑ καθορίζεται από τα ακόλουθα απλά
βήµατα:
\begin{enumerate}
\item Τυχαία αρχικοποίηση των µελών του αρχικού πληθυσµού.
\item Αξιολόγηση κάθε µέλους µε βάση τη δοθείσα συνάρτηση αξιολόγησης
καταλληλότητας.
\item  Εφαρµογή της διαδικασίας της επιλογής για κάθε µέλος του τρέχοντος πληθυσµού. Άτοµα µε µεγάλη τιµή στη συνάρτηση καταλληλότητας έχουν µεγαλύτερη πιθανότητα να επιλεγούν.
\item Εφαρµογή της διαδικασίας της διασταύρωσης για κάποια µέλη του πληθυσµού µε βάση την καθορισµένη πιθανότητα διασταύρωσης.
\item Εφαρµογή της διαδικασίας της µετάλλαξης για κάποια γονίδια των µελών του πληθυσµού µε βάση την καθορισµένη πιθανότητα µετάλλαξης.
\item Επιστροφή στο βήµα 2 και έλεγχος κριτηρίου τερµατισµού. Εάν δεν ικανοποιείται το κριτήριο τερµατισµού η διαδικασία επαναλαµβάνεται από το βήµα 2.
\end{enumerate}


Η κωδικοποίηση αποτελεί µια σηµαντική και απαραίτητη διαδικασία για έναν ΓΑ αφού ουσιαστικά συνδέει τον αλγόριθµο µε τα πραγµατικά δεδοµένα του προβλήµατος. Υπάρχουν διάφοροι τρόποι κωδικοποίησης οι οποίοι καθορίζονται από το είδος του προβλήµατος που θέλουµε να λύσουµε. Ο πιο δηµοφιλής και πιο απλός τρόπος, είναι η κωδικοποίηση των υποψήφιων λύσεων µέσω ενός διανύσµατος δυαδικών ψηφίων (binary bits). Στη δυαδική κωδικοποίηση κάθε παράµετρος µιας υποψήφιας λύσης κωδικοποιείται µέσω ενός γονιδίου, χρησιµοποιώντας ένα δυαδικο ψηφίο (0 ή 1). Ένας άλλος τρόπος κωδικοποίησης είναι µε χρήση πραγµατικών αριθµών όπου κάθε υποψήφια λύση κωδικοποιείται µέσω ενός διανύσµατος πραγµατικών αριθµών.

Η αρχικοποίηση των λύσεων σε µια σειρά προβληµάτων βελτιστοποίησης γίνεται µε τυχαίο τρόπο µε χρήση µιας γεννήτριας ισοκατανεµηµένων αριθµών 0 ή 1. Τα αντίστοιχα γονίδια κάθε υποψηφίας λύσης σχηµατίζουν ένα χρωµόσωµα/άτοµο το οποίο αποτελεί µια υποψήφια λύση για τον ΓΑ. Το πλήθος των χρωµοσωµάτων που αντιπροσωπεύει αυτόνοµες λύσεις του υποεξέταση προβλήµατος σχηµατίζει τον αρχικό πληθυσµό. Στην παρούσα εργασία και στα πλαίσια του υποεξέταση προβλήµατος διαπιστώσαµε πως η αρχικοποίηση των τιµών των δυαδικών ψηφίων των χρωµοσωµάτων παίζει έναν ιδιαίτερο και σηµαντικό ρόλο στην εύρεση της βέλτιστης λύσης, προσανατολίζοντας την αναζήτηση σε λύσεις µε το µικρότερο δυνατό αριθµό επιλεγµένων µεταβλητών. Κατά την εφαρµογή ενός απλού ΓΑ για την επιλογή χαρακτηριστικών σε δεδοµένα υψηλής διάστασης µεροληπτούµε κατά την αρχικοποίηση υπέρ των δυαδικών ψηφίων που φέρνουν την τιµή 0, δηλαδή αρχικά τα επιλεγµένα χαρακτηριστικά είναι ελάχιστα. Αυτό κρίνεται αναγκαίο, γιατί σε διαφορετική περίπτωση, ο αλγόριθµος δεν θα σύγκλινε ποτέ σε µια βέλτιστη λύση η οποία αποτελεί συνδυασµό της απόδοσης του ταξινοµητή και της διάστασης της βέλτιστης λύσης.

\label{fFunction}
Πολύ σημαντικό στοιχείο για την επίλυση είναι η \textbf{Συνάρτηση Καταλληλότητας} (fitness function). ∆έχεται ως είσοδο την αποκωδικοποιηµένη τιµή ενός χρωµοσώµατος και επιστρέφει έναν αριθµό που δηλώνει το βαθµό καταλληλότητας (fitness rate), του χρωµοσώµατος. Αποτελεί µέτρο της ποιότητας κάθε υποψήφιας λύσης. Οι τιµές της
συνάρτησης αξιολόγησης των ατόµων ορίζουν και τις αντίστοιχες πιθανότητες επιβίωσης οι οποίες θα χρησιµοποιηθούν στη φάση της επιλογής του πληθυσµού που θα αποτελέσει την επόµενη γενιά. Σε πολλές περιπτώσεις η τιµή της συνάρτησης αξιολόγησης αποτελεί και µέρος της συνθήκης τερµατισµού του αλγορίθµου.

\noindent\begin{minipage}{\linewidth}
\centering
\captionsetup{type={figure}}
\includegraphics{images/genetics}
\captionof{figure}{Genetics}
\label{fig:Genetics}
\end{minipage}

Η διαδικασία της επιλογής καθορίζει ποια από τα άτοµα του τρέχοντος πληθυσµού θα συνεχίσουν την εξελεγκτική διαδικασία περνώντας στη φάση της αναπαραγωγής για να κληροδοτήσουν στην επόµενη γενιά κάποια η όλα τα χαρακτηριστικά τους. Για την διαδικασία της επιλογής υπάρχουν διάφορες τεχνικές. ∆ηµοφιλέστερες είναι η τεχνική της \textbf{Ρουλέτας} \label{roul} (Roulette Wheel Selection) και η ταξινοµηµένη επιλογή (Rank Selection). Η µέθοδος του τροχού της τύχης (Ρουλέτα) αποτελεί µια στοχαστική διαδικασία δειγµατοληψίας µε αντικατάσταση και υλοποιείται ως εξής:
Η επιλογή των ατόµων αυτών γίνεται µε βάση την τιµή της καταλληλότητας τους. Όσο µεγαλύτερη είναι η τιµή της καταλληλότητας ενός ατόµου σε σχέση µε την τιµή των άλλων µελών τόσο αυξάνεται η πιθανότητα να επιλεγεί το άτοµο αυτό µία ή περισσότερες φορές. Οι τιµές καταλληλότητας του πληθυσµού κανονικοποιηµένες στο διάστηµα [0,1] παριστάνονται από ισάριθµα διαδοχικά ευθύγραµµα τµήµατα των οποίων το µήκος είναι ανάλογο της τιµής της καταλληλότητας (απόδοσης) του κάθε ατόµου. Τα ευθύγραµµα τµήµατα έχουν συνολικό µέγεθος 1.Έτσι σε ένα µέλος του πληθυσµού µε µεγάλη απόδοση θα αντιστοιχεί και µεγαλύτερο ευθύγραµµο τµήµα από ένα µέλος µε µικρότερη απόδοση. Αυτό σηµαίνει ότι θα έχει και µεγαλύτερη πιθανότητα να επιλεγεί. Στη συνέχει παράγεται ένας τυχαίος αριθµός στο διάστηµα [0,1] και το ευθύγραµµο τµήµα στο οποίο ανήκει µας φανερώνει και το αντίστοιχο µέλος του πληθυσµού που θα επιλεγεί. Η διαδικασία αυτή επαναλαµβάνεται µέχρι να επιλεγεί ο επιθυµητός αριθµός ατόµων που θα συµµετάσχουν στην αναπαραγωγική διαδικασία.
\label {cross} Η \textbf{διασταύρωση} αποτελεί µια σηµαντική λειτουργία των ΓΑ. Ο πληθυσµός που
προέκυψε µετά τη διαδικασία επιλογής αποτελεί τη δεξαµενή ζευγαρώµατος από όπου επιλέγονται τα άτοµα που θα συνδυαστούν ώστε να δηµιουργηθούν νέα άτοµα. Στόχος της διασταύρωσης είναι ο νέος πληθυσµός που θα προκύψει µετά την εφαρµογή της να είναι καλύτερος από τον προηγούµενο και τα άτοµα-απόγονοι να φέρουν τα καλύτερα χαρακτηριστικά των γονέων τους. Υπάρχουν διάφοροι τρόποι διασταύρωσης. Ο πιο απλός τρόπος είναι η διασταύρωση ενός σηµείου (Single Point Crossover). Σύµφωνα µε αυτόν τον τρόπο για κάθε ζεύγος χρωµοσωµάτων που επιλέχθηκε για διασταύρωση παράγεται τυχαία ένας ακέραιος k από το διάστηµα [1,m-1] όπου m το µήκος του δυαδικού ψηφίου σε χρωµοσώµατα. Ο αριθµός k προσδιορίζει το σηµείο κοπής ή αλλιώς σηµείο διασταύρωσης των χρωµοσωµάτων. Από το σηµείο αυτό, µέχρι και το τέλος του µήκους των χρωµοσωµάτων γίνεται ανταλλαγή των γονιδίων τους. Άλλοι τρόποι διασταύρωσης είναι η διασταύρωση πολλών σηµείων (Multi Point Crossover) και η οµοιόµορφη διασταύρωση (Uniform Crossover). Στη διασταύρωση πολλών σηµείων για κάθε ζεύγος χρωµοσωµάτων που επιλέχθηκε για διασταύρωση παράγεται τυχαία n ακέραιοι ki από το διάστηµα [1,m-1] όπου m το µήκος σε δυαδικά ψηφία των χρωµοσωµάτων. Οι αριθµοί ki προσδιορίζουν τα σηµεία κοπής ή αλλιώς σηµεία διασταύρωσης των χρωµοσωµάτων. Τα γονίδια των γονέων µεταξύ των διαδοχικών σηµείων διασταύρωσης ανταλλάσσονται για να παράγουν τα παιδιά τους. Στη οµοιόµορφη διασταύρωση οι δύο γονείς θα ανταλλάξουν το γενετικό υλικό
µε βάση µια µάσκα από δυαδικά ψηφία. Η µάσκα αυτή είναι µήκους m όσο το µήκος των χρωµοσωµάτων και επιλέγεται µε τυχαίο τρόπο. Ανάλογα µε την τιµή του δυαδικού ψηφίου στην κάθε θέση της µάσκας καθορίζεται για κάθε παιδί από ποιόν γονέα θα προέρχεται το γενετικό υλικό στην αντίστοιχη θέση. Για παράδειγµα αν το δυαδικό ψηφίο στην 1η θέση της µάσκας έχει τιµή 1 τότε το 1 ο παιδί θα πάρει την τιµή της αντίστοιχης θέση του 1ου γονέα αλλιώς αν είναι 0 τότε θα θα πάρει την τιµή της αντίστοιχης θέση του 2ου γονέα. Για το 2 ο παιδί ισχύει το αντίστροφο, δηλαδή από όποιον γονέα πάρει γενετικό υλικό το 1ο παιδί, το 2 ο παιδί θα πάρει από τον
άλλον γονέα.

\noindent\begin{minipage}{\linewidth}
\centering
\captionsetup{type={figure}}
\includegraphics{images/crossover}
\captionof{figure}{Διασταύρωση (Crossover) }
\label{fig:Crossover}
\end{minipage}

Οι ΓΑ εξαρτώνται από ένα πλήθος παραµέτρων το οποίο προσαρµόζεται ανάλογα µε το πρόβληµα που επιλύεται κάθε φορά. Το πρόβληµα είναι ότι η ανάθεση τιµών σε αυτές τις παραµέτρους δεν καθορίζεται από κάποιον κανόνα αλλά είναι αποτέλεσµα πειραµατικών µελετών για κάθε είδος προβλήµατος. Οι βασικές παράµετροι είναι η πιθανότητα διασταύρωσης (crossover probability), η \label{muta} \textbf{πιθανότητα µετάλλαξης} (mutation probability) και το µέγεθος του πληθυσµού (population size). Η πιθανότητα διασταύρωσης (pc) καθορίζει τη συχνότητα της διασταύρωσης, δηλαδή πόσα µέλη του πληθυσµού που επεξεργάζεται ο ΓΑ θα διασταυρωθούν. Η πιθανότητα αυτή ποικίλει ανάλογα µε το είδος του προβλήµατος. Πιθανότητα διασταύρωσης ίση µε 1 σηµαίνει όλα τα µέλη του πληθυσµού θα διασταυρωθούν µεταξύ τους, ενώ αν είναι 0, τότε οι απόγονοι θα είναι πιστά αντίγραφα των γονέων εκτός αν συµβούν αλλαγές στη φάση της µετάλλαξης. Η πιθανότητα µετάλλαξης (pm) καθορίζει το πόσο συχνά τα γονίδια των χρωµοσωµάτων θα αλλάζουν κατάσταση (από 0 σε 1 ή το αντίστροφο). Η τιµή της πιθανότητας µετάλλαξης θα πρέπει να είναι 1/n όπου n ο αριθµός των µεταβλητών παραµέτρων του υποεξέταση προβλήµατος. Σε γενικές γραµµές το ποσοστό της µετάλλαξης θα πρέπει να είναι χαµηλό, αλλιώς ο αλγόριθµος εγκλωβίζεται σε ένα βρόγχο τυχαίας αναζήτησης.
∆ιάφορες µελέτες έχουν γίνει για τη σωστή προσέγγιση του ποσοστού της µετάλλαξης. Ο Holland υποστήριξε ότι η µετάλλαξη είναι δευτερογενής τελεστής ενώ ο Goldberg προτείνει να αντιστρέφεται ένα στα χίλια δυαδικά ψηφία κατά µέσο όρο σε κάθε επανάληψη. Η µετάλλαξη αντιµετωπίζει τα δυαδικά ψηφία όλων των µελών του πληθυσµού σαν µια ενωµένη συµβολοσειρά και η αναφορά του Goldberg µιλάει για το σύνολο των δυαδικών ψηφίων του πληθυσµού. Το µέγεθος του πληθυσµού δηλώνει τον αριθµό των υποψηφίων λύσεων κάθε γενιάς. Ο καθορισµός του µεγέθους αυτού είναι συνάρτηση του είδους του προβλήµατος που θα επιλυθεί και των διαθέσιµων υπολογιστικών πόρων. Το µέγεθος του πληθυσµού είναι κρίσιµο για την εύρεση της βέλτιστης λύσης Οι µικροί πληθυσµοί συγκλίνουν πιο γρήγορα σε τοπικά βέλτιστα αλλά εγκλωβίζονται σε αυτά, ενώ οι µεγάλοι πληθυσµοί είναι πολύ πιθανόν να µην εγκλωβίσουν τον αλγόριθµο σε τοπικό βέλτιστα αλλά θέλουν περισσότερο υπολογιστικό χρόνο και πόρους για την εξεύρεση της λύσης.

\noindent\begin{minipage}{\linewidth}
\centering
\captionsetup{type={figure}}
\includegraphics{images/mutates}
\captionof{figure}{Μετάλλαξη (Mutates)}
\label{fig:Mutates}
\end{minipage}

\subsection{Πειραματικά Αποτελέσματα}
Ο λόγος που προχωρήσαμε σε μια τόσο μεγάλη εισαγώγη σχετικά με τους Γενετικούς Αλγορίθμους έχει να κάνει με το γεγονός ότι αυτοί δεν είναι αντικείμενο του μαθήματος Αναγνώριση Προτύπων και πιθανόν να υπάρχουν σημεία που δεν είναι πλήρως κατανοητά από τους φοιτητές που μελετάν το υπάρχων κείμενο. Η λογική που θα χρησιμοποιήσουμε είναι πολύ απλή. Ουσιαστικά υλοποιούμε ένα πρόβλημα ελαχιστοποίησης ως προς το κέντρο των ομάδων. Η συνάρτηση $fitness function$ ορίζεται ως η απόσταση των σημείων μας από το κέντρο της ομάδας στην οποία ανήκει. H απόσταση ορίζεται όπως και στην προηγούμενη περίπτωση σαν $cosine$ και $correlation$.

Τρέχουμε στο Matlab το αρχείο \lstinline[language=MATLAB]!geneticsa! το οποίο υλοποιεί τον γενετικό αλγόριθμο για την ομαδοποίηση με απόσταση ορισμένη σαν $correlation$ ή $cosine$. Δίνεται η επιλόγη στον χρήστη να επιλέξει. Το αρχείο \lstinline[language=MATLAB]!geneticsa! καλεί της παρακάτω συναρτήσεις
\begin{enumerate}
\item \lstinline[language=MATLAB]!ClusteringCost(m, X)!
\ref{fFunction}{Συνάρτηση κόστους} που έχει ορισθεί παραμετρικά ως προς το κέντρο m των ομάδων. Η αρχικοποίηση του κέντρου έγινε τυχαία.
\item \lstinline[language=MATLAB]!Crossover! : \ref{cross} Συνάρτηση που υλοποιεί την διασταύρωση.
\item  \lstinline[language=MATLAB]!data2! : Script το οποίο φορτώνει τα dataset.
\item  \lstinline[language=MATLAB]!Mutate! : \ref{muta} Συνάρτηση που υλοποιεί την μετάλλαξη.
\item \lstinline[language=MATLAB]!RouletteWheelSelection! \ref{roul} Συνάρτηση που υλοποιεί την διαδικασία της επιλογής.
\item \lstinline[language=MATLAB]!PlotSolution! : Συνάρτηση που μας δίνει το συνολικό ελάχιστο κόστος.
\end{enumerate}  

Επιλέξαμε σαν τιμές παραμέτρων $Population Size =100$, $percentage crossover=0.8$,$mutation percentage=0.3$ και $Maximum Iterations=1500$.

Ο κώδικας geneticsa και όλες οι συναρτήσεις βρέθηκαν έτοιμες και αναφέρονται τα απαραίτητα στοιχεία στον κώδικα. Τροποποιήθηκαν κάποιες αλλαγές για να προσαρμοστεί ο κώδικας στα δικά μας δεδομένα.

Τρέξαμε τον γενετικό αλγόριθμο για 2 από τα παραπάνω dataset για αποστάσεις $cosine$ και $correlation$ και παρατηρήσαμε ότι τα αποτελέσματα για το $cosine$ είναι ελαφρώς καλύτερα και για αυτό σας παρουσιάζουμε μόνο αυτά. Σκοπός μας δεν είναι να επιλέξουμε το βέλτιστο μοντέλο μέσω γενετικού αλγορίθμου, καθώς δεν ξέρουμε κατά πόσο μπορούν να είναι αποδοτικοί οι γενετικοί αλγόριθμοι στην ομαδοποίηση, αλλά να παρουσιάσουμε μια καινοτόμα ιδέα στα πλαίσια του μαθήματος. Αναφέρεται στα ανοιχτά θέματα ότι επιδέχεται βελτιώσεις η όλη διαδικασία.

Η \textbf{1η μέτρηση} που κάναμε ήταν στο \url{dataset3_70} με απόσταση $cosine$.

\noindent\begin{minipage}{\linewidth}
Παρακάτω παρουσιάζεται το διάγραμμα του ελάχιστου κόστους. Παρατηρούμε ότι στο τέλος η μεταβολή τείνει να γίνει παράλληλη στον οριζόντιο άξονα. Αυτό οφείλεται στο γεγονός ότι μεταβάλλεται πιο αργά το ελάχιστο κόστος. Αν αφήναμε και άλλο τον αλγόριθμο να τρέξει θα παίρναμε καλύτερη εικόνα.
\centering
\captionsetup{type={figure}}
\includegraphics[width=\linewidth]{images/BestCost3_70_cosine}
\captionof{figure}{Best Cost Case 1}
\label{fig:Besto Cost}
\end{minipage}

\noindent\begin{minipage}{\linewidth}
Παρακάτω παρουσιάζεται το διάγραμμα $Silhouette της πρώτης ομαδοποίησης$
\centering
\captionsetup{type={figure}}
\includegraphics[width=\linewidth]{images/Silhouete3_70_cosine}
\captionof{figure}{Silhouette }
\end{minipage}

Η \textbf{2η μέτρηση} που κάναμε ήταν στο \url{dataset8_50} με απόσταση $cosine$.

Αρχικά βλέπουμε το ελάχιστο κόστος. Οι παρατηρήσεις μας είναι ίδιες με πριν. Είναι εμφανές ότι χρειαζόταν περισσότερες επαναλήψεις χωρίς ωστόσο να ξέρουμε αν το $Success Rate$ θα βελτιωθεί.

\noindent\begin{minipage}{\linewidth}
\centering
\captionsetup{type={figure}}
\includegraphics[width=\linewidth]{images/BestCost8_50_cosine}
\captionof{figure}{Best Cost }
\label{fig:Best Cost}
\end{minipage}

\noindent\begin{minipage}{\linewidth}
Τέλος παίρνουμε το διάγραμμα $Silhouette$ της δεύτερης ομαδοποίησης.
\centering
\captionsetup{type={figure}}
\includegraphics[width=\linewidth]{images/Silhouete8_50_cosine}
\captionof{figure}{Silhouette }
\label{fig:Silhouette}
\end{minipage}

Παρουσιάζεται στον
\hyperref[table:genetics]{Πίνακα \ref{table:genetics}}
τα αποτελέσματα των 2 ομαδοποιήσεων.

\begin{table}[htb]
\centering
\resizebox{\textwidth}{!}{%
\begin{tabular}{lllllll}
\hline
Dataset      & Distance Type & Number of Clusters & Cohesion & Separation & Silhouette & Success Rate \\ \hline
dataset3\_70 & Cosine        & 8                  & 674.4    & 4949       & 0.2        & 0.75         \\
dataset8\_50 & Cosine        & 8                  & 927.4000 & 46964      & 0.1738     & 0.6875      
\end{tabular}
}
\caption{Ομαδοποίηση με γενετικούς}\label{table:genetics}
\end{table}

Παρατηρούμε ότι οι μετρικές μας είναι πολύ χειρότερες. Θα μπορούσα να τις βελτιώσουμε αυξάνοντας τον αριθμό των μέγιστων επαναλήψεων αλλά κάτι τέτοιο είναι χρονοβόρο για την ώρα και τον σκοπό της εργασίας.

