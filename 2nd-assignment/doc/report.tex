%&preamble
\usepackage[linesnumbered, boxruled]{algorithm2e}
\SetKwRepeat{Do}{do}{while}

% automatic hyphenation for 2 languages
% http://www.mechpedia.gr/wiki/Hyphenation_-_%CE%A5%CF%86%CE%B5%CE%BD%CF%8E%CF%83%CE%B5%CE%B9%CF%82#.CE.91.CF.85.CF.84.CF.8C.CE.BC.CE.B1.CF.84.CE.B5.CF.82_.CF.85.CF.86.CE.B5.CE.BD.CF.8E.CF.83.CE.B5.CE.B9.CF.82_.CF.83.CE.B5_.CE.B4.CE.AF.CE.B3.CE.BB.CF.89.CF.83.CF.83.CE.B1_.CE.BA.CE.B5.CE.AF.CE.BC.CE.B5.CE.BD.CE.B1
% very slow, enable only at final pdf.
%TODO: \usepackage[Greek,Latin]{ucharclasses}
%TODO: \setTransitionsForGreek{\selectlanguage{greek}}{\selectlanguage{english}}
% Save static part as preamble.tex and use command:
% xelatex -ini -shell-escape -job-name="preamble" "&xelatex preamble.tex\dump"
% to produce preamble.fmt

% polyglossia
\usepackage{polyglossia}
\setmainlanguage{greek}
\setotherlanguages{english}

% Fonts
% fonts can't go in the .fmt file
\usepackage{fontspec}
\setmainfont[Mapping=tex-text]{DejaVu Sans}
\newfontfamily\greekfont[Script=Greek]{DejaVu Sans}
\newfontfamily\greekfontsf[Script=Greek]{DejaVu Sans}
\setmonofont[Scale=1.0]{Source Code Pro Medium}		
\newfontfamily\greekfonttt[Scale=1.0]{Source Code Pro Medium}
\usepackage{microtype} % microtype is font-dependant

\title{Αναγνώριση Προτύπων\\
Δεύτερη Εργασία Μαθήματος – Clustering}
\author{
  Παυλίδης Αντώνης, 7758 \href{mailto:ant\_pavlidis@yahoo.gr}{ant\_pavlidis@yahoo.gr}\\
  Φλώρος-Μαλιβίτσης Ορέστης, 7796 \href{mailto:orestisf@ece.auth.gr}{orestisf@ece.auth.gr}\\
  Χαμζάς Κωνσταντίνος, 7798 \href{mailto:chamzask@ece.auth.gr}{chamzask@ece.auth.gr}\\
  \\\\ \textbf{Τομέας Ηλεκτρονικής}\\
  \textbf{Τμήμα Ηλ. Μηχανικών / Μηχανικών ΗΥ}\\
  \textbf{Αριστοτέλειο Πανεπιστήμιο Θεσσαλονίκης}}
\titlepic{\includegraphics[width=0.40\textwidth]{images/university}}
\renewcommand{\labelenumii}{\arabic{enumii}.}
\renewcommand{\labelenumiii}{\arabic{enumiii}.}
\begin{document}
\maketitle
\tableofcontents
\listoffigures
\listoftables

\chapter{Εισαγωγή}
\section{Περιγραφή του προβλήματος}

Το πρόβλημα που καλούμαστε να λύσουμε είναι η πρόβλεψη σφαλμάτων (bugs) σε έργα λογισμικού.
Ένας τρόπος αντιμετώπισης είναι η χρήση διάφορων μερτικών πάνω στα τμήματα που έχουν και που δεν έχουν σφάλμα. Στο συγκεκριμένο project μας δίνεται ένα σύνολο από κλάσεις με υπολογισμένες μετρικές και γνώση αν περιέχουν σφάλμα ή όχι (training set). Με την χρήση αυτών των δεδομένων θα δημιουργήσουμε διάφορα μοντέλα ταξινόμησης που θα ταξινομούν τις κλάσεις σε εσφαλμένες και μη.

Εδώ πρέπει να ορίσουμε με ποια κριτήρια θα κρίνουμε αν ένα μοντέλο είναι καλό ή όχι. Για αυτό θα λάβουμε υπόψιν 2 παράγοντες.
\begin{enumerate}
\item Θεωρούμε σημαντικό να βρεθούν όσο το δυνατόν περισσότερο bugs από αυτά που υπάρχουν (recall) καθώς η διόρθωση σε αρχικά στάδια ενός έργου λογισμικού είναι πολύ πιο φθηνή.
\item Θα διατεθούν πόροι για κάθε κλάση που θα θεωρήσει το μοντέλο ταξινόμησης μας εσφαλμένη ώστε είτε να επιδιορθωθεί το bug ή να εξακριβωθεί ότι δεν περιέχει κάποιο bug. Συνεπώς, είναι σημαντικό το ποσοστό των  σωστά προβλεμένων εσφαλμένων κλάσεων προς των συνολικών προβλεμένων εσφαλμέων κλάσεων (precision) να είναι υψηλό.
\end{enumerate}

Συνολικά λοιπόν θα δώσουμε βάση στην μετρική f-measure που συνδυάζει και το recall και το precision. Ακόμη, θα δώσουμε περισσότερη έμφαση στο recall σε σχέση με το precision γιατί θεωρούμε ότι είναι πιο οικονομικό να διορθωθούν πολλά bugs στην αρχή παρά στο λίγα στο τέλος. 

\section{Αλγόριθμοι που χρησιμοποιήθηκαν}
Οι αλγόριθμοι ταξινόμησης που χρησιμοποιήσαμε ήταν :
\begin{enumerate}
  \item \textbf{Bayes:}
  \begin{itemize}
     \item NaiveBayes
     \item BayesNet
  \end{itemize}  
  \item \textbf{Trees:}
  \begin{itemize}
     \item J48
     \item RandomForest
  \end{itemize}
  \item \textbf{SVM:}  
  \begin{itemize}
     \item C-SVC (libsvm)
     \item SMO
  \end{itemize}
  \item \textbf{NNs:}
  \begin{itemize}
     \item KNN
     \item NN
  \end{itemize}
\end{enumerate}

\section{Εργαλεία που χρησιμοποιήθηκαν}

\begin{enumerate}
  \item \textbf{Weka}
  \begin{itemize}
     \item TODO
  \end{itemize}  
  \item \textbf{Python:}
\begin{itemize}
     \item TODO
\end{itemize}
\end{enumerate}
\chapter{Προεπεξεργασία Δεδομένων}
\section{Μορφή των δεδομένων}
Τα δεδομένα μας προέρχονται από τους όρους-λέξεις 80 βιβλιοθηκών της γλώσσας προγραμματισμού Java που ανήκουν σε 8 κατηγορίες:
\begin{enumerate}\label{itemize:categories}
    \item android
    \item command-line-parsers
    \item csv-libraries
    \item http-clients
    \item json-libraries
    \item swing-libraries
    \item testing-frameworks
    \item xml-processing
\end{enumerate}

Το αρχικό μας dataset περιλαμβάνει $109706$ λέξεις οι οποίες παρουσιάζονται σαν στήλες σε ένα αρχείο \texttt{.csv}.
Κάτω από την καθεμία αναγράφονται οι απόλυτες συχνότητες εμφάνισης του κάθε όρου.
Επίσης, δίνονται οι στήλες \texttt{project}, κάτω από την οποία είναι τα ονόματα της κάθε βιβλιοθήκης, και \texttt{category}, κάτω από την οποία είναι η
\hyperref[itemize:categories]{κατηγορία} που ανήκει η κάθε βιβλιοθήκη.

\begin{center}
\captionsetup{type={table}}
\resizebox{\textwidth}{!}{%
\begin{tabular}{|c|c|c|c|c|c|c|c|c|c|}
\hline
project & category & ACONST & CompoundButtonCheckedChangeOnSubscribe & StatisticsComponent & getProjection & hashedSignature & propertyDescs & testGetBound & testReadPaths \\ \hline
com.appnexus.opensdk/appnexus-sdk & android & 0 & 0 & 0 & 0 & 0 & 0 & 0 & 0 \\
com.auth0.android/lock & android & 0 & 0 & 0 & 0 & 0 & 0 & 0 & 0 \\
com.bingzer.android.ads/adrunner & android & 0 & 0 & 0 & 0 & 0 & 0 & 0 & 0 \\
com.daimajia.easing/library & android & 0 & 0 & 0 & 0 & 0 & 0 & 0 & 0 \\
com.facebook.android/facebook-android-sdk & android & 0 & 0 & 0 & 0 & 2 & 0 & 0 & 0 \\
com.facebook.fresco/fbcore & android & 0 & 0 & 0 & 0 & 0 & 0 & 0 & 0 \\
com.facebook.fresco/fresco & android & 0 & 0 & 0 & 0 & 0 & 0 & 0 & 0 \\
com.github.asne/asne-core & android & 0 & 0 & 0 & 0 & 0 & 0 & 0 & 0 \\
com.github.castorflex.smoothprogressbar/library & android & 0 & 0 & 0 & 0 & 0 & 0 & 0 & 0 \\
com.hannesdorfmann.mosby/mvp & android & 0 & 0 & 1 & 0 & 0 & 0 & 0 & 0 \\
com.jakewharton.rxbinding/rxbinding & android & 0 & 3 & 0 & 0 & 0 & 0 & 0 & 0 \\
com.jakewharton.timber/timber & android & 0 & 0 & 0 & 0 & 0 & 0 & 0 & 0 \\
com.jdroidframework/jdroid-android & android & 0 & 0 & 0 & 3 & 0 & 0 & 0 & 0 \\
com.joanzapata.iconify/android-iconify & android & 0 & 0 & 0 & 0 & 0 & 0 & 0 & 0 \\
com.mixpanel.android/mixpanel-android & android & 0 & 0 & 0 & 0 & 0 & 3 & 0 & 1 \\
com.rengwuxian.materialedittext/library & android & 0 & 0 & 0 & 0 & 0 & 0 & 0 & 0 \\
com.shamanland/xdroid-core & android & 0 & 0 & 0 & 0 & 0 & 0 & 0 & 0 \\
com.stanfy.enroscar/enroscar-beans & android & 0 & 0 & 0 & 0 & 0 & 0 & 0 & 0 \\
com.stanfy.enroscar/enroscar-sdk-dep & android & 0 & 0 & 0 & 0 & 0 & 0 & 0 & 0 \\
com.stanfy.enroscar/enroscar-shared & android & 0 & 0 & 0 & 0 & 0 & 0 & 0 & 0 \\
de.quist.apps.maps/android-maps-abstraction & android & 0 & 0 & 0 & 6 & 0 & 0 & 0 & 0 \\
eu.inmite.android.lib/android-styled-dialogs & android & 0 & 0 & 0 & 0 & 0 & 0 & 0 & 0 \\
io.palaima.debugdrawer/debugdrawer & android & 0 & 0 & 0 & 0 & 0 & 0 & 0 & 0 \\
io.reactivex/rxandroid & android & 0 & 0 & 0 & 0 & 0 & 0 & 0 & 0 \\
args4j/args4j & command-line-parsers & 0 & 0 & 0 & 0 & 0 & 0 & 0 & 0 \\
com.beust/jcommander & command-line-parsers & 0 & 0 & 0 & 0 & 0 & 0 & 0 & 0 \\
com.github.spullara.cli-parser/cli-parser & command-line-parsers & 0 & 0 & 0 & 0 & 0 & 0 & 0 & 0 \\
com.lexicalscope.jewelcli/jewelcli & command-line-parsers & 0 & 0 & 0 & 0 & 0 & 0 & 0 & 0 \\
commons-cli/commons-cli & command-line-parsers & 0 & 0 & 0 & 0 & 0 & 0 & 0 & 0 \\
gnu.getopt/java-getopt & command-line-parsers & 0 & 0 & 0 & 0 & 0 & 0 & 0 & 0 \\
io.airlift/airline & command-line-parsers & 0 & 0 & 0 & 0 & 0 & 0 & 0 & 0 \\
net.sf.jopt-simple/jopt-simple & command-line-parsers & 0 & 0 & 0 & 0 & 0 & 0 & 0 & 0 \\
net.sourceforge.argparse4j/argparse4j & command-line-parsers & 0 & 0 & 0 & 0 & 0 & 0 & 0 & 0 \\
org.cyclopsgroup/jcli & command-line-parsers & 0 & 0 & 0 & 0 & 0 & 0 & 0 & 0 \\
org.realityforge.getopt4j/getopt4j & command-line-parsers & 0 & 0 & 0 & 0 & 0 & 0 & 0 & 0 \\
au.com.bytecode/opencsv & csv-libraries & 0 & 0 & 0 & 0 & 0 & 0 & 0 & 0 \\
com.fasterxml.jackson.dataformat/jackson-dataformat-csv & csv-libraries & 0 & 0 & 0 & 0 & 0 & 0 & 0 & 0 \\
com.opencsv/opencsv & csv-libraries & 0 & 0 & 0 & 0 & 0 & 0 & 0 & 0 \\
net.sf.flatpack/flatpack & csv-libraries & 0 & 0 & 0 & 0 & 0 & 0 & 0 & 0 \\
net.sf.supercsv/super-csv-dozer & csv-libraries & 0 & 0 & 0 & 0 & 0 & 0 & 0 & 0 \\
net.sf.supercsv/super-csv & csv-libraries & 0 & 0 & 0 & 0 & 0 & 0 & 0 & 0 \\
org.apache.commons/commons-csv & csv-libraries & 0 & 0 & 0 & 0 & 0 & 0 & 0 & 0 \\
org.beanio/beanio & csv-libraries & 0 & 0 & 0 & 0 & 0 & 0 & 0 & 0 \\
org.jdbi/jdbi & csv-libraries & 0 & 0 & 0 & 0 & 0 & 0 & 0 & 0 \\
org.onebusaway/onebusaway-csv-entities & csv-libraries & 0 & 0 & 0 & 0 & 0 & 0 & 0 & 0 \\
com.github.kevinsawicki/http-request & http-clients & 0 & 0 & 0 & 0 & 0 & 0 & 0 & 0 \\
com.google.http-client/google-http-client & http-clients & 0 & 0 & 0 & 0 & 0 & 0 & 1 & 0 \\
com.jcabi/jcabi-http & http-clients & 0 & 0 & 0 & 0 & 0 & 0 & 0 & 0 \\
com.m3/curly & http-clients & 0 & 0 & 0 & 0 & 0 & 0 & 0 & 0 \\
com.mashape.unirest/unirest-java & http-clients & 0 & 0 & 0 & 0 & 0 & 0 & 0 & 0 \\
com.metamx/http-client & http-clients & 0 & 0 & 0 & 0 & 0 & 0 & 0 & 0 \\
com.ning/async-http-client & http-clients & 0 & 0 & 0 & 0 & 0 & 0 & 0 & 0 \\
com.squareup.okhttp/okhttp & http-clients & 0 & 0 & 0 & 0 & 0 & 0 & 0 & 0 \\
org.apache.httpcomponents/httpasyncclient & http-clients & 0 & 0 & 0 & 0 & 0 & 0 & 0 & 0 \\
org.apache.httpcomponents/httpclient & http-clients & 0 & 0 & 0 & 0 & 0 & 0 & 0 & 0 \\
com.alibaba/fastjson & json-libraries & 5 & 0 & 0 & 0 & 0 & 0 & 0 & 0 \\
com.eclipsesource.minimal-json/minimal-json & json-libraries & 0 & 0 & 0 & 0 & 0 & 0 & 0 & 0 \\
com.fasterxml.jackson.core/jackson-core & json-libraries & 0 & 0 & 0 & 0 & 0 & 0 & 0 & 0 \\
com.google.code.gson/gson & json-libraries & 0 & 0 & 0 & 0 & 0 & 0 & 0 & 0 \\
com.googlecode.json-simple/json-simple & json-libraries & 0 & 0 & 0 & 0 & 0 & 0 & 0 & 0 \\
com.jayway.jsonpath/json-path & json-libraries & 0 & 0 & 0 & 0 & 0 & 0 & 0 & 0 \\
net.minidev/json-smart & json-libraries & 4 & 0 & 0 & 0 & 0 & 0 & 0 & 0 \\
org.codehaus.jettison/jettison & json-libraries & 0 & 0 & 0 & 0 & 0 & 0 & 0 & 0 \\
org.json/json & json-libraries & 0 & 0 & 0 & 0 & 0 & 0 & 0 & 0 \\
com.github.rickyclarkson/swingflow & swing-libraries & 0 & 0 & 0 & 0 & 0 & 0 & 0 & 0 \\
net.sf.cssbox/swingbox & swing-libraries & 0 & 0 & 0 & 0 & 0 & 0 & 0 & 0 \\
org.buildsomethingawesome.lib/awesome-java-swing & swing-libraries & 0 & 0 & 0 & 0 & 0 & 0 & 0 & 0 \\
org.fuin/utils4swing & swing-libraries & 0 & 0 & 0 & 0 & 0 & 0 & 0 & 0 \\
org.scijava/swing-checkbox-tree & swing-libraries & 0 & 0 & 0 & 0 & 0 & 0 & 0 & 0 \\
org.softsmithy.lib/softsmithy-lib-swing & swing-libraries & 0 & 0 & 0 & 0 & 0 & 0 & 0 & 0 \\
org.swinglabs/swingx & swing-libraries & 0 & 0 & 0 & 0 & 0 & 0 & 0 & 0 \\
com.novocode/junit-interface & testing-frameworks & 0 & 0 & 0 & 0 & 0 & 0 & 0 & 0 \\
junit/junit & testing-frameworks & 0 & 0 & 0 & 0 & 0 & 0 & 0 & 0 \\
org.testng/testng & testing-frameworks & 0 & 0 & 0 & 0 & 0 & 0 & 0 & 0 \\
xmlunit/xmlunit & testing-frameworks & 0 & 0 & 0 & 0 & 0 & 0 & 0 & 0 \\
com.fasterxml/aalto-xml & xml-processing & 0 & 0 & 0 & 0 & 0 & 0 & 0 & 0 \\
dom4j/dom4j & xml-processing & 0 & 0 & 0 & 0 & 0 & 0 & 0 & 0 \\
jdom/jdom & xml-processing & 0 & 0 & 0 & 0 & 0 & 0 & 0 & 0 \\
net.sf.kxml/kxml2 & xml-processing & 0 & 0 & 0 & 0 & 0 & 0 & 0 & 0 \\
xstream/xstream & xml-processing & 0 & 0 & 0 & 0 & 0 & 0 & 0 & 0 \\ \hline
\end{tabular}
}
\captionof{table}{Παράδειγμα μερικών στηλών για όλες τις γραμμές του αρχικού dataset}
\label{table:init}
\end{center}

\section{Εξερεύνηση Δεδομένων}
Πριν την διαδικασία της προεπεξεργασίας των δεδομένων, πραγματοποιούμε μία αρχική εξερεύνηση τους για την κατανόηση του προβλήματος ομαδοποίησης που καλούμαστε να αντιμετωπίσουμε.

Το αρχικό σύνολο δεδομένων, που έχουμε στη διάθεσή μας, παρατηρούμε ότι περιλαμβάνει έναν πολύ μεγάλο αριθμό λέξεων που αποτελούν ουσιαστικά τα attributes για τους αλγορίθμους μας.
Για αυτό το λόγο, στόχος μας ήταν η δραστική μείωση του αριθμού αυτών των λέξεων ενώ αποφασίσαμε ότι δεν ήταν σημαντικό να αφαιρέσουμε κάποια βιβλιοθήκη (γραμμή) καθώς ο αριθμός τους ήταν ήδη περιορισμένος.

Επίσης, παρατηρήσαμε ότι υπήρχαν διάφορα προβλήματα πάνω στην ποιότητα των λέξεων του dataset:
\begin{itemize}
    \item Πολλές λέξεις δεν είχαν κάποιο νόημα πχ
    % manual hyphenation because it didn't work with anything else.
    \texttt{ABCDE\-FGHIJKLMNO\-PQRSTU\-VWXYZ\-abc\-d\-e\-f\-ghij\-klmnopqrstuvwxyz}
    που λογικά χρησιμοποιείται ως ένα string που περιέχει όλους τους χαρακτήρες της Αγγλικής γλώσσας.

    \item Πολλές λέξεις υπήρχαν σε διάφορα σημεία αλλά λόγω της διάκρισης κεφαλαίων και πεζών θεωρούνταν ξεχωριστές λέξεις.
    πχ \texttt{ACCESS}, \texttt{access} και \texttt{Access}.

    \item Μερικές λέξεις ήταν παρόμοιες ή η μια ήταν στον ενικό ενώ η άλλη στον πληθυντικό ή υπήρχαν getters και setters.
    πχ \begin{itemize}
        \item \texttt{action} και \texttt{actions}.
        \item \texttt{accounts} και \texttt{getAccounts}.
    \end{itemize}
\end{itemize}

Τελικά, αποφασίσαμε ότι σε πολλές περιπτώσεις είναι πιο σημαντική η ύπαρξη ή όχι μιας λέξης σε μια βιβλιοθήκη παρά τον αριθμό των εμφανίσεών της.
Αυτό συμβαίνει γιατί γενικά το dataset μας είναι σχετικά αραιό.

\section{Η διαδικασία προεπεξεργασίας δεδομένων}
Η επεξεργασία έγινε στο Python
\footnote{Γραμμένο για να δουλεύει με python3 αλλά έγιναν διάφορες αλλαγές ώστε να τρέχει σωστά και σε python2.}
αρχείο \texttt{preprocess.py}.
Το dataset μας αρχικά διαβάζεται από ένα \texttt{.csv} αρχείο και αποθηκεύεται σε ένα αντικείμενο \lstinline!DataFrame! της βιβλιοθήκης \lstinline!pandas!.

Κατά την προεπεξεργασία των δεδομένων ακολουθήθηκε μια δεντρική δομή για την παραγωγή διάφορων τελικών dataset.
Αυτή η δομή προσδιορίζεται από τη συνάρτηση \lstinline!tree_init()!
και μπορεί να γίνει εύκολα η επεξεργασία της ώστε να αλλάξουν τα τελικά αποτελέσματα.

Ρίζα του δέντρου θεωρείται πάντα το αρχικό μας dataset και κάθε μεταβολή του αναπαριστάται σε ένα κόμβο παιδί.
Η κάθε διαδικασία επεξεργασίας αντιστοιχεί σε μία ακμή του δέντρου.
Τα διάφορα dataset που δημιουργούνται αποθηκεύονται σαν αρχεία και πάλι σε δεντρική δομή όπου οι φάκελοι είναι οι ακμές και τα \texttt{.csv} αρχεία οι κόμβοι.
Μια τέτοια δομή φαίνεται παρακάτω:
\begin{Verbatim}[frame=single]
datasets/root
├── join_duplicates
│   ├── frequency_based_selection
│   │   ├── gibberish_detector
│   │   │   ├── bool_it
│   │   │   │   ├── frequency_based_selection2
│   │   │   │   │   └── dataset.csv
│   │   │   │   └── dataset.csv
│   │   │   ├── join_similar
│   │   │   │   ├── drop_fry_words
│   │   │   │   │   ├── frequency_based_selection_df
│   │   │   │   │   │   └── dataset.csv
│   │   │   │   │   └── dataset.csv
│   │   │   │   └── dataset.csv
│   │   │   └── dataset.csv
│   │   └── dataset.csv
│   └── dataset.csv
└── dataset.csv
\end{Verbatim}

Στη συνέχεια περιγράφονται οι διάφορες διαδικασίες που αναπτύχθηκαν.

\subsection{Frequency Based Selection}
Αφαίρεση των λέξεων που εμφανίζονται πολύ συχνά ή πολύ σπάνια.
Η επιλογή βασίζεται μόνο στο αν μια λέξη εμφανίζεται ή όχι σε μια βιβλιοθήκη.
Ο απόλυτος αριθμός εμφανίσεων δεν έχει σημασία.
Έτσι, αν θέσουμε κατώτατο όριο $3$ και μία λέξη εμφανιστεί $500$ φορές σε μια βιβλιοθήκη αλλά πουθενά αλλού, τότε αυτή η λέξη θεωρείται ότι εμφανίζεται πολύ σπάνια και θα κοπεί.

Η υλοποίηση βρίσκεται στη συνάρτηση \lstinline!frequency_based_selection! και ακολουθούμε τα εξής βήματα:
\begin{enumerate}
\item Μετατροπή του dataset σε bool.
\begin{lstlisting}[numbers=none, aboveskip=\smallskipamount, belowskip=\smallskipamount, captionpos=none]
to_bool = dataset.applymap(lambda x: True if x else False)
\end{lstlisting}

\item Άθροισμα κατά γραμμή.
\begin{lstlisting}[numbers=none, aboveskip=\smallskipamount, belowskip=\smallskipamount, captionpos=none]
to_bool_sums = to_bool.sum(axis=0)
\end{lstlisting}

\item Εύρεση και αφαίρεση όσων στηλών είναι εκτός των ορίων από το αρχικό dataset \lstinline!low_bound! και \lstinline!upper_bound!
\begin{lstlisting}[numbers=none, aboveskip=\smallskipamount, belowskip=\smallskipamount, captionpos=none]
to_drop = [
    column
    for column, nonzeros in zip(columns, to_bool_sums)
    if nonzeros < low_bound or nonzeros > upper_bound
]
return dataset.drop(to_drop, axis=1)
\end{lstlisting}
\end{enumerate}

\subsection{Join Duplicates}
Συνάθροιση των λέξεων που είναι διπλές.
Αυτή η συνάρτηση συνήθως έχει νόημα αν μετατρέψουμε το αρχείο σε lower case καθώς το διαβάζουμε.
\begin{lstlisting}[captionpos=none, numbers=none]
def join_duplicates(dataset):
    """Join duplicate words."""
    return dataset.groupby(dataset.columns, axis=1).sum()
\end{lstlisting}

\subsection{Μετατροπή σε δυαδικό}
Όποτε μια λέξη δεν εμφανίζεται η τιμή παραμένει μηδέν.
Αλλιώς γίνεται 1.
\begin{lstlisting}[captionpos=none, numbers=none]
def bool_it(dataset):
    """Convert all int values too boolean."""
    return dataset.applymap(lambda x: 1 if x else 0)
\end{lstlisting}

\subsection{Drop Fry Words}
Αφαίρεση των \href{http://www.k12reader.com/subject/vocabulary/fry-words/}{Fry Words}.
Αποτελεί μια λίστα από τις $1000$ πιο συχνές λέξεις της Αγγλικής γλώσσας.
Καθώς αυτές οι λέξεις είναι πολύ γενικές δεν έχουν κάποιο ιδιαίτερο νόημα σε προγραμματιστικό περιβάλλον και έτσι τις αφαιρούμε.
Οι λέξεις φορτώνονται από ένα αρχείο και συγκρίνονται με τις λέξεις του dataset.
\begin{lstlisting}[captionpos=none, numbers=none, breaklines=true]
def drop_fry_words(dataset, filename='fry-words.txt'):
    """
    Drops columns that have a name that is a Fry word.
    The Fry Sight Word List is made up of the most frequently used words in
    children's books, novels, articles and textbooks.
    """
    with open(filename) as file_object:
        fry_words = []
        for line in file_object:
            fry_words += filter_line(line, delimiter=' ', startpos=0)
    to_drop = [word for word in fry_words if word in dataset.columns]
    return dataset.drop(to_drop, axis=1)
\end{lstlisting}

\subsection{Join Similar}
Συνάθροιση όλων των λέξεων που είναι παρόμοιες σαν strings (με βάση τους χαρακτήρες).
Υλοποιείται στην συνάρτηση \lstinline!join_similar!.

Η δομή της συνάρτησης είναι η εξής:
\begin{itemize}
\item Καθώς αυτού του τύπου σύγκριση μπορεί να δώσει αποτελέσματα που δεν θέλουμε να συναθροίσουμε έπρεπε να ελέγξουμε όλα τα αποτελέσματα
και να εξαιρέσουμε μερικά από αυτά μέσω της \lstinline!list! \lstinline!blacklist!.
\begin{lstlisting}[captionpos=none, numbers=none, breaklines=true]
blacklist = [
    ('adding', 'padding'),
    ...
    ('stats', 'status')
]
\end{lstlisting}

\item Στην κυρίως επανάληψη της συνάρτησης συγκρίνουμε κάθε λέξη με τις υπόλοιπες.
Αυτές που είναι αρκετά όμοιες τις ομαδοποιούμε μαζί.
\begin{lstlisting}[captionpos=none, numbers=none, breaklines=true]
for idx, word in enumerate(dataset.columns[:-1]):
    rest = dataset.columns[idx + 1:]
    ...
    for match in close:
        if (word, match) in blacklist:
            close.remove(match)
    if close:
        to_join.append([word] + close)
        to_drop += close
\end{lstlisting}

\item Για την εύρεση της ομοιότητας χρησιμοποιείται η συνάρτηση \lstinline!get_close_matches()! της βιβλιοθήκης \lstinline!difflib! της python.
\begin{lstlisting}[captionpos=none, numbers=none, breaklines=true]
close = get_close_matches(
    word=word,  # For which word to find similarities.
    possibilities=rest,  # search in the rest of the columns list.
    n=len(rest),  # Don't limit the search for too many results.
    cutoff=similarity_bound)  # At least this score.
\end{lstlisting}

\item Συναθροίζουμε όλες τις ομάδες παρόμοιων string στο πρώτο μέλος της ομάδας και αφαιρούμε τα υπόλοιπα.
Οι συναθροίσεις γίνονται από το τέλος έτσι ώστε αν
η λέξη \texttt{A} μοιάζει με την \texttt{B} και η \texttt{B} με την \texttt{C} στο τελικό αποτέλεσμα θα
συναθροίσουμε τις \texttt{B} και \texttt{C} στην \texttt{A}.
\begin{lstlisting}[captionpos=none, numbers=none, breaklines=true]
for group in to_join[::-1]:
    # sum group to the first member
    dataset[group[0]] = sum(dataset[member] for member in group)
return dataset.drop(to_drop, axis=1)
\end{lstlisting}
\end{itemize}

\subsection{Gibberish Detector}
\sloppy Για την εύρεση λέξεων που είναι "ασυναρτησίες" χρησιμοποιούμε την συνάρτηση
\lstinline!gibberish_detector()!.

Η συνάρτηση χρησιμοποιεί το πρόγραμμα \href{https://github.com/rrenaud/Gibberish-Detector}{Gibberish-Detector}
που βρέθηκε μέσω της ερώτησης \href{http://stackoverflow.com/a/6298193/3430986}{"Is there any way to detect strings like putjbtghguhjjjanika?"}
στο \href{stackoverflow.com}{stackoverflow}.
Το πρόγραμμα αυτό χρησιμοποιεί μια μαρκοβιανή αλυσίδα για να πετύχει τον στόχο του και είναι επίσης γραμμένο σε Python(2).
Για να το χρησιμοποιήσουμε πρέπει να τρέξουμε το \texttt{gib\_detect\_train.py} για την εκπαίδευση
και να μετακινήσουμε το αποτέλεσμα \texttt{gib\_model.pki} στον φάκελο με τα datasets.

Στην συνάρτηση αφαιρούμε κάθε λέξη του dataset που ταξινομείται ως "ασυναρτησία".
\begin{lstlisting}[captionpos=none, numbers=none, breaklines=true]
to_drop = [
    column for column in dataset.columns if is_word_gibberish(column)]
return dataset.drop(to_drop, axis=1)
\end{lstlisting}

\chapter{Ομαδοποίηση}
Πριν προχωρήσουμε στους αλγορίθμους που χρησιμοποιήσαμε είναι χρήσιμο να αναφέρουμε 2 σημαντικά ζητήματα που αφορούν την ομαδοποίηση μας. Το πρώτο αφορά την μετρική που θα χρησιμοποιήσουμε για την αξιολόγηση της ομαδοποίησης και ο δεύτερος τον τρόπο με τον οποίο θα μετρήσουμε την απόσταση. Όσον αφορά την μετρική μπορούμε να αναφέρουμε ότι σαν κύρια μετρική θα χρησιμοποιήσουμε το $Silhouette$ καθώς μέσα σε αυτήν εμπεριέχονται οι μετρικές $Cohesion$ και $Separation$ και μπορεί να θεωρηθεί πιο αντιπροσωπευτική.
\section{Διαχωριστικοί Αλγόριθμοι}
Ο αλγόριθμος που κυρίως αντιπροσωπεύει αυτή την κατηγορία αλγορίθμων είναι ο K-means. Η βασική ιδέα των διαχωριστικών αλγορίθμων είναι η ανάθεση των σημείων μας σε ομάδες προσπαθώντας να ελαχιστοποιήσουμε την απόσταση από ένα σημείο που αντιπροσωπεύει την ομάδα. Έτσι κάθε σημείο μας θεωρούμε ότι ανήκει στην ομάδα αυτή στην οποία η απόσταση από το αντιπροσωπευτικό σημείο είναι ελάχιστη. Εκτός από τον K-means σε αυτήν την κατηγορία ανήκουν διάφορες παραλλαγές του όπως ο bisection K-means, ο fuzzy Κ-means και ο K-medoid. Στην δικιά μας υλοποίηση χρησιμοποιήσαμε τον απλό K-means και τον K-medoid.

\subsection{Εισαγωγή}
Όπως αναφέραμε ο K-means είναι ο πιο συνηθισμένος και απλός διαχωριστικός αλγόριθμος. Η λογική που χρησιμοποιεί είναι η εξής:
\begin{enumerate}
    \item Κάθε ομάδα συνδέεται με ένα κέντρο (centroid) το οποίο είναι το αντιπροσωπευτικό σημείο της ομάδας.
    \item Κάθε σημείο αποδίδεται στην ομάδα με το πιο κοντινό κέντρο ελαχιστοποιώντας την μεταξύ τους απόσταση.
    \item Ο αριθμός των ομάδων Κ πρέπει να έχει καθοριστεί από πριν.
\end{enumerate}

Πιο συγκεκριμένα μπορούμε να αναλύσουμε τα βήματα του αλγορίθμου που ακολουθούμε ως εξής:
\begin{enumerate}
    \item Επιλέγουμε k σημεία ως αρχικά κέντρα.
    \item Δημιουργούμε k ομάδες με τον τρόπο που περιγράφτηκε.
    \item Υπολογίζουμε τα νέα κέντρα των ομάδων μας.
    \item Επαναλαμβάνουμε τα βήματα 2-3 μέχρις ότου δεν μεταβληθούν τα κέντρα.
\end{enumerate}

Παρακάτω παρουσιάζεται ο αλγόριθμος K-means σε ψευδογλώσσα:\\
\noindent\begin{minipage}{0.9\linewidth}
\centering
\begin{algorithm}[H]
    Select K points as the initial centroids.\;
    \Do{The centroids don't change}{
    Form K clusters by assigning all points to the closest centroid.\;
    Recompute the centroid of each cluster.\;
    }
\end{algorithm}
\end{minipage}

Αφού περιγράψαμε τις βασικές ιδέες του αλγορίθμου K-means μπορούμε πλέον να προχωρήσουμε σε ορισμένα σημαντικά θέματα που αφορούν τον αλγόριθμο K-means. Αυτά είναι ο αριθμός των ομάδων, η επιλογή των αρχικών κέντρων και ο τρόπος υπολογισμού της απόστασης.Ο τελικός αριθμός των ομάδων στην περίπτωση μας είναι ίσος με 8. Μπορούμε να δημιουργήσουμε περισσότερες ή λιγότερες ομάδες αρχικά και σταδιακά να φτάσουμε στον τελικό αριθμό. Όσον αφορά το θέμα της επιλογής των αρχικών κέντρων υπάρχουν διάφορες τεχνικές αντιμετώπισης αυτού του προβλήματος. Τα αρχικά κέντρα συνήθως επιλέγονται τυχαία. Αν και συνήθως αυτοί οι αλγόριθμοι συγκλίνουν με τυχαία επιλογή κέντρων υπάρχει πάντα η πιθανότητα να πέσουμε σε τοπικό ελάχιστο της προς ελαχιστοποίηση συνάρτησης. Για αυτόν τον λόγο χρησιμοποιήθηκαν 2 τρόποι αντιμετώπισης του προβλήματος των αρχικών τιμών.
\begin{enumerate}
    \item Τυχαία αρχικοποίηση των centroids αλλά επιλέγοντας να τρέξει πολλές φορές ο K-means. Έτσι ουσιαστικά τρέχουμε πολλές φορές τον αλγόριθμο ομαδοποίησης και επιλέγουμε κάθε φορά αυτόν που μας δίνει το ελάχιστο σφάλμα.

    \item Heuristic Μέθοδος επιλογής centroids.
    Υπάρχουν διάφορες τεχνικές επιλογής αρχικού κέντρου με Heuristic μεθόδους που προκύπτουν από την εμπειρία μας.
    Η τεχνική που χρησιμοποιήσαμε εμείς ακολουθά την παρακάτω λογική και έχει ως σκοπό την επιλογή Κ centroid, όσες και οι ομάδες μας.
    Επιλέγουμε σαν αρχική τιμή centroid το σημείο από τα δεδομένα μας που βρίσκεται πιο κοντά στον μέσο όρο των σημείων μας.
    Έτσι, έχουμε ένα centroid.
    Για το επόμενο, υπολογίζουμε τις αποστάσεις των σημείων μας από το centroid και ορίζουμε αυτό που βρίσκεται πιο μακριά από το centroid.
    Έτσι, έχουμε 2 centroids.
    Όμοια, προχωράμε επιλέγοντας σαν επόμενο centroid αυτό που απέχει περισσότερο από τα ήδη επιλεγμένα centroid.
    Συνεχίζουμε έτσι μέχρις ότου επιλέξουμε K centroids.
    Αναλυτικά τα βήματα του αλγορίθμου:\\
    \noindent\begin{minipage}{0.9\linewidth}
    \centering
    \begin{algorithm}[H]
        From n objects calculate a point whose attribute values are average of n-objects attribute values so first initial centroid is average of n-objects.\;
        Select next initial centroids from n-objects in such a way that the Euclidean distance of that object is maximum from other selected initial centroids.\;
        Repeat step2 until we get k initial centroids.
        From these steps we will get initial centroids and with these initial centroids perform K-means algorithm.\;
    \end{algorithm}
    \end{minipage}
\end{enumerate}

Ο παραπάνω αλγόριθμος υλοποιήθηκε στο Matlab μέσω της συνάρτησης
\lstinline[language=MATLAB]!initial_centroid=centroid_heuristic(X, number_of_features, total_centroid_counter)!

Όπου οι είσοδοι μας είναι:
\begin{enumerate}
    \item \lstinline[language=MATLAB]!X!: Το dataset μας σε μορφή πίνακα.
    \item \lstinline[language=MATLAB]!number_of_features!: Ο αριθμός των γνωρισμάτων μας που στην περίπτωση μας είναι ο αριθμός των βιβλιοθικών μας , δηλαδή 80.
    \item \lstinline[language=MATLAB]!total_centroid_counter!: Ο αριθμός των ομάδων που θέλουμε να δημιουργήσουμε.
\end{enumerate}

Έξοδος της συνάρτησης μας είναι τα Κ αρχικά κέντρα.


Ακόμα, για να αποφύγουμε το πρόβλημα των αρχικών centroids πολλές φορές χρησμιποιείται ο bisecting K-means καθώς εξαρτάται λιγότερο από την αρχική επιλογή των κεντρών.

Η πολυπλοκότητα του αλγορίθμου είναι $ O(n*K*I*d)$ όπου $n$ είναι ο αριθμός σημείων, $Κ$ είναι ο αριθμός ομάδων, $l$ είναι αριθμός επαναλήψεων και $d$ είναι αριθμός μεταβλητών. Πρόκειται για έναν αρκετά γρήγορο αλγόριθμο.

Τέλος η τελευταία παράμετρος που επιλέγεται στον αλγόριθμο K-means είναι η απόσταση. Οι μετρικές που χρησιμοποιούνται σαν απόσταση είναι:
\begin{enumerate}
    \item \textbf{Τετραγωνική Ευκλείδια (sqeuclidean)}: Τετραγωνική ευκλείδια απόσταση.
    \item \textbf{Cityblock}: To άθροισμα της απόλυτης διαφοράς γνωστή και ως $L1$ απόσταση.
    \item \textbf{Cosine}: Απόσταση που εμπεριέχει το συνημίτονο της γωνίας των σημείων.
    \item \textbf{Correlation}: Απόσταση που εμπεριέχει την συσχέτιση των σημείων.
    \item \textbf{Hamming}: Απόσταση που χρησιμοποιείται για δυαδικά δεδομένα. Είναι το ποσοστό των bit που διαφέρουν.
\end{enumerate}

Ένα ακόμα πρόβλημα του K-means είναι ότι όταν οι ομάδες μας είναι ανισομεγεθής, έχουν διαφορετική πυκνότητα και μη σφαιρικά σχήματα. Τα παραπάνω προβλήματα επιλύονται με την δημιουργία πολλών μικρών ομάδων και την σύνθεση τους σε επίπεδο μετ-επεξεργασίας, θέμα το οποίο θα αναλυθεί παρακάτω εκτενέστερα στα ανοιχτά θέματα.
Παρακάτω φαίνεται μια ομαδοποίηση που πραγματοποιήθηκε με K-means. Είναι ευδιάκριτα τόσο τα 3 clusters που δημιουργήθηκαν όσο και τα κέντρα τους. Κάθε σημείο του κάθε cluster απέχει την ελάχιστη απόσταση από το κέντρο του cluster στο οποιό ανήκει.

\noindent\begin{minipage}{\linewidth}
    \centering
    \captionsetup{type={figure}}
    \includegraphics[width=1.0\linewidth]{images/kmeans}
    \captionof{figure}{TODO}
    \label{fig:kmeans}
\end{minipage}

Τέλος βλέπουμε ένα παράδειγμα μιας ομαδοποίησης ενός dataset μέσω του αλγορίθμου K-means ανάλογα με το βήμα στο οποίο βρίσκεται. Βλέπουμε πως μεταβάλλονται τα κέντρα με το πέρασμα των επαναλήψεων και έτσι και τα σημεία που ανήκουν σε κάθε cluster. Στο τελευταίο βήμα παρατηρούμε πάλι ότι το σημείο κάθε cluster απέχει ελάχιστη απόσταση από το κέντρο του cluster στο οποίο βρίσκεται και συνεπώς δεν χρειάζεται να γίνει άλλη επανάληψη και ο αλγόριθμος έχει τερματιστεί.

\noindent\begin{minipage}{\linewidth}
    \centering
    \captionsetup{type={figure}}
    \includegraphics[width=1.0\linewidth]{images/kmeans_change_centroids}
    \captionof{figure}{TODO}
    \label{fig:kmeans_change_centroids}
\end{minipage}

\subsubsection{Πειραματικά Αποτελέσματα}

Στην συνέχεια προχωρήσαμε στην υλοποίηση του αλγορίθμου K-means για τα διαφορετικά datasets που φτιάξαμε και περιγράφτηκαν παραπάνω. Τρέξαμε όλα τα datasets μας για τον αλγόριθμο K-means στο Matlab για διαφορετικά σετ παραμέτρων του αλγορίθμου μας. Τα διαφορετικά σετ παραμέτρων που χρησιμοποιήσαμε όπως αναφέραμε και παραπάνω αφορούν τόσο τον τρόπο υπολογισμού της απόστασης όσο και τον τρόπο επιλογής των αρχικών κεντρών. Όυσιαστικά υλοποιήσαμε 9 σετ παραμέτρων τα οποία παρουσιάζονται παρακάτω:

\begin{enumerate}
    \item \textbf{1ο Σετ Παραμέτρων}:
    \begin{itemize}
        \item \textbf{Τρόπος Υπολογισμού Απόστασης} : Cosine
        \item \textbf{Τρόπος Επιλογής Αρχικού Κέντρου} : Τυχαίος
    \end{itemize}
    \item  \textbf{2ο Σετ Παραμέτρων}:
    \begin{itemize}
        \item \textbf{Τρόπος Υπολογισμού Απόστασης} : Cosine
        \item \textbf{Τρόπος Επιλογής Αρχικού Κέντρου} : Τυχαίος με επανάληψη αλγορίθμου 10 φορές
    \end{itemize}
    \item  \textbf{3ο Σετ Παραμέτρων}:
        \begin{itemize}
            \item \textbf{Τρόπος Υπολογισμού Απόστασης} : Cosine
            \item \textbf{Τρόπος Επιλογής Αρχικού Κέντρου} : Επιλογή μέσω του ευρυστικού κανόνα.
        \end{itemize}
    \item  \textbf{4ο Σετ Παραμέτρων}:
        \begin{itemize}
            \item \textbf{Τρόπος Υπολογισμού Απόστασης} : Correlation
            \item \textbf{Τρόπος Επιλογής Αρχικού Κέντρου} : Τυχαίος
        \end{itemize}
    \item  \textbf{5ο Σετ Παραμέτρων}:
        \begin{itemize}
            \item \textbf{Τρόπος Υπολογισμού Απόστασης} : Correlation
            \item \textbf{Τρόπος Επιλογής Αρχικού Κέντρου} : Τυχαίος με επανάληψη αλγορίθμου 10 φορές
        \end{itemize}
    \item  \textbf{6ο Σετ Παραμέτρων}:
        \begin{itemize}
            \item \textbf{Τρόπος Υπολογισμού Απόστασης} : Correlation
            \item \textbf{Τρόπος Επιλογής Αρχικού Κέντρου} :  Επιλογή μέσω του ευρυστικού κανόνα.
        \end{itemize}
\end{enumerate}

Τα παραπάνω σετ παραμέτρων υλοποιούνται στο Matlab μέσω των εντολών:
\begin{enumerate}
\item \textbf{1ο Σετ Παραμέτρων}:
\lstinline[language=MATLAB, breaklines=true]!kmeans(X, clnumber, 'Distance', cosine);!
\\Όπου \lstinline[language=MATLAB]!X! είναι το σύνολο δεδομένων εισόδου,
\lstinline[language=MATLAB]!clnumber! είναι ο αριθμός των ομάδων που δημιουργούνται και η παράμετρος
\lstinline[language=MATLAB]!Distance! που δηλώνει τον τρόπο υπολογισμού της απόστασης ορίζεται να ισούται με \lstinline[language=MATLAB]!cosine!. Δεν δηλώνεται τίποτα για τις αρχικές τιμές των κέντρων τον ομάδων.

\item \textbf{2ο Σετ Παραμέτρων}:
\lstinline[language=MATLAB, breaklines=true]!kmeans(X, clnumber, 'Distance', cosine'Replicates', number_of_iretation);!
\\Όπου
\lstinline[language=MATLAB]!X! είναι το σύνολο δεδομένων εισόδου,
\lstinline[language=MATLAB]!clnumber! είναι ο αριθμός των ομάδων που δημιουργούνται, η παράμετρος
\lstinline[language=MATLAB]!Distance! δηλώνει τον τρόπο υπολογισμού της απόστασης ορίζεται να ισούται με
\lstinline[language=MATLAB]!cosine! και η παράμετρος
\lstinline[language=MATLAB]!Replicates! δηλώνει την επανάληψη του K-means
\lstinline[language=MATLAB]!number_of_iretaion! φορές που στην περίπτωση μας επιλέχτηκε 10.

\item \textbf{3ο Σετ Παραμέτρων}:
\lstinline[language=MATLAB, breaklines=true]!kmeans(X, clnumber, 'Distance', cosine, 'Start', heuristic_centroid);!
\\Όπου
\lstinline[language=MATLAB]!X! είναι το σύνολο δεδομένων εισόδου,
\lstinline[language=MATLAB]!clnumber! είναι ο αριθμός των ομάδων που δημιουργούνται και η παράμετρος
\lstinline[language=MATLAB]!Distance! που δηλώνει τον τρόπο υπολογισμού της απόστασης ορίζεται να ισούται με
\lstinline[language=MATLAB]!cosine!.
Η παράμετρος \lstinline[language=MATLAB]!Start! δηλώνει την αρχική επιλογή των κέντρων και ισούται με \lstinline[language=MATLAB]!heuristic_centroid! που προκύπτουν από την κλήση της συνάρτησης \lstinline[language=MATLAB]!centroid_heuristic! που περιγράφτηκε παραπάνω.

\item \textbf{4ο Σετ Παραμέτρων}:
\lstinline[language=MATLAB, breaklines=true]!kmeans(X, clnumber, 'Distance', correlation);!
\\Όπου
\lstinline[language=MATLAB]!X! είναι το σύνολο δεδομένων εισόδου,
\lstinline[language=MATLAB]!clnumber! είναι ο αριθμός των ομάδων που δημιουργούνται και η παράμετρος
\lstinline[language=MATLAB]!Distance! που δηλώνει τον τρόπο υπολογισμού της απόστασης ορίζεται να ισούται με
\lstinline[language=MATLAB]!correlation!.
Δεν δηλώνεται τίποτα για τις αρχικές τιμές των κέντρων τον ομάδων.

\item \textbf{5ο Σετ Παραμέτρων}:
\lstinline[language=MATLAB, breaklines=true]!kmeans(X, clnumber, 'Distance', correlation'Replicates', number_of_iretation);!
\\Όπου
\lstinline[language=MATLAB]!X! είναι το σύνολο δεδομένων εισόδου,
\lstinline[language=MATLAB]!clnumber! είναι ο αριθμός των ομάδων που δημιουργούνται, η παράμετρος
\lstinline[language=MATLAB]!Distance! δηλώνει τον τρόπο υπολογισμού της απόστασης ορίζεται να ισούται με
\lstinline[language=MATLAB]!correlation! και η παράμετρος \lstinline[language=MATLAB]!Replicates! δηλώνει την επανάληψη του K-means
\lstinline[language=MATLAB]!number_of_iretaion! φορές που στην περίπτωση μας επιλέχτηκε 10.

\item \textbf{6ο Σετ Παραμέτρων}:
\lstinline[language=MATLAB, breaklines=true]!kmeans(X, clnumber, 'Distance', correlation, 'Start', heuristic_centroid);!
\\Όπου
\lstinline[language=MATLAB]!X! είναι το σύνολο δεδομένων εισόδου,
\lstinline[language=MATLAB]!clnumber! είναι ο αριθμός των ομάδων που δημιουργούνται και η παράμετρος
\lstinline[language=MATLAB]!Distance! που δηλώνει τον τρόπο υπολογισμού της απόστασης ορίζεται να ισούται με
\lstinline[language=MATLAB]!correlation!.
Η παράμετρος \lstinline[language=MATLAB]!Start! δηλώνει την αρχική επιλογή των κέντρων και ισούται με
\lstinline[language=MATLAB]!heuristic_centroid! που προκύπτουν από την κλήση της συνάρτησης
\end{enumerate}

Αφού ορίσαμε τα σετ παραμέτρων μας, τρέξαμε στο Matlab το script με όνομα \lstinline[language=MATLAB]!scriptk!.
Tο παραπάνω script καλεί την συνάρτηση \lstinline[language=MATLAB]!optimizer_kmeans!, η οποία διαβάζει τα dataset μας, υλοποιεί τον αλγόριθμο K-means και μας βγάζει τα διαγράμματα στα οποία έχουμε τις μετρικές μας.
Η φόρτωση των dataset μας γίνεται με την συνάρτηση \lstinline[language=MATLAB]!file_paths! και τα διαγράμματα με την συνάρτηση \lstinline[language=MATLAB]!plot_bars!.

Το πρώτο διάγραμμα που πήραμε περιέχει σε ένα κοινό διάγραμμα το 1o, 2ο και 3ο σετ δεδομένων κάνοντας την ομαδοποίηση για 8 clusters.
\noindent\begin{minipage}{\linewidth}
    \centering
    \captionsetup{type={figure}}
    \includegraphics[width=1.0\linewidth]{images/kmeansCosBar8}
    \captionof{figure}{TODO}
    \label{fig:kmeansCosBar8}
\end{minipage}

Οι 2 μετρικές στις οποίες δίνουμε μεγαλύτερη βαρύτητα είναι το $Silhouette$ και το $Success Rate$. Το $Silhouette$ είναι μια μετρική που περιγράψαμε παραπάνω. Το $Success Rate$ είναι η μετρική που μας δίνει τι ποσοστό από τις υπάρχουσες βιβλιοθήκες βρήκαμε σωστό. Όπως είναι λογικό μεγάλο $Silhouette$ αντιστοιχεί συνήθως σε μεγάλο $Success Rate$ χωρίς αυτό να σημαίνει ότι το μέγιστο $Silhouette$ αντιστοιχεί στο μέγιστο $Success Rate$. Σαν βασική μας μετρική θα θεωρήσουμε το $Silhouette$ καθώς θεωρούμε ότι είναι πιο σωστό αφού σε κάθε πρόβλημα ομαδοποίησης δεν έχουμε τα τελικά αποτελέσματα για να μπορούμε να κάνουμε την σύγκριση. Το $Success Rate$ δηλαδή δεν μπορεί να θεωρηθεί αντιπροσωπευτική μετρική αλλά μπορεί να χρησιμοποιηθεί για επαλήθευση.
Στο παραπάνω διάγραμμα, είπαμε ότι στον κατακόρυφο άξονα έχουμε τα διάφορα dataset μας.Οι 3 πρώτες ομάδες τιμών αντιστοιχούν στο ίδιο dataset υλοποιημένο με διαφορετικό σετ παραμέτρων. Έτσι η πρώτη ομάδα τιμών (από πάνω προς τα κάτω) μας δίνει το πρώτο dataset με επιλογή κέντρου με $heuristic$ τρόπο, η δεύτερη ομάδα τιμών το πρώτο dataset με τυχαία επιλογή κέντρου με πολλαπλές επαναλήψεις και η τρίτη ομάδα τιμών το πρώτο dataset με τυχαία επιλογή κέντρου. Η τέταρτη ομάδα τιμών μας δίνει το δεύτερο dataset με επιλογή κέντρου με $heuristic$ τρόπο, η πέμπτη ομάδα το δεύτερο dataset  με τυχαία επιλογή κέντρου με πολλαπλές επαναλήψεις και η έκτη ομάδα τιμών το δεύτερο dataset με τυχαία επιλογή κέντρου. Με παρόμοιο τρόπο συνεχίζουμε στις επόμενες ομάδες τιμών. Παρατηρούμε δηλαδή ότι ένα dataset μας κατέχει 3 συνεχόμενες θέσεις στον κατακόρυφο άξονα.
Κάθε ομάδα τιμών περιέχει τις 4 μετρικές που περιγράψαμε με διαφορετικό χρώμα.

Παρατηρούμε ότι από το διάγραμμα έχουμε το μέγιστο $Silhouette$ για την 5η εγγραφή στον κατακόρυφο άξονα. Αυτή η εγγραφή αφορά το dataset με όνομα \url{join_duplicates/freq_8_70/gibberish_detector/join_similar/drop_fry_words/dataset.csv} με επιλογή κέντρου τυχαία με πολλαπλές επαναλήψεις. Ωστόσο τρέξαμε στο Matlab μια σύγκριση με το επόμενο διάγραμμα και παρατηρήσαμε ότι το επόμενο διάγραμμα δίνει υψηλότερο μέγιστο $Silhouette$. Αυτό που είναι χρήσιμο από αυτό το διάγραμμα είναι να πάρουμε το βέλτιστο $Success Rate$ το οποίο βρίσκεται στο dataset \url{join_duplicates/freq_8_70/dataset.csv} με τυχαία επιλογή κέντρου με πολλαπλές επαναλήψεις. Η τιμή του $Success Rate$ σε αυτήν την περίπτωση είναι ίση με 92.5\% το οποίο μεταφράζεται ότι πετύχαμε σωστή ομαδοποίηση για 74 βιβλιοθήκες από το σύνολο των 80. Αυτό είναι το καλύτερο ποσοστό σωστών αποτελεσμάτων που επιτύχαμε. Γενικά τα βέλτιστα αποτελέσματα στον K-means επιτυγχάνονται με τυχαία επιλογή κέντρου με πολλαπλές επαναλήψεις και για αυτόν τον λόγο δεν θα αναφέρεται.

Στην συνέχεια πήραμε σε ένα κοινό διάγραμμα το 4o, 5ο και 6ο σετ δεδομένων κάνοντας την ομαδοποίηση για 8 clusters.

\noindent\begin{minipage}{\linewidth}
    \centering
    \captionsetup{type={figure}}
    \includegraphics[width=1.0\linewidth]{images/kmeansCorBar8}
    \captionof{figure}{TODO}
    \label{fig:kmeansCorBar8}
\end{minipage}

Από το παραπάνω διάγραμμα προκύπτει ότι το dataset με το μέγιστο $Silhouette$ είναι το \url{join_duplicates/freq_3_70/gibberish_detector/join_similar/dataset.csv}. Η τιμή του $Silhouette$ είναι 0.259 και το αντίστοιχο $SuccessRate$ είναι 88.7\% δηλαδή 70 βιβλιοθήκες ομαδοποιήθηκαν σωστά. Αυτή η τιμή του $Silhouette$ ξεπερνάει και όλες τις προηγούμενες από το διάγραμμα όπου η απόσταση μετρήθηκε ως $Cosine$. Παρατηρούμε λοιπόν ότι ενώ μεγάλες τιμές του συντελεστή $Silhouette$ αντιστοιχούν σε μεγάλες τιμές $SuccessRate$ και το αντίστροφο, οι μέγιστες τιμές της μιας μετρικής δεν σημαίνουν και μέγιστες τιμές της άλλης μετρικής.

Η μετρική $Success Rate$ στις 2 παραπάνω ομαδοποιήσεις μετρήθηκε με τον 1ο τρόπο.
%ΤΟ DO HYPERLINK TO DESCRIBE OF eval_clust(type1)
Αφού υλοποιήσαμε την παραπάνω ομαδοποίηση περάσαμε τα αποτελέσματα που πήραμε στο Weka για οπτικοποίηση των καλύτερων αποτελεσμάτων μας.
\noindent\begin{minipage}{\linewidth}
    \centering
    \captionsetup{type={figure}}
    \includegraphics[width=\linewidth]{images/kmeans8_result1.eps}
    \includegraphics[width=\linewidth]{images/kmeans8_result2.eps}
    \captionof{figure}{TODO}
    \label{fig:kmeans8_result}
\end{minipage}

Παρατηρούμε ότι στην πρώτη περίπτωση όπου είναι η ομαδοποίηση με βέλτιστο $Success Rate$ έχούμε πετύχει όλες τις βιβλιοθήκες \url{http-clients} (γκρι), τις \url{swing-libraries} λαχανί και τις \url{xml-processing}. Από την βιβλιοθήκη \url{android} (μπλε) πετύχαμε 22 βιβλιοθήκες και από τις υπόλοιπες ομαδοποιήσαμε επιτυχώς το μεγαλύτερο ποσοστό. Διαπυστώνουμε ότι όντως λάθος ομαδοποίηση υπέστησαν 6 βιβλιοθήκες. Άρα ορθώς το $Success Rate$ είναι 92.5\%.

Όμοια στο δεύτερο διάγραμμα παρατηρούμε ότι επιτύχαμε το μεγαλύτερο ποσοστό \url{android} (μπλε) βιβλιοθηκών. Παρατηρούμε ότι 7 βιβλιοθήκες ομαδοποιήθηκαν λάθος. 
Έπειτα υλοποιήσαμε μια μέτρηση με το 1ο, 2ο και 3ο σετ παραμέτρων και υλοποιήσαμε 10 ομάδες. Πήραμε το παρακάτω διάγραμμα:

\noindent\begin{minipage}{\linewidth}
    \centering
    \captionsetup{type={figure}}
    \includegraphics[width=1.0\linewidth]{images/kmeansCosBar10}
    \captionof{figure}{TODO}
    \label{fig:kmeansCosBar10}
\end{minipage}

Από το παραπάνω διάγραμμα προκύπτει ότι το dataset \url{join_duplicates/freq_3_70/gibberish_detector/join_similar/dataset.csv} έχει το βέλτιστο $Silhouette$ και τιμή ίση με 0.263. Το αντίστοιχο ποσοστό $SuccessRate$ είναι 87.5\% το οποίο σημαίνει ότι αν συνενώσουμε τις ομάδες μας και δημιουργήσουμε 8 clusters, οι 70 βιβλιοθήκες μας θα έχουν ομαδοποιηθεί σωστά.

Το αντίστοιχο διάγραμμα για το 4ο, 5ο και 6ο σετ παραμέτρων με 10 κλάσεις είναι το παρακάτω:\\
\noindent\begin{minipage}{\linewidth}
    \centering
    \captionsetup{type={figure}}
    \includegraphics[width=1.0\linewidth]{images/kmeansCorBar10}
    \captionof{figure}{TODO}
    \label{fig:kmeansCorBar10}
\end{minipage}

Από αυτό το διάγραμμα παίρνουμε το καλύτερο $SuccessRate$ που ισούται με 93.8\% στο dataset \url{join_duplicates/freq_8_50/gibberish_detector/join_similar/drop_fry_words/dataset.csv}.

Όμοια με πριν χρησιμοποιήσαμε το Weka για οπτικοποίηση των αποτελεσμάτων μας. Στην πρώτη περίπτωση έχουμε το μεγαλύτερο $Success Rate$ και στην δεύτερη το μεγαλύτερο $Silhouete$.

\noindent\begin{minipage}{\linewidth}
    \centering
    \captionsetup{type={figure}}
    \includegraphics[width=\linewidth]{images/kmeans10_result1.eps}
    \includegraphics[width=\linewidth]{images/kmeans10_result2.eps}
    \captionof{figure}{TODO}
    \label{fig:kmeans10_result}
\end{minipage}

Παρατηρούμε ότι ο αριθμός των κλάσεων είναι 10. Αυτό που αξίζει να παρατηρήσουμε είναι ότι αν προχωρούσαμε στο επόμενο στάδιο της μετ-επεξεργασίας, στην πρώτη περίπτωση, η 1η κλάση και η 8η κλάση θα ενώνονταν καθώς είναι και οι 2 καθαρά android βιβλιοθήκες. Όμοια θα μπορούσαμε να κάνουμε και άλλες συνενώσεις.

Τέλος υλοποιήσαμε ομαδοποίηση για 20 κλάσεις. Αρχικά πήραμε το 1ο, 2ο και 3ο σετ παραμέτρων:\\
\noindent\begin{minipage}{\linewidth}
    \centering
    \captionsetup{type={figure}}
    \includegraphics[width=1.0\linewidth]{images/kmeansCosBar20}
    \captionof{figure}{TODO}
    \label{fig:kmeansCosBar20}
\end{minipage}

Από το παραπάνω διάγραμμα πήραμε το καλύτερο $Success Rate$ να ισούται με 93.8\% στο dataset \url{join_duplicates/freq_8_50/gibberish_detector/join_similar/drop_fry_words/bool_it/dataset.csv}.

Τα αντίστοιχα αποτελέσματα στο 4ο, 5ο και 6ο dataset για 20 clusters:\\
\noindent\begin{minipage}{\linewidth}
    \captionsetup{type={figure}}
    \centering
    \includegraphics[width=1.0\linewidth]{images/kmeansCorBar20}
    \captionof{figure}{TODO}
    \label{fig:kmeansCorBar20}
\end{minipage}

Το παραπάνω διάγραμμα μας δίνει καλύτερη μετρική $Silhouette$ ίση με 0.278.

Η μετρική $Success Rate$ στις 4 παραπάνω ομαδοποιήσεις μετρήθηκε με τον 2ο τρόπο.
%ΤΟ DO HYPERLINK TO DESCRIBE OF eval_clust(type2)
Οι 4 τελευταίες ομαδοποιήσεις, δηλαδή ο διαχώρισμος των ομάδων σε 10 και 20 αντίστοιχα έγιναν για να αντιμετωπίσουν το πρόβλημα που έχει ο k-means σε ανισομεγεθής ομάδες όπως αναφέρθηκε και στην παράγραφο της εισαγωγής για τον K-means. Ουσιαστικά δεν έγινε έπειτα η μείωση των ομάδων σε 8 όπως θέλουμε να είναι ο τελικός αριθμός. Περισσότερες πληροφορίες σχετικά με το θέμα της μετ-επεξεργασίας δίνονται στο κεφάλαιο με τα Ανοιχτά Θέματα.
%TO DO HYPERLINK TO OPENTHEMES
Παρακάτω βλέπουμε τα αποτελέσματα της βέλτιστης ομαδοποίησης ως προς $Success Rate$ kai $Silhouette$. Παρατηρούμε ότι πρόκειται όντως για 20 ομάδες. Πολλές ομάδες θα μπορούσαν να συνενωθούν σε εννιαίες όπως οι \url{android} (ομάδες 2,4,5,10,11), οι \url{http-clients} (ομάδες 17,18) και oi \url{command-line-parsers} (ομάδες 9,15). Οι υπόλοιπες συνενώσεις θα γίνουν για να διατηρηθεί ελάχιστο το $SSE$.
\\\noindent\begin{minipage}{\linewidth}
    \centering
    \captionsetup{type={figure}}
    \includegraphics[width=\linewidth]{images/kmeans20_result1.eps}
    \includegraphics[width=\linewidth]{images/kmeans20_result2.eps}
    \captionof{figure}{TODO}
    \label{fig:kmeans20_result}
\end{minipage}
% Please add the following required packages to your document preamble:
% \usepackage{graphicx}
Στο παρακάτω πινακάκι παρουσιάζονται μαζεμένες όλες οι βέλτιστες λύσεις για το $Silhouette$ και για το $Success Rate$ για τις ομαδοποιήσεις που έγιναν.

\begin{minipage}{\linewidth}
    \centering
    \resizebox{\textwidth}{!}{%
        \begin{tabular}{lllllllll}
            \cline{7-7}
            Dataset & Center Type & Distance Type & Number of Clusters & Cohesion & \multicolumn{1}{l|}{Separation} & \multicolumn{1}{l|}{Silhouette} & Success Rate & Max Type \\ \cline{7-7}
            Α & random multiple iterations & Cosine & 8 & 592 & 5080 & 0.231 & 0.925 & Best Success Rate \\
            B & random multiple iterations & Correlation & 8 & 660 & 5090 & 0.259 & 0.887 & Best Silhouette \\
            C & random multiple iterations & Correlation & 10 & 462 & 5300 & 0.222 & 0.938 & Best Success Rate \\
            B & random multiple iterations & Cosine & 10 & 544 & 2630 & 0.263 & 0.875 & Best Silhouette \\
            D & random multiple iterations & Cosine & 20 & 168 & 5500 & 0.213 & 0.938 & Best Success Rate \\
            B & random multiple iterations & Correlation & 20 & 228 & 5520 & 0.278 & 0.9 & Best Silhouette
        \end{tabular}
    }
    \captionof{table}{TODO}
    \label{K-means Best Results}
\end{minipage}

Στο παραπάνω πινακάκι χρησιμοποιήσαμε τους κωδικούς των dataset τα οποία ορίσαμε ως εξής:
\begin{enumerate}
    \item A=\url{join_duplicates/freq_8_70/dataset.csv}
    \item B=\url{join_duplicates/freq_3_70/gibberish_detector/join_similar/dataset.csv}
    \item C=\url{join_duplicates/freq_8_50/gibberish_detector/join_similar/drop_fry_words/dataset.csv}
    \item D=\url{join_duplicates/freq_8_50/gibberish_detector/join_similar/drop_fry_words/bool_it/dataset.csv}
\end{enumerate}

\subsection{Κ-medoids}

\subsubsection{Εισαγωγή}
Ο K-medoids είναι μία παραλλαγή του k-means που χρησιμοποιείται σε κατηγορικά η διακριτά δεδομένα.
Η βασική του διαφορά είναι ότι χρησιμοποιεί σαν κέντρο του cluster ένα σημείο από αυτό το επονομαζόμενο medoid.

Γενικά θεωρείται πιο ανθεκτικός στο θόρυβο επειδή προσπαθεί να ελαχιστοποιήσει την ανομοιότητα ανάμεσα σε στοιχεία παρά το τετραγωνικό άθροισμα των αποστάσεων όπως κάνει συχνά ο k-means.
Σαν medoid επιλέγεται συνήθως το σημείο το οποίο διαφέρει λιγότερα από όλα τα σημεία του cluster.

Ένας από του πιο συχνούς αλγορίθμους του k-medoids είναι o \textbf{Partitioning Around Medoids (PAM)}.
Ο τρόπος με τον οποίο δουλεύει
περιγράφεται με τον ακόλουθο ψευδοκώδικα:\\
\begin{algorithm}[H]
    Αρχικοποίηση$\:$διάλεξε στην τύχη $\kappa$ σημεία ως medoids\;
    Συσχέτισε το κάθε σημείο με το κοντινότερο medoid\;
    \While{το συνολικό κόστος μειώνεται(συνολικό κόστος των cluster)}{
        \For{κάθε σημείο medoid $m$, για κάθε μη-medoid σημείο $o$}{
            Άλλαξε(swap) το $m$ με το $o$ \;
            Ξαναυπολόγισε το κόστος του cluster(άθροισμα αποστάσεων από το medoid) \;
            \If{το συνολικό κόστος αυξηθεί}{
                Αναίρεσε την αλλαγή(swap)
            }
        }
    }
\end{algorithm}

Ένα παράδειγμα του παραπάνω αλγορίθμου φάινεται στο παρακάτω σχήμα:\\
\noindent\begin{minipage}{\linewidth}
    \centering
    \captionsetup{type={figure}}
	\makebox[\linewidth]{
	\includegraphics{../../../../../Dropbox/protypa-figs/pictures-2/kmedoid}}
	\captionof{figure}{Παράδειγμα k-medoid}\label{fig:pam}
\end{minipage}

\subsubsection{Πειραματικά Αποτελέσματα}

Καθώς ο K-medoids αποτελεί μια παραλλαγή του K-means πιο ανθεκτική στον θόρυβο δεν περιμένουμε να έχουμε καλύτερα αποτελέσματα με την εφαρμογή του καθώς δεν έχουμε ανεπιθύμητο θόρυβο στα δεδομένα μας. Τρέξαμε όλα τα dataset μας για τον αλγόριρμο K-medoids στο Matlab χρησιμοποιώνας την εντολή \lstinline[language=MATLAB]!kmedoids(X,clnumber,'Distance',correlation);!
Η μόνη παράμετρος που αλλάξαμε ήταν η απόσταση και χρησιμοποιήσαμε τον τύπο correlation. Τρέξαμε για είδος απόστασης cosine και τα αποτελέσματα ήταν ίδια. Παρατηρήσαμε ότι οι άλλες παράμετροι της εντολής \lstinline[language=MATLAB]!kmedoids! δεν δίνουν διαφορετικά αποτελέσματα για αυτό χρησιμοποιήθηκαν οι default τιμές τους. Για να πάρουμε τα διαγράμματα με τις μετρικές τρέξαμε το script  \lstinline[language=MATLAB]!scriptm!. Αυτό το script καλεί την συνάρτηση \lstinline[language=MATLAB]!optimizer_kmedoids! η οποία διαβάζει τα dataset μας, υλοποιεί τον αλγόριθμο K-medoids και μας βγάζει τα διαγράμματα στα οποία έχουμε τις μετρικές μας. Η φόρτωση των dataset μας γίνεται με την συνάρτηση \lstinline[language=MATLAB]!file_paths! και τα διαγράμματα με την συνάρτηση \lstinline[language=MATLAB]!plot_bars!.

Το διάγραμμα που παίρνουμε είναι το εξής:
\includegraphics{../../../../../Dropbox/protypa-figs/pictures-2/MedCorBar}

Παρατηρούμε ότι τα αποτελέσματα μας είναι αρκετά ικανοποιητικά, δεν φτάνουν όμως τα επίπεδα των αποτελεσμάτων του K-means.

Παρακάτω βλέπουμε τα αποτελέσματα από τις βέλτιστες ομαδοποιήσεις ως προς τις μετρικές $SuccessRate$ και $Silhouette$ αντίστοιχα.
\noindent\begin{minipage}{\linewidth}
	\centering
	\captionsetup{type={figure}}
\includegraphics{../../../../../Dropbox/protypa-figs/pictures-2/kmedoid_result1}
\includegraphics{../../../../../Dropbox/protypa-figs/pictures-2/kmedoid_2}
	\captionof{figure}{TODO}
	\label{fig:kmeans8_result}
\end{minipage}

Παρατηρούμε ότι και στις 2 ομαδοποιήσεις ενώ πετυχαίνουμε υψηλά ποσοστά στην ομαδοποίηση των μεγάλων cluster χάνουμε πιο πολλές βιβλιοθήκες από πριν.

Τέλος στο παρακάτω πινακάκι παρουσιάζονται μαζεμένες οι βέλτιστες λύσεις για το $Silhouette$ και για το $Success Rate$ για την ομαδοποιήση που έγινε.
% Please add the following required packages to your document preamble:
% \usepackage{graphicx}
\begin{table}[]
	\centering
	\resizebox{\textwidth}{!}{%
		\begin{tabular}{llllll}
			\cline{4-4}
			Dataset & Cohesion & \multicolumn{1}{l|}{Separation} & \multicolumn{1}{l|}{Silhouette} & Success Rate & Max Type \\ \cline{4-4}
			&  &  &  &  & Best Success Rate \\
			&  &  &  &  & Best Silhouette
		\end{tabular}
	}
	\caption{K-medoids Best Results}
	\label{my-label}
\end{table}
\chapter{Ιεραρχικοί αλγόριθμοι}
Οι ιεραρχικοί αλγόριθμοι (hieratchical cluster analysis or HCA) είναι μία μεγάλη κατηγορία αλγορίθμων που χρησιμοποιούνται για να κάνουμε ομαδοποίηση δεδομένων .Ονομάζονται έτσι επειδή προσπαθούν να δημιουργήγουν μία ιεραρχία από συστάδες (clusters)
Οο στρατηγικές που χρησιμοπούνται είναι γενικά 2
\begin{itemize}
\item{η \textbf{Συνάνθροισης (Aglomerative)} Είναι μία μέθοδος 'bottom-up' δηλαδή : στην αρχή κάθε cluster αποτελείται από μία παρατήρηση και μετά ζευγάρια από cluster 'συναθροίζονται' και γίνονται ένα cluster όσο άνεβάινουμε την ιεραρχία  }
\item{}οι\textbf{Διαχωρισμού (Divisive)} Είναι ουσιαστικά το ανάποδο ('top down) όπου αρχικά όλες οι παρατηρήσεις είναι ένα cluster και έπειτα διαχωρίζοντε όσο κατεβαίνουμε την ιεραρχία.(Τέτοιος αλγόριθμος είναι το Ελαφρύτατου Συνδετικού Δεντρο (Minimum Spanning Tree)) 
end{itemize}

Εφαρμόστηκαν αλγόριθμοι με την μέθοδο της συνάθροισης για αυτό και θα αναλύσουμε κυρίως   
\section{Περιγραφή εξερεύνησης των ιεραρχικών μοντέλων}

Τα μοντέλα μας δημιουργήθηκαν με την χρήση του Matlab και η διαδικασία που περιγράφτηκε στην εισαγωγή πραγματοποιείται στο αρχείο optimizer\_hier.m
για τις διάφορες παραμέτρους. Οι αποστάσεις ανάμεσα στα σημεία ή υπολογίζονται για τους ιεραρχικούς με την χρήση της συνάρτησης pdist , η οποία μπορεί να υπολογίσει τις εξής αποστάσεις:
\begin{itemize}
	\item euclidean
	\item seuclidean
	\item Mankowski
	\item chebychev
	\item mahalanobis
	\item cosine
	\item correlation
	\item spearman
	\item jaccard 
\end{itemize}


Περισσότερες πληροφορίες για την pdist \href{http://www.mathworks.com/help/stats/pdist.html}{εδώ}
Σαν αποστάσεις ικανοποιητικά αποτελέσματα έδιναν μόνο το correlation και το 
cosine  μία πιθανή εξήγηση βρίσκεται \textbf{(Εδώ τα reference})

Έπειτα χρησιμοποιήθηκε η συνάτηση linkage οποία δημιουργεί ουσιαστικά το δέντρο ιεραρχίας .Οι μετρική που χρησιθμοποιεί για να συνδέσει τα κοντινοερα cluster μπορεί να είναι μία από της εξής:
\begin{itemize}
  	\item single
  	\item complete
  	\item average
  	\item weighted
  	\item centroid
  	\item median
  	\item ward 
  \end{itemize}

Από άποψη χρόνου και αποτελεσμάτων ύστερα από δοκιμές για τα τελικά πειράματα επιλέχτηκαν οι weighted ,ward, complete,average.

Έπειτα με την χρήση της συνάρτησης cluster δημιουργήσαμε τα τελικά cluster που θέλαμε τα παραπάνω βήματα υλοποιούνται από τις παρακάτω γραμμές κώδικα 


\begin{lstlisting}[language=Matlab]
%simple example of hierarchical clustering
Y = pdist (X,distance); %X is an array containig the data
YY = squareform(Y); %convert Y in a square form
Z =linkage(YY,); %find  hierarchical cluster tree,
CDX = cluster(Z,'maxclust',8);
\end{lstlisting}


επίσης υπάρχει η συνάρτηση cophonet(Z,YY) η οποία είναι μια μετρική που υπολογίζει την συσχέτιση ανάμεσα στις συνδέσεις που έχει δημιουργήσει το Z και τις αντίστοιχες αποστάσεις στο YY .Όσο πιο μεγάλη είναι η συσχέτιση τόσο πιο καλά έχει αποτυπωθεί η διαφορετικότητα των αρχικών σημείων, σαν συνδέσεις μεταξύ clusters , περισσότερα
\href{https://en.wikipedia.org/wiki/Cophenetic\_correlation{εδώ}}

Στη συνέχεια παρουσιάζονται και σχολιάζονται τα πειραματικά αποτελέσματα.

\section{Πειραματικά αποτελέσματα}

Χρησιμοποιούνται οι γνωστές μετρικές silhouette ,cohesion,separation
αλλά και η success1 η οποία είναι μία από τις 2 μετρικές που  υπολογίζει το πόσα πετύχαμε από τα πραγματικά δεδομένα.Περισσότερα για το πως υπολογίστηκαν και τη αντιπροσωπεύει η success1 στο \textbf{TELOS} του κεφαλαίου.   	

Έγινε χρήση των προαναφερθένων  αλγορίθμων απόστασης και εφαρμόστηκαν στον καθένα 4 διαφορετικοί τρόποι σύνδεσης για όλα τα dataset


\begin{figure}
	\includegraphics{images/hierCosBar.pdf}
	\caption{Μετρικές για τον ιεραρχικό αλγόριθμο με την χρήση του Cosine}
	\label{fig:CosineHier}
\end{figure}



\begin{figure}
	\includegraphics{images/hierCorBar.pdf}
	\caption{Μετρικές για τον ιεραρχικό αλγόριθμο με την χρήση του Cosine}
	\label{fig:CorrelationHier}
\end{figure}





\include{openissues}
\end{document}
