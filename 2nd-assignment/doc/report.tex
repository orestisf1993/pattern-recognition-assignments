%&preamble
\usepackage[linesnumbered, boxruled]{algorithm2e}
\SetKwRepeat{Do}{do}{while}

% automatic hyphenation for 2 languages
% http://www.mechpedia.gr/wiki/Hyphenation_-_%CE%A5%CF%86%CE%B5%CE%BD%CF%8E%CF%83%CE%B5%CE%B9%CF%82#.CE.91.CF.85.CF.84.CF.8C.CE.BC.CE.B1.CF.84.CE.B5.CF.82_.CF.85.CF.86.CE.B5.CE.BD.CF.8E.CF.83.CE.B5.CE.B9.CF.82_.CF.83.CE.B5_.CE.B4.CE.AF.CE.B3.CE.BB.CF.89.CF.83.CF.83.CE.B1_.CE.BA.CE.B5.CE.AF.CE.BC.CE.B5.CE.BD.CE.B1
% very slow, enable only at final pdf.
%TODO: \usepackage[Greek,Latin]{ucharclasses}
%TODO: \setTransitionsForGreek{\selectlanguage{greek}}{\selectlanguage{english}}
% Save static part as preamble.tex and use command:
% xelatex -ini -shell-escape -job-name="preamble" "&xelatex preamble.tex\dump"
% to produce preamble.fmt

% polyglossia
\usepackage{polyglossia}
\setmainlanguage{greek}
\setotherlanguages{english}

% Fonts
% fonts can't go in the .fmt file
\usepackage{fontspec}
\setmainfont[Mapping=tex-text]{DejaVu Sans}
\newfontfamily\greekfont[Script=Greek]{DejaVu Sans}
\newfontfamily\greekfontsf[Script=Greek]{DejaVu Sans}
\setmonofont[Scale=1.0]{Source Code Pro Medium}		
\newfontfamily\greekfonttt[Scale=1.0]{Source Code Pro Medium}
\usepackage{microtype} % microtype is font-dependant

\title{Αναγνώριση Προτύπων\\
Δεύτερη Εργασία Μαθήματος – Clustering}
\author{
  Παυλίδης Αντώνης, 7758 \href{mailto:ant\_pavlidis@yahoo.gr}{ant\_pavlidis@yahoo.gr}\\
  Φλώρος-Μαλιβίτσης Ορέστης, 7796 \href{mailto:orestisf@ece.auth.gr}{orestisf@ece.auth.gr}\\
  Χαμζάς Κωνσταντίνος, 7798 \href{mailto:chamzask@ece.auth.gr}{chamzask@ece.auth.gr}\\
  \\\\ \textbf{Τομέας Ηλεκτρονικής}\\
  \textbf{Τμήμα Ηλ. Μηχανικών / Μηχανικών ΗΥ}\\
  \textbf{Αριστοτέλειο Πανεπιστήμιο Θεσσαλονίκης}}
\titlepic{\includegraphics[width=0.40\textwidth]{images/university}}
\renewcommand{\labelenumii}{\arabic{enumii}.}
\renewcommand{\labelenumiii}{\arabic{enumiii}.}
\begin{document}
\maketitle
\tableofcontents
\listoffigures
\listoftables

\chapter{Εισαγωγή}
\section{Περιγραφή του προβλήματος}

Το πρόβλημα που καλούμαστε να λύσουμε είναι η πρόβλεψη σφαλμάτων (bugs) σε έργα λογισμικού.
Ένας τρόπος αντιμετώπισης είναι η χρήση διάφορων μερτικών πάνω στα τμήματα που έχουν και που δεν έχουν σφάλμα. Στο συγκεκριμένο project μας δίνεται ένα σύνολο από κλάσεις με υπολογισμένες μετρικές και γνώση αν περιέχουν σφάλμα ή όχι (training set). Με την χρήση αυτών των δεδομένων θα δημιουργήσουμε διάφορα μοντέλα ταξινόμησης που θα ταξινομούν τις κλάσεις σε εσφαλμένες και μη.

Εδώ πρέπει να ορίσουμε με ποια κριτήρια θα κρίνουμε αν ένα μοντέλο είναι καλό ή όχι. Για αυτό θα λάβουμε υπόψιν 2 παράγοντες.
\begin{enumerate}
\item Θεωρούμε σημαντικό να βρεθούν όσο το δυνατόν περισσότερο bugs από αυτά που υπάρχουν (recall) καθώς η διόρθωση σε αρχικά στάδια ενός έργου λογισμικού είναι πολύ πιο φθηνή.
\item Θα διατεθούν πόροι για κάθε κλάση που θα θεωρήσει το μοντέλο ταξινόμησης μας εσφαλμένη ώστε είτε να επιδιορθωθεί το bug ή να εξακριβωθεί ότι δεν περιέχει κάποιο bug. Συνεπώς, είναι σημαντικό το ποσοστό των  σωστά προβλεμένων εσφαλμένων κλάσεων προς των συνολικών προβλεμένων εσφαλμέων κλάσεων (precision) να είναι υψηλό.
\end{enumerate}

Συνολικά λοιπόν θα δώσουμε βάση στην μετρική f-measure που συνδυάζει και το recall και το precision. Ακόμη, θα δώσουμε περισσότερη έμφαση στο recall σε σχέση με το precision γιατί θεωρούμε ότι είναι πιο οικονομικό να διορθωθούν πολλά bugs στην αρχή παρά στο λίγα στο τέλος. 

\section{Αλγόριθμοι που χρησιμοποιήθηκαν}
Οι αλγόριθμοι ταξινόμησης που χρησιμοποιήσαμε ήταν :
\begin{enumerate}
  \item \textbf{Bayes:}
  \begin{itemize}
     \item NaiveBayes
     \item BayesNet
  \end{itemize}  
  \item \textbf{Trees:}
  \begin{itemize}
     \item J48
     \item RandomForest
  \end{itemize}
  \item \textbf{SVM:}  
  \begin{itemize}
     \item C-SVC (libsvm)
     \item SMO
  \end{itemize}
  \item \textbf{NNs:}
  \begin{itemize}
     \item KNN
     \item NN
  \end{itemize}
\end{enumerate}

\section{Εργαλεία που χρησιμοποιήθηκαν}

\begin{enumerate}
  \item \textbf{Weka}
  \begin{itemize}
     \item TODO
  \end{itemize}  
  \item \textbf{Python:}
\begin{itemize}
     \item TODO
\end{itemize}
\end{enumerate}
\chapter{Προεπεξεργασία Δεδομένων}
\section{Μορφή των δεδομένων}
Τα δεδομένα μας προέρχονται από τους όρους-λέξεις 80 βιβλιοθηκών της γλώσσας προγραμματισμού Java που ανήκουν σε 8 κατηγορίες:
\begin{enumerate}\label{itemize:categories}
    \item android
    \item command-line-parsers
    \item csv-libraries
    \item http-clients
    \item json-libraries
    \item swing-libraries
    \item testing-frameworks
    \item xml-processing
\end{enumerate}

Το αρχικό μας dataset περιλαμβάνει $109706$ λέξεις οι οποίες παρουσιάζονται σαν στήλες σε ένα αρχείο \texttt{.csv}.
Κάτω από την καθεμία αναγράφονται οι απόλυτες συχνότητες εμφάνισης του κάθε όρου.
Επίσης, δίνονται οι στήλες \texttt{project}, κάτω από την οποία είναι τα ονόματα της κάθε βιβλιοθήκης, και \texttt{category}, κάτω από την οποία είναι η
\hyperref[itemize:categories]{κατηγορία} που ανήκει η κάθε βιβλιοθήκη.

\begin{center}
\captionsetup{type={table}}
\resizebox{\textwidth}{!}{%
\begin{tabular}{|c|c|c|c|c|c|c|c|c|c|}
\hline
project & category & ACONST & CompoundButtonCheckedChangeOnSubscribe & StatisticsComponent & getProjection & hashedSignature & propertyDescs & testGetBound & testReadPaths \\ \hline
com.appnexus.opensdk/appnexus-sdk & android & 0 & 0 & 0 & 0 & 0 & 0 & 0 & 0 \\
com.auth0.android/lock & android & 0 & 0 & 0 & 0 & 0 & 0 & 0 & 0 \\
com.bingzer.android.ads/adrunner & android & 0 & 0 & 0 & 0 & 0 & 0 & 0 & 0 \\
com.daimajia.easing/library & android & 0 & 0 & 0 & 0 & 0 & 0 & 0 & 0 \\
com.facebook.android/facebook-android-sdk & android & 0 & 0 & 0 & 0 & 2 & 0 & 0 & 0 \\
com.facebook.fresco/fbcore & android & 0 & 0 & 0 & 0 & 0 & 0 & 0 & 0 \\
com.facebook.fresco/fresco & android & 0 & 0 & 0 & 0 & 0 & 0 & 0 & 0 \\
com.github.asne/asne-core & android & 0 & 0 & 0 & 0 & 0 & 0 & 0 & 0 \\
com.github.castorflex.smoothprogressbar/library & android & 0 & 0 & 0 & 0 & 0 & 0 & 0 & 0 \\
com.hannesdorfmann.mosby/mvp & android & 0 & 0 & 1 & 0 & 0 & 0 & 0 & 0 \\
com.jakewharton.rxbinding/rxbinding & android & 0 & 3 & 0 & 0 & 0 & 0 & 0 & 0 \\
com.jakewharton.timber/timber & android & 0 & 0 & 0 & 0 & 0 & 0 & 0 & 0 \\
com.jdroidframework/jdroid-android & android & 0 & 0 & 0 & 3 & 0 & 0 & 0 & 0 \\
com.joanzapata.iconify/android-iconify & android & 0 & 0 & 0 & 0 & 0 & 0 & 0 & 0 \\
com.mixpanel.android/mixpanel-android & android & 0 & 0 & 0 & 0 & 0 & 3 & 0 & 1 \\
com.rengwuxian.materialedittext/library & android & 0 & 0 & 0 & 0 & 0 & 0 & 0 & 0 \\
com.shamanland/xdroid-core & android & 0 & 0 & 0 & 0 & 0 & 0 & 0 & 0 \\
com.stanfy.enroscar/enroscar-beans & android & 0 & 0 & 0 & 0 & 0 & 0 & 0 & 0 \\
com.stanfy.enroscar/enroscar-sdk-dep & android & 0 & 0 & 0 & 0 & 0 & 0 & 0 & 0 \\
com.stanfy.enroscar/enroscar-shared & android & 0 & 0 & 0 & 0 & 0 & 0 & 0 & 0 \\
de.quist.apps.maps/android-maps-abstraction & android & 0 & 0 & 0 & 6 & 0 & 0 & 0 & 0 \\
eu.inmite.android.lib/android-styled-dialogs & android & 0 & 0 & 0 & 0 & 0 & 0 & 0 & 0 \\
io.palaima.debugdrawer/debugdrawer & android & 0 & 0 & 0 & 0 & 0 & 0 & 0 & 0 \\
io.reactivex/rxandroid & android & 0 & 0 & 0 & 0 & 0 & 0 & 0 & 0 \\
args4j/args4j & command-line-parsers & 0 & 0 & 0 & 0 & 0 & 0 & 0 & 0 \\
com.beust/jcommander & command-line-parsers & 0 & 0 & 0 & 0 & 0 & 0 & 0 & 0 \\
com.github.spullara.cli-parser/cli-parser & command-line-parsers & 0 & 0 & 0 & 0 & 0 & 0 & 0 & 0 \\
com.lexicalscope.jewelcli/jewelcli & command-line-parsers & 0 & 0 & 0 & 0 & 0 & 0 & 0 & 0 \\
commons-cli/commons-cli & command-line-parsers & 0 & 0 & 0 & 0 & 0 & 0 & 0 & 0 \\
gnu.getopt/java-getopt & command-line-parsers & 0 & 0 & 0 & 0 & 0 & 0 & 0 & 0 \\
io.airlift/airline & command-line-parsers & 0 & 0 & 0 & 0 & 0 & 0 & 0 & 0 \\
net.sf.jopt-simple/jopt-simple & command-line-parsers & 0 & 0 & 0 & 0 & 0 & 0 & 0 & 0 \\
net.sourceforge.argparse4j/argparse4j & command-line-parsers & 0 & 0 & 0 & 0 & 0 & 0 & 0 & 0 \\
org.cyclopsgroup/jcli & command-line-parsers & 0 & 0 & 0 & 0 & 0 & 0 & 0 & 0 \\
org.realityforge.getopt4j/getopt4j & command-line-parsers & 0 & 0 & 0 & 0 & 0 & 0 & 0 & 0 \\
au.com.bytecode/opencsv & csv-libraries & 0 & 0 & 0 & 0 & 0 & 0 & 0 & 0 \\
com.fasterxml.jackson.dataformat/jackson-dataformat-csv & csv-libraries & 0 & 0 & 0 & 0 & 0 & 0 & 0 & 0 \\
com.opencsv/opencsv & csv-libraries & 0 & 0 & 0 & 0 & 0 & 0 & 0 & 0 \\
net.sf.flatpack/flatpack & csv-libraries & 0 & 0 & 0 & 0 & 0 & 0 & 0 & 0 \\
net.sf.supercsv/super-csv-dozer & csv-libraries & 0 & 0 & 0 & 0 & 0 & 0 & 0 & 0 \\
net.sf.supercsv/super-csv & csv-libraries & 0 & 0 & 0 & 0 & 0 & 0 & 0 & 0 \\
org.apache.commons/commons-csv & csv-libraries & 0 & 0 & 0 & 0 & 0 & 0 & 0 & 0 \\
org.beanio/beanio & csv-libraries & 0 & 0 & 0 & 0 & 0 & 0 & 0 & 0 \\
org.jdbi/jdbi & csv-libraries & 0 & 0 & 0 & 0 & 0 & 0 & 0 & 0 \\
org.onebusaway/onebusaway-csv-entities & csv-libraries & 0 & 0 & 0 & 0 & 0 & 0 & 0 & 0 \\
com.github.kevinsawicki/http-request & http-clients & 0 & 0 & 0 & 0 & 0 & 0 & 0 & 0 \\
com.google.http-client/google-http-client & http-clients & 0 & 0 & 0 & 0 & 0 & 0 & 1 & 0 \\
com.jcabi/jcabi-http & http-clients & 0 & 0 & 0 & 0 & 0 & 0 & 0 & 0 \\
com.m3/curly & http-clients & 0 & 0 & 0 & 0 & 0 & 0 & 0 & 0 \\
com.mashape.unirest/unirest-java & http-clients & 0 & 0 & 0 & 0 & 0 & 0 & 0 & 0 \\
com.metamx/http-client & http-clients & 0 & 0 & 0 & 0 & 0 & 0 & 0 & 0 \\
com.ning/async-http-client & http-clients & 0 & 0 & 0 & 0 & 0 & 0 & 0 & 0 \\
com.squareup.okhttp/okhttp & http-clients & 0 & 0 & 0 & 0 & 0 & 0 & 0 & 0 \\
org.apache.httpcomponents/httpasyncclient & http-clients & 0 & 0 & 0 & 0 & 0 & 0 & 0 & 0 \\
org.apache.httpcomponents/httpclient & http-clients & 0 & 0 & 0 & 0 & 0 & 0 & 0 & 0 \\
com.alibaba/fastjson & json-libraries & 5 & 0 & 0 & 0 & 0 & 0 & 0 & 0 \\
com.eclipsesource.minimal-json/minimal-json & json-libraries & 0 & 0 & 0 & 0 & 0 & 0 & 0 & 0 \\
com.fasterxml.jackson.core/jackson-core & json-libraries & 0 & 0 & 0 & 0 & 0 & 0 & 0 & 0 \\
com.google.code.gson/gson & json-libraries & 0 & 0 & 0 & 0 & 0 & 0 & 0 & 0 \\
com.googlecode.json-simple/json-simple & json-libraries & 0 & 0 & 0 & 0 & 0 & 0 & 0 & 0 \\
com.jayway.jsonpath/json-path & json-libraries & 0 & 0 & 0 & 0 & 0 & 0 & 0 & 0 \\
net.minidev/json-smart & json-libraries & 4 & 0 & 0 & 0 & 0 & 0 & 0 & 0 \\
org.codehaus.jettison/jettison & json-libraries & 0 & 0 & 0 & 0 & 0 & 0 & 0 & 0 \\
org.json/json & json-libraries & 0 & 0 & 0 & 0 & 0 & 0 & 0 & 0 \\
com.github.rickyclarkson/swingflow & swing-libraries & 0 & 0 & 0 & 0 & 0 & 0 & 0 & 0 \\
net.sf.cssbox/swingbox & swing-libraries & 0 & 0 & 0 & 0 & 0 & 0 & 0 & 0 \\
org.buildsomethingawesome.lib/awesome-java-swing & swing-libraries & 0 & 0 & 0 & 0 & 0 & 0 & 0 & 0 \\
org.fuin/utils4swing & swing-libraries & 0 & 0 & 0 & 0 & 0 & 0 & 0 & 0 \\
org.scijava/swing-checkbox-tree & swing-libraries & 0 & 0 & 0 & 0 & 0 & 0 & 0 & 0 \\
org.softsmithy.lib/softsmithy-lib-swing & swing-libraries & 0 & 0 & 0 & 0 & 0 & 0 & 0 & 0 \\
org.swinglabs/swingx & swing-libraries & 0 & 0 & 0 & 0 & 0 & 0 & 0 & 0 \\
com.novocode/junit-interface & testing-frameworks & 0 & 0 & 0 & 0 & 0 & 0 & 0 & 0 \\
junit/junit & testing-frameworks & 0 & 0 & 0 & 0 & 0 & 0 & 0 & 0 \\
org.testng/testng & testing-frameworks & 0 & 0 & 0 & 0 & 0 & 0 & 0 & 0 \\
xmlunit/xmlunit & testing-frameworks & 0 & 0 & 0 & 0 & 0 & 0 & 0 & 0 \\
com.fasterxml/aalto-xml & xml-processing & 0 & 0 & 0 & 0 & 0 & 0 & 0 & 0 \\
dom4j/dom4j & xml-processing & 0 & 0 & 0 & 0 & 0 & 0 & 0 & 0 \\
jdom/jdom & xml-processing & 0 & 0 & 0 & 0 & 0 & 0 & 0 & 0 \\
net.sf.kxml/kxml2 & xml-processing & 0 & 0 & 0 & 0 & 0 & 0 & 0 & 0 \\
xstream/xstream & xml-processing & 0 & 0 & 0 & 0 & 0 & 0 & 0 & 0 \\ \hline
\end{tabular}
}
\captionof{table}{Παράδειγμα μερικών στηλών για όλες τις γραμμές του αρχικού dataset}
\label{table:init}
\end{center}

\section{Εξερεύνηση Δεδομένων}
Πριν την διαδικασία της προεπεξεργασίας των δεδομένων, πραγματοποιούμε μία αρχική εξερεύνηση τους για την κατανόηση του προβλήματος ομαδοποίησης που καλούμαστε να αντιμετωπίσουμε.

Το αρχικό σύνολο δεδομένων, που έχουμε στη διάθεσή μας, παρατηρούμε ότι περιλαμβάνει έναν πολύ μεγάλο αριθμό λέξεων που αποτελούν ουσιαστικά τα attributes για τους αλγορίθμους μας.
Για αυτό το λόγο, στόχος μας ήταν η δραστική μείωση του αριθμού αυτών των λέξεων ενώ αποφασίσαμε ότι δεν ήταν σημαντικό να αφαιρέσουμε κάποια βιβλιοθήκη (γραμμή) καθώς ο αριθμός τους ήταν ήδη περιορισμένος.

Επίσης, παρατηρήσαμε ότι υπήρχαν διάφορα προβλήματα πάνω στην ποιότητα των λέξεων του dataset:
\begin{itemize}
    \item Πολλές λέξεις δεν είχαν κάποιο νόημα πχ
    % manual hyphenation because it didn't work with anything else.
    \texttt{ABCDE\-FGHIJKLMNO\-PQRSTU\-VWXYZ\-abc\-d\-e\-f\-ghij\-klmnopqrstuvwxyz}
    που λογικά χρησιμοποιείται ως ένα string που περιέχει όλους τους χαρακτήρες της Αγγλικής γλώσσας.

    \item Πολλές λέξεις υπήρχαν σε διάφορα σημεία αλλά λόγω της διάκρισης κεφαλαίων και πεζών θεωρούνταν ξεχωριστές λέξεις.
    πχ \texttt{ACCESS}, \texttt{access} και \texttt{Access}.

    \item Μερικές λέξεις ήταν παρόμοιες ή η μια ήταν στον ενικό ενώ η άλλη στον πληθυντικό ή υπήρχαν getters και setters.
    πχ \begin{itemize}
        \item \texttt{action} και \texttt{actions}.
        \item \texttt{accounts} και \texttt{getAccounts}.
    \end{itemize}
\end{itemize}

Τελικά, αποφασίσαμε ότι σε πολλές περιπτώσεις είναι πιο σημαντική η ύπαρξη ή όχι μιας λέξης σε μια βιβλιοθήκη παρά τον αριθμό των εμφανίσεών της.
Αυτό συμβαίνει γιατί γενικά το dataset μας είναι σχετικά αραιό.

\section{Η διαδικασία προεπεξεργασίας δεδομένων}
Η επεξεργασία έγινε στο Python
\footnote{Γραμμένο για να δουλεύει με python3 αλλά έγιναν διάφορες αλλαγές ώστε να τρέχει σωστά και σε python2.}
αρχείο \texttt{preprocess.py}.
Το dataset μας αρχικά διαβάζεται από ένα \texttt{.csv} αρχείο και αποθηκεύεται σε ένα αντικείμενο \lstinline!DataFrame! της βιβλιοθήκης \lstinline!pandas!.

Κατά την προεπεξεργασία των δεδομένων ακολουθήθηκε μια δεντρική δομή για την παραγωγή διάφορων τελικών dataset.
Αυτή η δομή προσδιορίζεται από τη συνάρτηση \lstinline!tree_init()!
και μπορεί να γίνει εύκολα η επεξεργασία της ώστε να αλλάξουν τα τελικά αποτελέσματα.

Ρίζα του δέντρου θεωρείται πάντα το αρχικό μας dataset και κάθε μεταβολή του αναπαριστάται σε ένα κόμβο παιδί.
Η κάθε διαδικασία επεξεργασίας αντιστοιχεί σε μία ακμή του δέντρου.
Τα διάφορα dataset που δημιουργούνται αποθηκεύονται σαν αρχεία και πάλι σε δεντρική δομή όπου οι φάκελοι είναι οι ακμές και τα \texttt{.csv} αρχεία οι κόμβοι.
Μια τέτοια δομή φαίνεται παρακάτω:
\begin{Verbatim}[frame=single]
datasets/root
├── join_duplicates
│   ├── frequency_based_selection
│   │   ├── gibberish_detector
│   │   │   ├── bool_it
│   │   │   │   ├── frequency_based_selection2
│   │   │   │   │   └── dataset.csv
│   │   │   │   └── dataset.csv
│   │   │   ├── join_similar
│   │   │   │   ├── drop_fry_words
│   │   │   │   │   ├── frequency_based_selection_df
│   │   │   │   │   │   └── dataset.csv
│   │   │   │   │   └── dataset.csv
│   │   │   │   └── dataset.csv
│   │   │   └── dataset.csv
│   │   └── dataset.csv
│   └── dataset.csv
└── dataset.csv
\end{Verbatim}

Στη συνέχεια περιγράφονται οι διάφορες διαδικασίες που αναπτύχθηκαν.

\subsection{Frequency Based Selection}
Αφαίρεση των λέξεων που εμφανίζονται πολύ συχνά ή πολύ σπάνια.
Η επιλογή βασίζεται μόνο στο αν μια λέξη εμφανίζεται ή όχι σε μια βιβλιοθήκη.
Ο απόλυτος αριθμός εμφανίσεων δεν έχει σημασία.
Έτσι, αν θέσουμε κατώτατο όριο $3$ και μία λέξη εμφανιστεί $500$ φορές σε μια βιβλιοθήκη αλλά πουθενά αλλού, τότε αυτή η λέξη θεωρείται ότι εμφανίζεται πολύ σπάνια και θα κοπεί.

Η υλοποίηση βρίσκεται στη συνάρτηση \lstinline!frequency_based_selection! και ακολουθούμε τα εξής βήματα:
\begin{enumerate}
\item Μετατροπή του dataset σε bool.
\begin{lstlisting}[numbers=none, aboveskip=\smallskipamount, belowskip=\smallskipamount, captionpos=none]
to_bool = dataset.applymap(lambda x: True if x else False)
\end{lstlisting}

\item Άθροισμα κατά γραμμή.
\begin{lstlisting}[numbers=none, aboveskip=\smallskipamount, belowskip=\smallskipamount, captionpos=none]
to_bool_sums = to_bool.sum(axis=0)
\end{lstlisting}

\item Εύρεση και αφαίρεση όσων στηλών είναι εκτός των ορίων από το αρχικό dataset \lstinline!low_bound! και \lstinline!upper_bound!
\begin{lstlisting}[numbers=none, aboveskip=\smallskipamount, belowskip=\smallskipamount, captionpos=none]
to_drop = [
    column
    for column, nonzeros in zip(columns, to_bool_sums)
    if nonzeros < low_bound or nonzeros > upper_bound
]
return dataset.drop(to_drop, axis=1)
\end{lstlisting}
\end{enumerate}
\chapter{Ομαδοποίηση}
\section{Διαχωριστικοί Αλγόριθμοι}

Ο αλγόριθμος που κυρίως αντιπροσωπεύει αυτή την κατηγορία αλγορίθμων είναι ο K-means. Η βασική ιδέα των διαχωριστικών αλγορίθμων είναι η ανάθεση των σημείων μας σε ομάδες προσπαθώντας να ελαχιστοποιήσουμε την απόσταση από ένα σημείο που αντιπροσωπεύει την ομάδα. Έτσι κάθε σημείο μας θεωρούμε ότι ανήκει στην ομάδα αυτή στην οποία η απόσταση από το αντιπροσωπευτικό σημείο είναι ελάχιστη. Εκτός από τον K-means σε αυτήν την κατηγορία ανήκουν διάφορες παραλλαγές του όπως ο bisection K-means ,ο fuzzy Κ-means και ο K-medoid. Στην δικιά μας υλοποίηση χρησιμοποιήσαμε τον απλό K-means και τον K-medoid.

\subsection{K-means}

Όπως αναφέραμε ο K-means είναι ο πιο συνηθισμένος και απλός διαχωριστικός αλγόριθμος. Η λογική που χρησιμοποιεί είναι η εξής:
\begin{enumerate}
	\item Κάθε ομάδα συνδέεται με ένα κέντρο (centroid) το οποίο είναι το αντιπροσωπευτικό σημείο της ομάδας.
	\item Κάθε σημείο αποδίδεται στην ομάδα με το πιο κοντινό κέντρο ελαχιστοποιώντας την μεταξύ τους απόσταση. 
	\item Ο αριθμός των ομάδων Κ πρέπει να έχει καθοριστεί από πριν.
\end{enumerate}

Πιο συγκεκριμένα μπορούμε να αναλύσουμε τα βήματα του αλγορίθμου που ακολουθούμε ως εξής:
\begin{enumerate}
	\item Επιλέγουμε k σημεία ως αρχικά κέντρα.
	\item Δημιουργούμε k ομάδες με τον τρόπο που περιγράφτηκε.
	\item Υπολογίζουμε τα νέα κέντρα των ομάδων μας.
	\item Επαναλαμβάνουμε τα βήματα 2-3 μέχρις ώτου δεν μεταβληθούν τα κέντρα.
\end{enumerate}

Παρακάτω παρουσιάζεται ο αλγόριθμος K-means σε ψευδογλώσσα:

\begin{algorithmic}[H]
	\State Select K points as the initial centroids.
	\Do
	\State Form K clusters by assigning all points to the closest centroid.
	\State Recompute the centroid of each cluster.
	\doWhile{The centroids dont change} % <--- use \doWhile for the "while" at the end
\end{algorithmic}

Αφού περιγράψαμε τις βασικές ιδέες του αλγορίθμου K-means μπορούμε πλέον να προχωρήσουμε σε ορισμένα σημαντικά θέματα που αφορούν τον αλγόριθμο K-means. Αυτά είναι ο αριθμός των ομάδων, η επιλογή των αρχικών κέντρων και ο τρόπος υπολογισμού της απόστασης.Ο τελικός αριθμός των ομάδων στην περίπτωση μας είναι ίσος με 8. Μπορούμε να δημιουργήσουμε περισσότερες ή λιγότερες ομάδες αρχικά και σταδιακά να φτάσουμε στον τελικό αριθμό. Όσον αφορά το θέμα της επιλογής των αρχικών κέντρων υπάρχουν διάφορες τεχνικές αντιμετώπισης αυτού του προβλήματος. Τα αρχικά κέντρα συνήθως επιλέγονται τυχαία. Αν και συνήθως αυτοί οι αλγόριθμοι συγκλίνουν με τυχαία επιλογή κέντρων υπάρχει πάντα η πιθανότητα να πέσουμε σε τοπικό ελάχιστο της προς ελαχιστοποίηση συνάρτησης. Για αυτόν τον λόγο χρησιμοποιήθηκαν 2 τρόποι αντιμετώπισης του προβλήματος των αρχικών τιμών. 
\begin{enumerate}
	\item Τυχαία αρχικοποίηση των centroids αλλά επιλέγοντας να τρέξει πολλές φορές ο K-means. Έτσι ουσιαστικά τρέχουμε πολλές φορές τον αλγόριθμο ομαδοποίησης και επιλέγουμε κάθε φορά αυτόν που μας δίνει το ελάχιστο σφάλμα.
	\item Heuristic Μέθοδος επιλογής centroids. Υπάρχουν διάφορες τεχνικές επιλογής αρχικού κέντρου με Heuristic μεθόδους που προκύπτουν από την εμπειρία μας. Η τεχνική που χρησιμοποιήσαμε εμείς ακολουθά την παρακάτω λογική και έχει ως σκοπό την επιλογή Κ centroid, όσες και οι ομάδες μας. Επιλέγουμε σαν αρχική τιμή centroid το σημείο από τα δεδομένα μας που βρίσκεται πιο κοντά στον μέσο όρο των σημείων μας. Έτσι έχουμε ένα centroid. Για το επόμενο υπολογίζουμε τις αποστάσεις των σημείων μας από το centroid και ορίζουμε αυτό που βρίσκεται πιο μακριά από το centroid. Έτσι έχουμε 2 centroids. Όμοια προχωράμε επιλέγοντας σαν επόμενο centroid αυτό που απέχει περισσότερο από τα ήδη επιλεγμένα centroid. Συνεχίζουμε έτσι μέχρις ώτου επιλέξουμε K centroids. Αναλυτικά τα βήματα του αλγορίθμου:
	
	\begin{algorithmic}[H]
		\State Step1: From n objects calculate a point whose attribute values are average of n-objects atrribute values so first initial centroid is average of n-objects.
		\State
	    \State Step2: Select next initial centroids from n-objects in such a way that the Euclidean distance of that object is maximum from other selected initial centroids.
	    \State
	    \State Step3: Repeat step2 until we get k initial centroids.
	    \State
	    \State From these steps we will get initial centroids and with these initial centroids perform K-means algorithm.
	\end{algorithmic}
\end{enumerate}

Ακόμα για να αποφύγουμε το πρόβλημα των αρχικών centroids πολλές φορές χρησμιποιείται ο bisecting K-means καθώς εξαρτάται λιγότερο από την αρχική επιλογή των κεντρών.

Η πολυπλοκότητα του αλγορίθμου είναι Ο(n*K*I*d) όπου n=αριθμός σημείων, Κ=αριθμός ομάδων, l=αριθμός επαναλήψεων , d=αριθμός μεταβλητών. Πρόκειται για έναν αρκετά γρήγορο αλγόριθμο.

Τέλος η τελευταία παράμετρος που επιλέγεται στον αλγόριθμο K-means είναι η απόσταση. Οι μετρικές που χρησιμοποιούνται σαν απόσταση είναι:
\begin{enumerate}
	\item \textbf{Ευκλείδια(euclidean) }: H κλασική ευκλίδεια απόσταση
	\item \textbf{Τετραγωνική Ευκλείδια(sqeuclidean) }: Τετραγωνική ευκλείδια απόσταση.
	\item \textbf{Cityblock }: To άθροισμα της απόλυτης διαφόρας γνωστή και ως $L1$ απόσταση.
	\item \textbf{Cosine }: Απόσταση που εμπεριέχει το συνημίτονο της γωνίας των σημείων.
	\item \textbf{Correlation }: Απόσταση που εμπεριέχει την συσχέτηση των σημείων.
	\item \textbf{Hamming }: Απόσταση που χρησιμοποιείται για δυαδικά δεδομένα. Είναι το ποσοστό των bit που διαφέρουν.
\end{enumerate}

Παρακάτω φαίνεται μια ομαδοποίηση που πραγματοποιήθηκε με K-means. Είναι ευδιάκριτα τόσο τα 3 clusters που δημιουργήθηκαν όσο και τα κέντρα τους. Κάθε σημείο του κάθε cluster απέχει την ελάχιστη απόσταση από το κέντρο του cluster στο οποιό ανήκει.

	\begin{figure}
\centering
\includegraphics[width=0.7\linewidth]{../../../../../Dropbox/protypa-figs/pictures-2/kmeans}
\caption{}
\label{fig:kmeans}
\end{figure}


Τέλος βλέπουμε ένα παράδειγμα μιας ομαδοποίησης ενός dataset μέσω του αλγορίθμου K-means ανάλογα με το βήμα στο οποίο βρίσκεται. Βλέπουμε πως μεταβάλλονται τα κέντρα με το πέρασμα των επαναλήψεων και έτσι και τα σημεία που ανήκουν σε κάθε cluster. Στο τελευταίο βήμα παρατηρούμε πάλι ότι το σημείο κάθε cluster απέχει ελάχιστη απόσταση από το κέντρο του cluster στο οποίο βρίσκεται και συνεπώς δεν χρειάζεται να γίνει άλλη επανάληψη και ο αλγόριθμος έχει τερματιστεί.

\begin{figure}
\centering
\includegraphics[width=0.7\linewidth]{../../../../../Dropbox/protypa-figs/pictures-2/kmeans_change_centroids}
\caption{}
\label{fig:kmeans_change_centroids}
\end{figure}




\subsection{Κ-medoids}

\subsubsection{Εισαγωγή}
Ο K-medoids είναι μία παραλλαγή του k-means που χρησιμοποιείται σε κατηγορικά η διακριτά δεδομένα.
Η βασική του διαφορά είναι ότι χρησιμοποιεί σαν κέντρο του cluster ένα σημείο από αυτό το επονομαζόμενο medoid.

Γενικά θεωρείται πιο ανθεκτικός στο θόρυβο επειδή προσπαθεί να ελαχιστοποιήσει την ανομοιότητα ανάμεσα σε στοιχεία παρά το τετραγωνικό άθροισμα των αποστάσεων όπως κάνει συχνά ο k-means.
Σαν medoid επιλέγεται συνήθως το σημείο το οποίο διαφέρει λιγότερα από όλα τα σημεία του cluster.

Ένας από του πιο συχνούς αλγορίθμους του k-medoids είναι o \textbf{Partitioning Around Medoids (PAM)}.
Ο τρόπος με τον οποίο δουλεύει
περιγράφεται με τον ακόλουθο ψευδοκώδικα:\\
\begin{algorithm}[H]
    Αρχικοποίηση$\:$διάλεξε στην τύχη $\kappa$ σημεία ως medoids\;
    Συσχέτισε το κάθε σημείο με το κοντινότερο medoid\;
    \While{το συνολικό κόστος μειώνεται(συνολικό κόστος των cluster)}{
        \For{κάθε σημείο medoid $m$, για κάθε μη-medoid σημείο $o$}{
            Άλλαξε(swap) το $m$ με το $o$ \;
            Ξαναυπολόγισε το κόστος του cluster(άθροισμα αποστάσεων από το medoid) \;
            \If{το συνολικό κόστος αυξηθεί}{
                Αναίρεσε την αλλαγή(swap)
            }
        }
    }
\end{algorithm}

Ένα παράδειγμα του παραπάνω αλγορίθμου φάινεται στο παρακάτω σχήμα:\\
\noindent\begin{minipage}{\linewidth}
    \centering
    \captionsetup{type={figure}}
	\makebox[\linewidth]{
	\includegraphics{images/kmedoid}}
	\captionof{figure}{Παράδειγμα k-medoid}\label{fig:pam}
\end{minipage}

\subsubsection{Πειραματικά Αποτελέσματα}

Καθώς ο K-medoids αποτελεί μια παραλλαγή του K-means πιο ανθεκτική στον θόρυβο δεν περιμένουμε να έχουμε καλύτερα αποτελέσματα με την εφαρμογή του καθώς δεν έχουμε ανεπιθύμητο θόρυβο στα δεδομένα μας. Τρέξαμε όλα τα dataset μας για τον αλγόριρμο K-medoids στο Matlab χρησιμοποιώνας την εντολή \lstinline[language=MATLAB]!kmedoids(X,clnumber,'Distance',correlation);!
Η μόνη παράμετρος που αλλάξαμε ήταν η απόσταση και χρησιμοποιήσαμε τον τύπο correlation. Τρέξαμε για είδος απόστασης cosine και τα αποτελέσματα ήταν ίδια. Παρατηρήσαμε ότι οι άλλες παράμετροι της εντολής \lstinline[language=MATLAB]!kmedoids! δεν δίνουν διαφορετικά αποτελέσματα για αυτό χρησιμοποιήθηκαν οι default τιμές τους. Για να πάρουμε τα διαγράμματα με τις μετρικές τρέξαμε το script  \lstinline[language=MATLAB]!scriptm!. Αυτό το script καλεί την συνάρτηση \lstinline[language=MATLAB]!optimizer_kmedoids! η οποία διαβάζει τα dataset μας, υλοποιεί τον αλγόριθμο K-medoids και μας βγάζει τα διαγράμματα στα οποία έχουμε τις μετρικές μας. Η φόρτωση των dataset μας γίνεται με την συνάρτηση \lstinline[language=MATLAB]!file_paths! και τα διαγράμματα με την συνάρτηση \lstinline[language=MATLAB]!plot_bars!.

Το διάγραμμα που παίρνουμε είναι το εξής:
\includegraphics{images/MedCorBar}

Παρατηρούμε ότι τα αποτελέσματα μας είναι αρκετά ικανοποιητικά, δεν φτάνουν όμως τα επίπεδα των αποτελεσμάτων του K-means.

Παρακάτω βλέπουμε τα αποτελέσματα από τις βέλτιστες ομαδοποιήσεις ως προς τις μετρικές $SuccessRate$ και $Silhouette$ αντίστοιχα.
\noindent\begin{minipage}{\linewidth}
    \centering
    \captionsetup{type={figure}}
    \includegraphics{images/kmedoid_result1}
    \includegraphics{images/kmedoid_2}
    \captionof{figure}{TODO}
    \label{fig:kmedoid_result}
\end{minipage}

Παρατηρούμε ότι και στις 2 ομαδοποιήσεις ενώ πετυχαίνουμε υψηλά ποσοστά στην ομαδοποίηση των μεγάλων cluster χάνουμε πιο πολλές βιβλιοθήκες από πριν.

Τέλος στο παρακάτω πινακάκι παρουσιάζονται μαζεμένες οι βέλτιστες λύσεις για το $Silhouette$ και για το $Success Rate$ για την ομαδοποίηση που έγινε.
\begin{table}[]
	\centering
	\resizebox{\textwidth}{!}{%
		\begin{tabular}{llllll}
			\cline{4-4}
			Dataset & Cohesion & \multicolumn{1}{l|}{Separation} & \multicolumn{1}{l|}{Silhouette} & Success Rate & Max Type \\ \cline{4-4}
			&  &  &  &  & Best Success Rate \\
			&  &  &  &  & Best Silhouette
		\end{tabular}
	}
	\caption{K-medoids Best Results}
	\label{table:k-medoids-best}
\end{table}
\chapter{Ιεραρχικοί αλγόριθμοι}
Οι ιεραρχικοί αλγόριθμοι (hierarchical cluster analysis or HCA) είναι μία μεγάλη κατηγορία αλγορίθμων που χρησιμοποιούνται για να κάνουμε ομαδοποίηση δεδομένων .Ονομάζονται έτσι επειδή προσπαθούν να δημιουργήσουν μία ιεραρχία από συστάδες (clusters).

Οι στρατηγικές που χρησιμοποιούνται είναι γενικά 2:
\begin{itemize}
\item{ \textbf{Συνάθροισης (Agglomerative)} Είναι μία μέθοδος 'bottom-up' δηλαδή : στην αρχή κάθε cluster αποτελείται από μία παρατήρηση και μετά ανά 2 'συναθροίζονται' και γίνονται ένα cluster όσο ανεβαίνουμε την ιεραρχία  }
\item{}\textbf{Διαχωρισμού (Divisive)} Είναι ουσιαστικά το ανάποδο 'top dow' όπου αρχικά όλες οι παρατηρήσεις είναι ένα cluster και έπειτα διαχωρίζονται όσο κατεβαίνουνε την ιεραρχία.Τέτοιος αλγόριθμος είναι το Ελαφρύτατου Συνδετικού Δέντρο (Minimum Spanning Tree))
\end{itemize}

Εφαρμόστηκαν αλγόριθμοι με την μέθοδο της συνάθροισης για αυτό και θα αναλύσουμε κυρίως   αυτούς αν και υπάρχουν αρκετά κοινά σημεία στις 2 μεθόδους
Σε αυτούς τους Αλγορίθμους 2 είναι οι βασικές έννοιες που χρειάζονται .
Η πρώτη είναι η μετρική(metric) με την οποία θα μετράμε την απόσταση ανάμεσα σε 2 clusters και η δεύτερη είναι ο τρόπος με τον οποίος θα συνδέουμε τα κοντινότερα clusters (linkage)

Μερικές μετρικές που χρησιμοποιούνται σαν απόσταση είναι:
\begin{enumerate}
\item \textbf{Ευκλείδια }H κλασική ευκλίδεια απόσταση
\item \textbf{Mahalanobis }Απόσταση προσαρμοσμένη στην διασπορά
\item \textbf{Correlation }??????????????????
\item \textbf{Minkowsi }  ??????????????
\end{enumerate}
Μπορούν να χρησιμοποιηθούν πολλές διαφορετικές μετρικές ανάλογα με το πρόβλημα και την περίσταση.

\begin{minipage}{\linewidth}% to keep image and caption on one page
H απόσταση ανάμεσα στα   clusters  συνήθως οπτικοποιείται με έναν πίνακα ομοιότητας χρωματισμένο σαν θερμοκρασία .το λεγόμενο heatmap  όπως το παρακάτω :
\makebox[\linewidth]{%        to center the image
  \includegraphics[keepaspectratio=true,scale=0.3]{images/heat1}}
\captionof{figure}{Πίνακας ομοιότητα από τα δεδομένα της άσκησης}\label{fig:heat1}%      only if needed  
\end{minipage}

Οι πιο γνωστοί τρόποι για linkage είναι 
\begin{enumerate}
\item \textbf{Max or Complete }Η μεγαλύτερη απόσταση ανάμεσα σε 2 cluster
\item \textbf{Min or Single}H μικρότερη απόσταση ανάμεσα σε 2 clusters 
\item \textbf{Average }Ο μέσος όρος της απόστασης  ανάμεσα 2 clusters 
\item \textbf{Centroid } απόσταση ανάμεσα στα κέντρα των clusters.
\end{enumerate}
Εννοείται ότι η απόσταση εξαρτάται από την μετρική που έχουμε επιλέξει.


 Αφού έχουμε επιλέξει την μετρική της απόστασης και της σύνδεσης η διαδικασία που ακολουθείται για να δημιουργηθεί η ιεραρχία μπορεί να περογραφέι από τον παρακάτω ψευδοκώδικα:
 \newline 
 \indent
\begin{algorithm}[H]
Calculate proximity\_Matrix\\
Define each point as cluster\\
 \While{num\_clust bigger than 1}{\\
  link closest clusters\\
  update proximity\_Matrix\\
  num\_clust +=1
 }
\end{algorithm}

\begin{minipage}{\linewidth}% to keep image and caption on one page
Ένας σύνηθες τρόπος να το δούμε είναι η  χρήση ενός δένδρο-διαγράμματος
\makebox[\linewidth]{%        to center the image
  \includegraphics[keepaspectratio=true,scale=0.6]{images/hier_tree}}
\captionof{figure}{Παράδειγμα δεντρο-διαγράμματος}\label{fig:dentro1}%      only if needed  
\end{minipage}
 
Τέλος αφού έχουμε την ιεραρχία μπορούμε να σπάμε τους πιο ασθενείς συνδέσμους(πιο ψηλούς στο δένδρο-διαγράμμα ) και να δημιουργούμε μικρότερα clusters.Το κάνουμε ώσπου να έχουμε τον επιθυμητό αριθμό από clusters
ή όταν όλες οι συνδέσεις είναι αρκετά ισχυρές.



\subsection{Περιγραφή εξερεύνησης των ιεραρχικών μοντέλων}

Τα μοντέλα μας δημιουργήθηκαν με την χρήση του Matlab και η διαδικασία που περιγράφτηκε στην εισαγωγή πραγματοποιείται στο αρχείο optimizer\_hier.m
για τις διάφορες παραμέτρους. Οι αποστάσεις ανάμεσα στα σημεία ή υπολογίζονται για τους ιεραρχικούς με την χρήση της συνάρτησης pdist , η οποία μπορεί να υπολογίσει τις εξής αποστάσεις:
\begin{itemize}
    \item euclidean
    \item seuclidean
    \item Mankowski
    \item chebychev
    \item mahalanobis
    \item cosine
    \item correlation
    \item spearman
    \item jaccard 
\end{itemize}


Περισσότερες πληροφορίες για την pdist \href{http://www.mathworks.com/help/stats/pdist.html}{εδώ}
Σαν αποστάσεις ικανοποιητικά αποτελέσματα έδιναν μόνο το correlation και το 
cosine  μία πιθανή εξήγηση βρίσκεται \textbf{(Εδώ τα reference})

Έπειτα χρησιμοποιήθηκε η συνάτηση linkage οποία δημιουργεί ουσιαστικά το δέντρο ιεραρχίας .Οι μετρική που χρησιθμοποιεί για να συνδέσει τα κοντινοερα cluster μπορεί να είναι μία από της εξής:
\begin{itemize}
    \item single
    \item complete
    \item average
    \item weighted
    \item centroid
    \item median
    \item ward 
  \end{itemize}

Από άποψη χρόνου και αποτελεσμάτων ύστερα από δοκιμές για τα τελικά πειράματα επιλέχτηκαν οι weighted ,ward, complete,average.

Έπειτα με την χρήση της συνάρτησης cluster δημιουργήσαμε τα τελικά cluster που θέλαμε τα παραπάνω βήματα υλοποιούνται από τις παρακάτω γραμμές κώδικα 


\begin{lstlisting}[language=Matlab]
%simple example of hierarchical clustering
Y = pdist (X,distance); %X is an array containig the data
YY = squareform(Y); %convert Y in a square form
Z =linkage(YY,); %find  hierarchical cluster tree,
CDX = cluster(Z,'maxclust',8);
\end{lstlisting}


επίσης υπάρχει η συνάρτηση cophonet(Z,YY) η οποία είναι μια μετρική που υπολογίζει την συσχέτιση ανάμεσα στις συνδέσεις που έχει δημιουργήσει το Z και τις αντίστοιχες αποστάσεις στο YY .Όσο πιο μεγάλη είναι η συσχέτιση τόσο πιο καλά έχει αποτυπωθεί η διαφορετικότητα των αρχικών σημείων, σαν συνδέσεις μεταξύ clusters , περισσότερα
\href{https://en.wikipedia.org/wiki/Cophenetic_correlation}{εδώ}

Στη συνέχεια παρουσιάζονται και σχολιάζονται τα πειραματικά αποτελέσματα.

\section{Πειραματικά αποτελέσματα}

Χρησιμοποιούνται οι γνωστές μετρικές $Silhouette$ ,$Cohesion$,$Separation$
αλλά και η $Success Rate 1$ η οποία είναι μία από τις 2 μετρικές που  υπολογίζει το πόσα πετύχαμε από τα πραγματικά δεδομένα.Περισσότερα για το πως υπολογίστηκαν και τη αντιπροσωπεύει η success1 στο \textbf{TELOS} του κεφαλαίου.    

Έγινε χρήση των προαναφερθέντων αλγορίθμων απόστασης και εφαρμόστηκαν στον καθένα 4 διαφορετικοί τρόποι σύνδεσης για όλα τα dataset.

\noindent\begin{minipage}{\linewidth}
    \centering
    \captionsetup{type={figure}}
    \includegraphics[width=\linewidth]{images/hierCosBar.pdf}
    \captionof{figure}{Μετρικές για τον ιεραρχικό αλγόριθμο με την χρήση του Cosine}
    \label{fig:CosineHier}
\end{minipage}

\noindent\begin{minipage}{\linewidth}
    \centering
    \captionsetup{type={figure}}
    \includegraphics[width=\linewidth]{images/hierCorBar.pdf}
    \captionof{figure}{Μετρικές για τον ιεραρχικό αλγόριθμο με την χρήση του Cosine}
    \label{fig:hierCorBar}
\end{minipage}

\newpage
Βλέπουμε ότι και στις 2 περιπτώσεις ότι όσο υψηλότερο είναι το πσοσοστό επιτυχία
τόσο υψηλότερο είναι  η μετρική $Silhouette$ και η $Separation$.Αντιθετα H $Cohesion$ όσο αυξάνεται το ποσοστό επιτυχίας είναι μικρότερη. Αυτά ήταν και τα επιθυμητά αποτελέσματα . Η μετρικές μας να βελτιώνονται όταν πετυχαίνουμε καλύτερα αποτελέσματα. 
Τα καλυτερα ποσοστά σημειώνονται από τις μεθόδους average και weighted
και όσο αφορά τα dataset βλέπουμε ότι υπάρχουν διαφορές αν και μικρές

Να σημειώσουμε ότι το αρχικό dataset δεν συμπεριλαμβάνεται σε αυτές τις μετρήσεις

\noindent\begin{minipage}{\linewidth}
    \centering
    \captionsetup{type={figure}}
    \includegraphics[width=\linewidth]{images/hierCorBar.pdf}
    \captionof{figure}{Μετρικές για τον ιεραρχικό αλγόριθμο με την χρήση του Cosine}
    \label{fig:CorrelationHier}
\end{minipage}

To dataset που έδωσε καλύτερο(αν και υπήρχαν ισοβαθμίες) $Silhouette$ και $Success Rate 1$ ήταν το ίδιο: \\ \url{join_duplicates/freq_8_70/gibberish_detector/join_similar/dataset.csv-average}\\

\begin{minipage}{\linewidth}
    \centering
    \captionsetup{type={table}}
    \resizebox{\textwidth}{!}{%
        \begin{tabular}{lllllllll}
            \cline{7-7}
            Dataset & Method & Distance Type & Number of Clusters & Cohesion & \multicolumn{1}{l|}{Separation} & \multicolumn{1}{l|}{Silhouette} & Success Rate & Max Type \\ \cline{7-7}
            Α & average & Cosine & 8 & 0.351 & 0.504 & 0.229 & 0.787 & Best Success Rate \\
            A & average & Correlation & 8 & 0.360 & 0.504 & 0.259 & 0.231& Best Silhouette \\
        \end{tabular}
    }
    \captionof{table}{TODO}
    \label{my-label}
\end{minipage}

Επίσης παραθέτουμε το σχήμα που προέκυψε από το weka και δείχνει ποιες
βιβλιοθήκες μπήκαν σε ποια Clusters (Προέκυψε το ίδιο και για best $Success Rate 1$ και για $Silhouette$ ,λογικά υπήρχαν αμοιβαίες ανταλαες ανάμεσα στα clusters)

\noindent\begin{minipage}{\linewidth}
    \centering
    \captionsetup{type={figure}}
    \includegraphics[width=\linewidth]{images/hier_result.eps}
    \captionof{figure}{Libraries σε Clusters}
    \label{fig:clustering}
\end{minipage}

Βλέπουμε ότι η τα swing-libraries(κόκκινα) τα http-clients(γκρι), 
και τα command-line parsers(λαχανί) τα πετύχαμε ακριβώς με ενώ χάθηκαν 
λίγα από τα android σε γειτονικά clusters.to Cluster 4 έμεινε μόνο με 2 στοιχεία ενώ το 3 είχε μεγάλη ανομοιογένεια το 2 είχε την πλειοψηφία των xml(φουξ),json(ροζ),csv(γαλάζιο)
Μπορούμε να χαρακτηρίσουμε αυτό το αποτέλεσμα ώς μέτριο προς καλό
καθώς πετύχαμε ακριβώς 3 από τα 8 clusters και μία σχετική πληοψηφία από τα υπόλοιπα.


Τέλος παραθέτουμε τον πίνακα ομοιότητας και το δεντροδιάγραμμα για το καλύτερο dataset και clustering

\noindent\begin{minipage}{\linewidth}
    \centering
    \captionsetup{type={figure}}
    \includegraphics[width=\linewidth]{images/heatHier.eps}
    \captionof{figure}{Το δεντροδιάγραμμα με τα καλύτερα αποτελέσματα}
    \label{fig:Heat}
\end{minipage}

Είναι εμφανές ότι οι μετρική της απόστασης που διαλέξαμε είναι σωστή
καθώς τα clusters διαφαίνονται σχετικά με το ματι

\noindent\begin{minipage}{\linewidth}
    \centering
    \captionsetup{type={figure}}
    \includegraphics[width=\linewidth]{images/dentroHier.eps}
    \captionof{figure}{Το δεντροδιάγραμμα με τα καλύτερα αποτελέσματα}
    \label{fig:Dentro}
\end{minipage}

Τέλος να προσθέσουμε ότι το $coph\_coeff = 0.7067$ για το συγκεκριμένο δεντροδιάγραμμα και πίνακα ομοιότητας.

\chapter{Ανοιχτά θέματα}
Στο πλαίσιο της συγκεκριμένης εργασίας προέκυψαν ορισμένα θέματα τα οποία θα μπορούσαμε να έχουμε διαχειριστεί διαφορετικά. Αυτά τα θέματα αφορούν τα dataset μας , το στάδιο της προ-επεξεργασίας, τους αλγορίθμους που χρησιμοποιήσαμε και διάφορες παραμέτρους μέσα σε αυτές καθώς και τις μετρικές για την αξιολόγηση των μοντέλων. Ορισμένα από τα θέματα αυτά είναι:

\begin{enumerate}
    \item Κατά την διάρκεια της προ-επεξεργασίας επιλέξαμε να βγάλουμε ορισμένα outliers τα οποία θεωρήθηκε ότι δεν προσδίδουν καμία χρήσιμη πληροφορία. Η επιλογή μας αυτή έγινε μετά από εξερεύνηση του dataset μας και θεωρούμε ότι έγινε μια αρκετά αντιπροσωπευτική επιλογή. Ωστόσο θα μπορούσαμε να έχουμε δοκιμάσει να απομακρύνουμε περισσότερα outliers. Στην περίπτωση που είχαμε ένα διαφορετικό αρχικό dataset η επιλογή αυτή θα ήταν διαφορετική.
    \item Μία ιδέα που θα μπορούσε να εφαρμοστεί είναι να απομακρυνθούν τα features τα οποία έχουν πολλά distinct values.
    \item Επίσης, κατά την διάρκεια της προ-επεξεργασίας δοκιμάσαμε να απομακρύνουμε τις λέξεις (feature) τα οποία έχουν πολύ χαμηλό ή πολύ υψηλό variance. Αυτή μας η προσπάθεια δεν οδήγησε σε καλύτερα αποτελέσματα και για αυτόν τον λόγο δεν χρησιμοποιήθηκε τελικά.
    \item Για την επιπλέον μείωση του αριθμού των λέξεων μας από τα τελικά dataset μπορούμε να χρησιμοποιήσουμε την τεχνική της Principal Component Analysis (PCA). Αν και δεν θα βελτίωνε τα αποτελέσματα της ομαδοποίησης μας θα παρουσίαζε ένα πιο μικρό σε μέγεθος τελικό dataset.
    \item Χρησιμοποιήσαμε αλγορίθμους Διαχωριστικούς και Ιεραρχικούς. Για τον υπολογισμό των αποστάσεων που απαιτούνται χρησιμοποιήσαμε στο Matlab όλες των ειδών τις αποστάσεις αλλά επικεντρωθήκαμε στις αποστάσεις τύπου cosine και correlation καθώς αυτές ήταν οι πιο αποτελεσματικές για το dataset μας.
    \item Θα μπορούσαμε να χρησιμοποιήσουμε και άλλους αλγορίθμους ομαδοποίησης όπως τους Πυκνωτικούς (DBSCAN), Particle Swarm Optimazation(PSO) ή Διαφορικούς(Differential Evaluation). Ακόμα θα μπορούσαμε να αναπτύξουμε σε μεγαλύτερο βαθμό τους Γενετικούς Αλγορίθμους (GA).
    \item Για την φάση της αξιολόγησης της ομαδοποίησης αν και υπολογίσαμε όλες τις μετρικές (SSE,Cohesion,Separation,Silhouette) δώσαμε μεγαλύτερη βαρύτητα στην μετρική Silhouette η οποία συνδυάζει τις $Seperation$ και $Cohesion$. Θα μπορούσαμε να έχουμε επιλέξει ένα άλλο βέλτιστο μοντέλο αν δίναμε βαρύτητα σε άλλη μετρική.
    \item Τέλος για καλύτερη ομαδοποίηση θα μπορούσαμε να χρησιμοποιήσουμε την τεχνική της μετά-επεξεργασίας (post-processing). Γενικά είναι μια καλή τεχνική για να ελαχιστοποιήσουμε το SSE να βρούμε περισσότερα clusters (επιλέγοντας μεγαλύτερο Κ). Εμείς όμως επειδή θέλουμε ο τελικός αριθμός των ομάδων να είναι ίσος με 8 μπορούμε να αλλάξουμε το συνολικό SSE υλοποιώντας διάφορες λειτουργίες πάνω στις ομάδες, όπως να διαχωρίζουμε ή να ενώνουμε ομάδες. Μία αρκετά διαδεδομένη τεχνική είναι να χρησιμοποιούμε φάσεις διαχωρισμού και συνένωσης. Κατά την διάρκεια του διαχωρισμού, οι ομάδες διαιρούνται, ενώ κατά την διάρκεια της συνένωσης οι ομάδες συνδυάζονται. Με αυτόν τον τρόπο είναι πιθανό να αποφύγουμε να κολλήσουμε σε κάποιο τοπικό ελάχιστο SSE και να επιτύχουμε την βέλτιστη ομαδοποίηση με τον επιθυμητό αριθμό ομάδων.
    
     Οι 2 στρατηγικές για να μειώσουμε το συνολικό SSE αυξάνοντας τον αριθμό των ομάδων είναι η εξής:
    \begin{itemize}
        \item \textbf{Διαχωρισμός Ομάδας} : Η ομάδα με το μεγαλύτερο SSE επιλέγεται συνήθως για να διασπαστεί. Ακόμα θα μπορούσαμε να διασπάσουμε μια ομάδα με την μεγαλύτερη τυπική απόκλιση για ένα συγκεκριμένο γνώρισμα.
        \item \textbf{Δημιουργία ενός νέου Κέντρου Ομάδας} : Συχνά επιλέγεται το σημείο που είναι πιο μακριά από το κέντρο μιας ομάδας.
    \end{itemize} 
    
    Οι 2 στρατηγικές για να μειώσουμε τον συνολικό αριθμό των ομάδων καθώς προσπαθούμε να διατηρήσουμε ελάχιστο το συνολικό SSE είναι οι εξής:
    
        \begin{itemize}
            \item \textbf{Διάλυση Ομάδας} : Αυτό επιτυγχάνεται διαγράφοντας το κέντρο που αντιστοιχεί σε μία ομάδα και αντιστοιχίζοντας τα σημεία σε άλλες ομάδες. Ιδανικά η ομάδα που διαλύεται είναι αυτή που αυξάνει το συνολικό SSE λιγότερο.
            \item \textbf{Συνένωση 2 Ομάδων} : Συνήθως επιλέγονται οι ομάδες με τα κοντινότερα κέντρα αν και ίσως μια καλύτερη τεχνική θα ήταν να επιλέξουμε να συνενώσουμε 2 ομάδες που οδηγούν στην ελάχιστη αύξηση του συνολικού SSE.
        \end{itemize} 
        
        Αυτές οι 2 τεχνικές είναι αυτές που χρησιμοποιούνται σε διάφορους ιεραρχικούς αλγορίθμους ομαδοποίησης. 
        
        Στην περίπτωση μας θα μπορούσαμε στην περίπτωση των διαχωριστικών αλγορίθμων να επιλέξουμε αρχικά ένα μεγαλύτερο αριθμό ομάδων για παράδειγμα Κ=12 και στην συνέχεια εφαρμόζοντας μία από τις 2 στρατηγικές μείωσης ομάδων να φτάσουμε στον τελικό αριθμό ομάδων κρατώντας το συνολικό SSE όσο το δυνατόν μικρότερο μπορούμε.  
\end{enumerate}

\end{document}
