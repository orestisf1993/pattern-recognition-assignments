\chapter{Ανοιχτά θέματα}
Στο πλαίσιο της συγκεκριμένης εργασίας προέκυψαν ορισμένα θέματα τα οποία θα μπορούσαμε να έχουμε διαχειριστεί διαφορετικά. Αυτά τα θέματα αφορούν τα dataset μας , το στάδιο της προ-επεξεργασίας, τους αλγορίθμους που χρησιμοποιήσαμε και διάφορες παραμέτρους μέσα σε αυτές καθώς και τις μετρικές για την αξιολόγηση των μοντέλων. Ορισμένα από τα θέματα αυτά είναι:

\begin{enumerate}
	\item Κατά την διάρκεια της προ-επεξεργασίας επιλέξαμε να βγάλουμε ορισμένα outliers τα οποία θεωρήθηκε ότι δεν προσδίδουν καμία χρήσιμη πληροφορία. Η επιλογή μας αυτή έγινε μετά από εξευρεύνηση του dataset μας και θεωρούμε ότι έγινε μια αρκετά αντιπροσωπευτική επιλογή. Ωστόσο θα μπορούσαμε να έχουμε δοκιμάσει να απομακρύνουμε περισσότερα outliers. Στην περιπτώση που είχαμε ένα διαφορετικό αρχικό dataset η επιλογή αυτή θα ήταν διαφορετική.
	\item Επίσης, κατά την διάρκεια της προ-επεξεργασίας δοκιμάσαμε να απομακρύνουμε τις λέξεις (feature) τα οποία έχουν πολύ χαμηλό ή πολύ υψηλό variance. Αυτή μας η προσπάθεια δεν οδήγησε σε καλύτερα αποτελέσματα και για αυτόν τον λόγο δεν χρησιμοποιήθηκε τελικά.
	\item Για την επιπλέον μείωση του αριθμού των λέξεων μας από τα τελικά dataset μπορούμε να χρησιμοποιήσουμε την τεχνική της Principal Component Analysis (PCA). Αν και δεν θα βελτίωνε τα αποτελέσματα της ομαδοποίησης μας θα παρουσίαζε ένα πιο μικρό σε μέγεθος τελικό dataset.
	\item Χρησιμοποιήσαμε αλγορίθμους 
		Genetic Algorithm(GA)
		Particle Swarm Optimization(PSO)
		Differential Evaluation(DE)
\end{enumerate}