\chapter{Ομαδοποίηση}
Πριν προχωρήσουμε στους αλγορίθμους που χρησιμοποιήσαμε είναι χρήσιμο να αναφέρουμε 2 σημαντικά ζητήματα που αφορούν την ομαδοποίηση μας. Το πρώτο αφορά την μετρική που θα χρησιμοποιήσουμε για την αξιολόγηση της ομαδοποίησης και ο δεύτερος τον τρόπο με τον οποίο θα μετρήσουμε την απόσταση. Όσον αφορά την μετρική μπορούμε να αναφέρουμε ότι σαν κύρια μετρική θα χρησιμοποιήσουμε το $Silhouette$ καθώς μέσα σε αυτήν εμπεριέχονται οι μετρικές $Cohesion$ και $Separation$ και μπορεί να θεωρηθεί πιο αντιπροσωπευτική.
\section{Διαχωριστικοί Αλγόριθμοι}
Ο αλγόριθμος που κυρίως αντιπροσωπεύει αυτή την κατηγορία αλγορίθμων είναι ο K-means. Η βασική ιδέα των διαχωριστικών αλγορίθμων είναι η ανάθεση των σημείων μας σε ομάδες προσπαθώντας να ελαχιστοποιήσουμε την απόσταση από ένα σημείο που αντιπροσωπεύει την ομάδα. Έτσι κάθε σημείο μας θεωρούμε ότι ανήκει στην ομάδα αυτή στην οποία η απόσταση από το αντιπροσωπευτικό σημείο είναι ελάχιστη. Εκτός από τον K-means σε αυτήν την κατηγορία ανήκουν διάφορες παραλλαγές του όπως ο bisection K-means ,ο fuzzy Κ-means και ο K-medoid. Στην δικιά μας υλοποίηση χρησιμοποιήσαμε τον απλό K-means και τον K-medoid.

\subsection{K-means}

\subsubsection{Εισαγωγή}
Όπως αναφέραμε ο K-means είναι ο πιο συνηθισμένος και απλός διαχωριστικός αλγόριθμος. Η λογική που χρησιμοποιεί είναι η εξής:
\begin{enumerate}
    \item Κάθε ομάδα συνδέεται με ένα κέντρο (centroid) το οποίο είναι το αντιπροσωπευτικό σημείο της ομάδας.
    \item Κάθε σημείο αποδίδεται στην ομάδα με το πιο κοντινό κέντρο ελαχιστοποιώντας την μεταξύ τους απόσταση.
    \item Ο αριθμός των ομάδων Κ πρέπει να έχει καθοριστεί από πριν.
\end{enumerate}

Πιο συγκεκριμένα μπορούμε να αναλύσουμε τα βήματα του αλγορίθμου που ακολουθούμε ως εξής:
\begin{enumerate}
    \item Επιλέγουμε k σημεία ως αρχικά κέντρα.
    \item Δημιουργούμε k ομάδες με τον τρόπο που περιγράφτηκε.
    \item Υπολογίζουμε τα νέα κέντρα των ομάδων μας.
    \item Επαναλαμβάνουμε τα βήματα 2-3 μέχρις ότου δεν μεταβληθούν τα κέντρα.
\end{enumerate}

Παρακάτω παρουσιάζεται ο αλγόριθμος K-means σε ψευδογλώσσα:\\
\noindent\begin{minipage}{0.9\linewidth}
\centering
\begin{algorithm}[H]
    Select K points as the initial centroids.\;
    \Do{The centroids don't change}{
    Form K clusters by assigning all points to the closest centroid.\;
    Recompute the centroid of each cluster.\;
    }
\end{algorithm}
\end{minipage}

Αφού περιγράψαμε τις βασικές ιδέες του αλγορίθμου K-means μπορούμε πλέον να προχωρήσουμε σε ορισμένα σημαντικά θέματα που αφορούν τον αλγόριθμο K-means. Αυτά είναι ο αριθμός των ομάδων, η επιλογή των αρχικών κέντρων και ο τρόπος υπολογισμού της απόστασης.Ο τελικός αριθμός των ομάδων στην περίπτωση μας είναι ίσος με 8. Μπορούμε να δημιουργήσουμε περισσότερες ή λιγότερες ομάδες αρχικά και σταδιακά να φτάσουμε στον τελικό αριθμό. Όσον αφορά το θέμα της επιλογής των αρχικών κέντρων υπάρχουν διάφορες τεχνικές αντιμετώπισης αυτού του προβλήματος. Τα αρχικά κέντρα συνήθως επιλέγονται τυχαία. Αν και συνήθως αυτοί οι αλγόριθμοι συγκλίνουν με τυχαία επιλογή κέντρων υπάρχει πάντα η πιθανότητα να πέσουμε σε τοπικό ελάχιστο της προς ελαχιστοποίηση συνάρτησης. Για αυτόν τον λόγο χρησιμοποιήθηκαν 2 τρόποι αντιμετώπισης του προβλήματος των αρχικών τιμών.
\begin{enumerate}
    \item Τυχαία αρχικοποίηση των centroids αλλά επιλέγοντας να τρέξει πολλές φορές ο K-means. Έτσι ουσιαστικά τρέχουμε πολλές φορές τον αλγόριθμο ομαδοποίησης και επιλέγουμε κάθε φορά αυτόν που μας δίνει το ελάχιστο σφάλμα.

    \item Heuristic Μέθοδος επιλογής centroids.
    Υπάρχουν διάφορες τεχνικές επιλογής αρχικού κέντρου με Heuristic μεθόδους που προκύπτουν από την εμπειρία μας.
    Η τεχνική που χρησιμοποιήσαμε εμείς ακολουθά την παρακάτω λογική και έχει ως σκοπό την επιλογή Κ centroid, όσες και οι ομάδες μας.
    Επιλέγουμε σαν αρχική τιμή centroid το σημείο από τα δεδομένα μας που βρίσκεται πιο κοντά στον μέσο όρο των σημείων μας.
    Έτσι, έχουμε ένα centroid.
    Για το επόμενο, υπολογίζουμε τις αποστάσεις των σημείων μας από το centroid και ορίζουμε αυτό που βρίσκεται πιο μακριά από το centroid.
    Έτσι, έχουμε 2 centroids.
    Όμοια, προχωράμε επιλέγοντας σαν επόμενο centroid αυτό που απέχει περισσότερο από τα ήδη επιλεγμένα centroid.
    Συνεχίζουμε έτσι μέχρις ότου επιλέξουμε K centroids.
    Αναλυτικά τα βήματα του αλγορίθμου:\\
    \noindent\begin{minipage}{0.9\linewidth}
    \centering
    \begin{algorithm}[H]
        From n objects calculate a point whose attribute values are average of n-objects attribute values so first initial centroid is average of n-objects.\;
        Select next initial centroids from n-objects in such a way that the Euclidean distance of that object is maximum from other selected initial centroids.\;
        Repeat step2 until we get k initial centroids.
        From these steps we will get initial centroids and with these initial centroids perform K-means algorithm.\;
    \end{algorithm}
    \end{minipage}
\end{enumerate}

Ο παραπάνω αλγόριθμος υλοποιήθηκε στο Matlab μέσω της συνάρτησης
\lstinline[language=MATLAB]!initial_centroid=centroid_heuristic(X,number_of_features,total_centroid_counter)!

Όπου οι είσοδοι μας είναι:
\begin{enumerate}
    \item \lstinline[language=MATLAB]!X)! : Το dataset μας σε μορφή πίνακα.
    \item \lstinline[language=MATLAB]!number_of_features! : Ο αριθμός των γνωρισμάτων μας που στην περίπτωση μας είναι ο αριθμός των βιβλιοθικών μας ,δηλαδή 80.
    \item \lstinline[language=MATLAB]!total_centroid_counter! : Ο αριθμός των ομάδων που θέλουμε να δημιουργήσουμε.
\end{enumerate}

Έξοδος της συνάρτησης μας είναι τα Κ αρχικά κέντρα.


Ακόμα για να αποφύγουμε το πρόβλημα των αρχικών centroids πολλές φορές χρησμιποιείται ο bisecting K-means καθώς εξαρτάται λιγότερο από την αρχική επιλογή των κεντρών.

Η πολυπλοκότητα του αλγορίθμου είναι $ O(n*K*I*d)$ όπου $n$ είναι ο αριθμός σημείων, $Κ$ είναι ο αριθμός ομάδων, $l$ είναι αριθμός επαναλήψεων και $d$ είναι αριθμός μεταβλητών. Πρόκειται για έναν αρκετά γρήγορο αλγόριθμο.

Τέλος η τελευταία παράμετρος που επιλέγεται στον αλγόριθμο K-means είναι η απόσταση. Οι μετρικές που χρησιμοποιούνται σαν απόσταση είναι:
\begin{enumerate}
    \item \textbf{Τετραγωνική Ευκλείδια (sqeuclidean)}: Τετραγωνική ευκλείδια απόσταση.
    \item \textbf{Cityblock}: To άθροισμα της απόλυτης διαφοράς γνωστή και ως $L1$ απόσταση.
    \item \textbf{Cosine}: Απόσταση που εμπεριέχει το συνημίτονο της γωνίας των σημείων.
    \item \textbf{Correlation}: Απόσταση που εμπεριέχει την συσχέτιση των σημείων.
    \item \textbf{Hamming}: Απόσταση που χρησιμοποιείται για δυαδικά δεδομένα. Είναι το ποσοστό των bit που διαφέρουν.
\end{enumerate}

Ένα ακόμα πρόβλημα του K-means είναι ότι όταν οι ομάδες μας είναι ανισομεγεθής, έχουν διαφορετική πυκνότητα και μη σφαιρικά σχήματα. Τα παραπάνω προβλήματα επιλύονται με την δημιουργία πολλών μικρών ομάδων και την σύνθεση τους σε επίπεδο μετ-επεξεργασίας, θέμα το οποίο θα αναλυθεί παρακάτω εκτενέστερα στα ανοιχτά θέματα.
Παρακάτω φαίνεται μια ομαδοποίηση που πραγματοποιήθηκε με K-means. Είναι ευδιάκριτα τόσο τα 3 clusters που δημιουργήθηκαν όσο και τα κέντρα τους. Κάθε σημείο του κάθε cluster απέχει την ελάχιστη απόσταση από το κέντρο του cluster στο οποιό ανήκει.

\begin{figure}
    \centering
    \includegraphics[width=1.0\linewidth]{images/kmeans}
    \caption{}
    \label{fig:kmeans}
\end{figure}

Τέλος βλέπουμε ένα παράδειγμα μιας ομαδοποίησης ενός dataset μέσω του αλγορίθμου K-means ανάλογα με το βήμα στο οποίο βρίσκεται. Βλέπουμε πως μεταβάλλονται τα κέντρα με το πέρασμα των επαναλήψεων και έτσι και τα σημεία που ανήκουν σε κάθε cluster. Στο τελευταίο βήμα παρατηρούμε πάλι ότι το σημείο κάθε cluster απέχει ελάχιστη απόσταση από το κέντρο του cluster στο οποίο βρίσκεται και συνεπώς δεν χρειάζεται να γίνει άλλη επανάληψη και ο αλγόριθμος έχει τερματιστεί.

\begin{figure}
\centering
\includegraphics[width=1.0\linewidth]{images/kmeans_change_centroids}
\caption{}
\label{fig:kmeans_change_centroids}
\end{figure}

\subsubsection{Πειραματικά Αποτελέσματα}

Στην συνέχεια προχωρήσαμε στην υλοποίηση του αλγορίθμου K-means για τα διαφορετικά datasets που φτιάξαμε και περιγράφτηκαν παραπάνω. Τρέξαμε όλα τα datasets μας για τον αλγόριθμο K-means στο Matlab για διαφορετικά σετ παραμέτρων του αλγορίθμου μας. Τα διαφορετικά σετ παραμέτρων που χρησιμοποιήσαμε όπως αναφέραμε και παραπάνω αφορούν τόσο τον τρόπο υπολογισμού της απόστασης όσο και τον τρόπο επιλογής των αρχικών κεντρών. Όυσιαστικά υλοποιήσαμε 9 σετ παραμέτρων τα οποία παρουσιάζονται παρακάτω: 

\begin{enumerate}
    \item \textbf{1ο Σετ Παραμέτρων}: 
    \begin{itemize}
        \item \textbf{Τρόπος Υπολογισμού Απόστασης} : Cosine
        \item \textbf{Τρόπος Επιλογής Αρχικού Κέντρου} : Τυχαίος
    \end{itemize}
    \item  \textbf{2ο Σετ Παραμέτρων}:
    \begin{itemize}
        \item \textbf{Τρόπος Υπολογισμού Απόστασης} : Cosine
        \item \textbf{Τρόπος Επιλογής Αρχικού Κέντρου} : Τυχαίος με επανάληψη αλγορίθμου 10 φορές
    \end{itemize}
    \item  \textbf{3ο Σετ Παραμέτρων}:
        \begin{itemize}
            \item \textbf{Τρόπος Υπολογισμού Απόστασης} : Cosine
            \item \textbf{Τρόπος Επιλογής Αρχικού Κέντρου} : Επιλογή μέσω του ευρυστικού κανόνα.
        \end{itemize}
    \item  \textbf{4ο Σετ Παραμέτρων}:
        \begin{itemize}
            \item \textbf{Τρόπος Υπολογισμού Απόστασης} : Correlation
            \item \textbf{Τρόπος Επιλογής Αρχικού Κέντρου} : Τυχαίος
        \end{itemize}
    \item  \textbf{5ο Σετ Παραμέτρων}:
        \begin{itemize}
            \item \textbf{Τρόπος Υπολογισμού Απόστασης} : Correlation
            \item \textbf{Τρόπος Επιλογής Αρχικού Κέντρου} : Τυχαίος με επανάληψη αλγορίθμου 10 φορές
        \end{itemize}
    \item  \textbf{6ο Σετ Παραμέτρων}:
        \begin{itemize}
            \item \textbf{Τρόπος Υπολογισμού Απόστασης} : Correlation
            \item \textbf{Τρόπος Επιλογής Αρχικού Κέντρου} :  Επιλογή μέσω του ευρυστικού κανόνα.
        \end{itemize}
\end{enumerate}

Τα παραπάνω σετ παραμέτρων υλοποιούνται στο Matlab μέσω των εντολών:
\begin{enumerate}
    \item  \textbf{1ο Σετ Παραμέτρων}: \lstinline[language=MATLAB]!kmeans(X,clnumber,'Distance',cosine);!
    Όπου \lstinline[language=MATLAB]!X! είναι το σύνολο δεδομένων εισόδου, \lstinline[language=MATLAB]!clnumber! είναι ο αριθμός των ομάδων που δημιουργούνται και η παράμετρος \lstinline[language=MATLAB]!Distance! που δηλώνει τον τρόπο υπολογισμού της απόστασης ορίζεται να ισούται με \lstinline[language=MATLAB]!cosine!. Δεν δηλώνεται τίποτα για τις αρχικές τιμές των κέντρων τον ομάδων.
    \item  \textbf{2ο Σετ Παραμέτρων}:
\lstinline[language=MATLAB]!kmeans(X,clnumber,'Distance',cosine'Replicates',number_of_iretation);!
Όπου \lstinline[language=MATLAB]!X! είναι το σύνολο δεδομένων εισόδου, \lstinline[language=MATLAB]!clnumber! είναι ο αριθμός των ομάδων που δημιουργούνται, η παράμετρος \lstinline[language=MATLAB]!Distance! δηλώνει τον τρόπο υπολογισμού της απόστασης ορίζεται να ισούται με \lstinline[language=MATLAB]!cosine! και η παράμετρος \lstinline[language=MATLAB]!Replicates! δηλώνει την επανάληψη του K-means \lstinline[language=MATLAB]!number_of_iretaion! φορές που στην περίπτωση μας επιλέχτηκε 10. 
    \item  \textbf{3ο Σετ Παραμέτρων}:
\lstinline[language=MATLAB]!kmeans(X,clnumber,'Distance',cosine,'Start',heuristic_centroid);!
Όπου \lstinline[language=MATLAB]!X! είναι το σύνολο δεδομένων εισόδου, \lstinline[language=MATLAB]!clnumber! είναι ο αριθμός των ομάδων που δημιουργούνται και η παράμετρος \lstinline[language=MATLAB]!Distance! που δηλώνει τον τρόπο υπολογισμού της απόστασης ορίζεται να ισούται με \lstinline[language=MATLAB]!cosine!. Η παράμετρος \lstinline[language=MATLAB]!Start! δηλώνει την αρχική επιλογή των κέντρων και ισούται με \lstinline[language=MATLAB]!heuristic_centroid! που προκύπτουν από την κλήση της συνάρτησης \lstinline[language=MATLAB]!centroid_heuristic! που περιγράφτηκε παραπάνω.
    \item  \textbf{4ο Σετ Παραμέτρων}:
\lstinline[language=MATLAB]!kmeans(X,clnumber,'Distance',correlation);!
Όπου \lstinline[language=MATLAB]!X! είναι το σύνολο δεδομένων εισόδου, \lstinline[language=MATLAB]!clnumber! είναι ο αριθμός των ομάδων που δημιουργούνται και η παράμετρος \lstinline[language=MATLAB]!Distance! που δηλώνει τον τρόπο υπολογισμού της απόστασης ορίζεται να ισούται με \lstinline[language=MATLAB]!correlation!. Δεν δηλώνεται τίποτα για τις αρχικές τιμές των κέντρων τον ομάδων.
    \item  \textbf{5ο Σετ Παραμέτρων}:
\lstinline[language=MATLAB]!kmeans(X,clnumber,'Distance',correlation'Replicates',number_of_iretation);!
Όπου \lstinline[language=MATLAB]!X! είναι το σύνολο δεδομένων εισόδου, \lstinline[language=MATLAB]!clnumber! είναι ο αριθμός των ομάδων που δημιουργούνται, η παράμετρος \lstinline[language=MATLAB]!Distance! δηλώνει τον τρόπο υπολογισμού της απόστασης ορίζεται να ισούται με \lstinline[language=MATLAB]!correlation! και η παράμετρος \lstinline[language=MATLAB]!Replicates! δηλώνει την επανάληψη του K-means \lstinline[language=MATLAB]!number_of_iretaion! φορές που στην περίπτωση μας επιλέχτηκε 10. 
    \item  \textbf{6ο Σετ Παραμέτρων}:
\lstinline[language=MATLAB]!kmeans(X,clnumber,'Distance',correlation,'Start',heuristic_centroid);!
Όπου \lstinline[language=MATLAB]!X! είναι το σύνολο δεδομένων εισόδου, \lstinline[language=MATLAB]!clnumber! είναι ο αριθμός των ομάδων που δημιουργούνται και η παράμετρος \lstinline[language=MATLAB]!Distance! που δηλώνει τον τρόπο υπολογισμού της απόστασης ορίζεται να ισούται με \lstinline[language=MATLAB]!correlation!. Η παράμετρος \lstinline[language=MATLAB]!Start! δηλώνει την αρχική επιλογή των κέντρων και ισούται με \lstinline[language=MATLAB]!heuristic_centroid! που προκύπτουν από την κλήση της συνάρτησης
\end{enumerate}

Αφού ορίσαμε τα σετ παραμέτρων μας, τρέξαμε στο Matlab το script με όνομα \lstinline[language=MATLAB]!scriptk!. Tο παραπάνω script καλεί την συνάρτηση \lstinline[language=MATLAB]!optimizer_kmeans!, η οποία διαβάζει τα dataset μας, υλοποιεί τον αλγόριθμο K-means και μας βγάζει τα διαγράμματα στα οποία έχουμε τις μετρικές μας. Η φόρτωση των dataset μας γίνεται με την συνάρτηση \lstinline[language=MATLAB]!file_paths! και τα διαγράμματα με την συνάρτηση \lstinline[language=MATLAB]!plot_bars!.

Το πρώτο διάγραμμα που πήραμε περιέχει σε ένα κοινό διάγραμμα το 1o, 2ο και 3ο σετ δεδομένων κάνοντας την ομαδοποίηση για 8 clusters.

\begin{figure}
\centering
\includegraphics[width=0.7\linewidth]{../../../../../Dropbox/protypa-figs/pictures-2/kmeansCosBar8}
\caption{}
\label{fig:kmeansCosBar8}
\end{figure}

Στο παραπάνω σχήμα βλέπουμε στον κάθετο άξονα τα διάφορετικα dataset που χρησιμοποιήσαμε για το 1ο, 2ο και 3ο σετ παραμέτρων αντίστοιχα. Στον οριζόντιο άξονα έχουμε τις μετρικές $Silhouette$, $Cohesion$, $Separation$ και $Success Rate$. Καταρχήν να σημειώσουμε ότι οι μετρικές $Cohesion$ και $Separation$ επειδή παίρνουν πολύ μεγάλες τιμές και θέλαμε να τις βάλουμε στο ίδιο διάγραμμα με τις άλλες μετρικές τις κανονικοποιήσαμε ως προς τον ίδιο αριθμό. Αυτός ο συντελεστής για το $Cohesion$ ισούται με 2000 και για το $Separation$ ισούται με 10000. 
Οι 2 μετρικές στις οποίες δίνουμε μεγαλύτερη βαρύτητα είναι το $Silhouette$ και το $Success Rate$. Το $Silhouette$ είναι μια μετρική που περιγράψαμε παραπάνω. Το $Success Rate$ είναι η μετρική που μας δίνει τι ποσοστό από τις υπάρχουσες βιβλιοθήκες βρήκαμε σωστό. Όπως είναι λογικό μεγάλο $Silhouette$ αντιστοιχεί συνήθως σε μεγάλο $Success Rate$ χωρίς αυτό να σημαίνει ότι το μέγιστο $Silhouette$ αντιστοιχεί στο μέγιστο $Success Rate$. Σαν βασική μας μετρική θα θεωρήσουμε το $Silhouette$ καθώς θεωρούμε ότι είναι πιο σωστό αφού σε κάθε πρόβλημα ομαδοποίησης δεν έχουμε τα τελικά αποτελέσματα για να μπορούμε να κάνουμε την σύγκριση. Το $Success Rate$ δηλαδή δεν μπορεί να θεωρηθεί αντιπροσωπευτική μετρική αλλά μπορεί να χρησιμοποιηθεί για επαλήθευση. 
Στο παραπάνω διάγραμμα, είπαμε ότι στον κατακόρυφο άξονα έχουμε τα διάφορα dataset μας.Οι 3 πρώτες ομάδες τιμών αντιστοιχούν στο ίδιο dataset υλοποιημένο με διαφορετικό σετ παραμέτρων. Έτσι η πρώτη ομάδα τιμών (από πάνω προς τα κάτω) μας δίνει το πρώτο dataset με επιλογή κέντρου με $heuristic$ τρόπο, η δεύτερη ομάδα τιμών το πρώτο dataset με τυχαία επιλογή κέντρου με πολλαπλές επαναλήψεις και η τρίτη ομάδα τιμών το πρώτο dataset με τυχαία επιλογή κέντρου. Η τέταρτη ομάδα τιμών μας δίνει το δεύτερο dataset με επιλογή κέντρου με $heuristic$ τρόπο, η πέμπτη ομάδα το δεύτερο dataset  με τυχαία επιλογή κέντρου με πολλαπλές επαναλήψεις και η έκτη ομάδα τιμών το δεύτερο dataset με τυχαία επιλογή κέντρου. Με παρόμοιο τρόπο συνεχίζουμε στις επόμενες ομάδες τιμών. Παρατηρούμε δηλαδή ότι ένα dataset μας κατέχει 3 συνεχόμενες θέσεις στον κατακόρυφο άξονα.
Κάθε ομάδα τιμών περιέχει τις 4 μετρικές που περιγράψαμε με διαφορετικό χρώμα.

Παρατηρούμε ότι από το διάγραμμα έχουμε το μέγιστο $Silhouette$ για την 5η εγγραφή στον κατακόρυφο άξονα. Αυτή η εγγραφή αφορά το dataset με όνομα \url{join_duplicates/freq_8_70/gibberish_detector/join_similar/drop_fry_words/dataset.csv} με επιλογή κέντρου τυχαία με πολλαπλές επαναλήψεις. Ωστόσο τρέξαμε στο Matlab μια σύγκριση με το επόμενο διάγραμμα και παρατηρήσαμε ότι το επόμενο διάγραμμα δίνει υψηλότερο μέγιστο $Silhouette$. Αυτό που είναι χρήσιμο από αυτό το διάγραμμα είναι να πάρουμε το βέλτιστο $Success Rate$ το οποίο βρίσκεται στο dataset \url{join_duplicates/freq_8_70/dataset.csv} με τυχαία επιλογή κέντρου με πολλαπλές επαναλήψεις. Η τιμή του $Success Rate$ σε αυτήν την περίπτωση είναι ίση με 92.5\% το οποίο μεταφράζεται ότι πετύχαμε σωστή ομαδοποίηση για 74 βιβλιοθήκες από το σύνολο των 80. Αυτό είναι το καλύτερο ποσοστό σωστών αποτελεσμάτων που επιτύχαμε. Γενικά τα βέλτιστα αποτελέσματα στον K-means επιτυγχάνονται με τυχαία επιλογή κέντρου με πολλαπλές επαναλήψεις και για αυτόν τον λόγο δεν θα αναφέρεται.

Στην συνέχεια πήραμε σε ένα κοινό διάγραμμα το 4o, 5ο και 6ο σετ δεδομένων κάνοντας την ομαδοποίηση για 8 clusters.

\begin{figure}
\centering
\includegraphics[width=0.7\linewidth]{../../../../../Dropbox/protypa-figs/pictures-2/kmeansCorBar8}
\caption{}
\label{fig:kmeansCorBar8}
\end{figure}

Από το παραπάνω διάγραμμα προκύπτει ότι το dataset με το μέγιστο $Silhouette$ είναι το \url{join_duplicates/freq_3_70/gibberish_detector/join_similar/dataset.csv}. Η τιμή του $Silhouette$ είναι 0.259 και το αντίστοιχο $SuccessRate$ είναι 88.7\% δηλαδή 70 βιβλιοθήκες ομαδοποιήθηκαν σωστά. Αυτή η τιμή του $Silhouette$ ξεπερνάει και όλες τις προηγούμενες από το διάγραμμα όπου η απόσταση μετρήθηκε ως $Cosine$. Παρατηρούμε λοιπόν ότι ενώ μεγάλες τιμές του συντελεστή $Silhouette$ αντιστοιχούν σε μεγάλες τιμές $SuccessRate$ και το αντίστροφο, οι μέγιστες τιμές της μιας μετρικής δεν σημαίνουν και μέγιστες τιμές της άλλης μετρικής.

Η μετρική $Success Rate$ στις 2 παραπάνω ομαδοποιήσεις μετρήθηκε με τον 1ο τρόπο.
%ΤΟ DO HYPERLINK TO DESCRIBE OF eval_clust(type1)
Αφού υλοποιήσαμε την παραπάνω ομαδοποίηση περάσαμε τα αποτελέσματα που πήραμε στο Weka για οπτικοποίηση των καλύτερων αποτελεσμάτων μας.
\noindent\begin{minipage}{\linewidth}
    \centering
    \captionsetup{type={figure}}
    \includegraphics[width=\linewidth]{images/kmeans8_result1.eps}
    \includegraphics[width=\linewidth]{images/kmeans8_result2.eps}
    \captionof{figure}{TODO}
    \label{fig:kmeans8_result}
\end{minipage}

Παρατηρούμε ότι στην πρώτη περίπτωση όπου είναι η ομαδοποίηση με βέλτιστο $Success Rate$ έχούμε πετύχει όλες τις βιβλιοθήκες \url{http-clients} (γκρι), τις \url{swing-libraries} λαχανί και τις \url{xml-processing}. Από την βιβλιοθήκη \url{android} (μπλε) πετύχαμε 22 βιβλιοθήκες και από τις υπόλοιπες ομαδοποιήσαμε επιτυχώς το μεγαλύτερο ποσοστό. Διαπυστώνουμε ότι όντως λάθος ομαδοποίηση υπέστησαν 6 βιβλιοθήκες. Άρα ορθώς το $Success Rate$ είναι 92.5\%.

Όμοια στο δεύτερο διάγραμμα παρατηρούμε ότι επιτύχαμε το μεγαλύτερο ποσοστό \url{android} (μπλε) βιβλιοθηκών. Παρατηρούμε ότι 7 βιβλιοθήκες ομαδοποιήθηκαν λάθος. 
Έπειτα υλοποιήσαμε μια μέτρηση με το 1ο, 2ο και 3ο σετ παραμέτρων και υλοποιήσαμε 10 ομάδες. Πήραμε το παρακάτω διάγραμμα:

\begin{figure}
\centering
\includegraphics[width=0.7\linewidth]{../../../../../Dropbox/protypa-figs/pictures-2/kmeansCosBar10}
\caption{}
\label{fig:kmeansCosBar10}
\end{figure}

Από το παραπάνω διάγραμμα προκύπτει ότι το dataset \url{join_duplicates/freq_3_70/gibberish_detector/join_similar/dataset.csv} έχει το βέλτιστο $Silhouette$ και τιμή ίση με 0.263. Το αντίστοιχο ποσοστό $SuccessRate$ είναι 87.5\% το οποίο σημαίνει ότι αν συνενώσουμε τις ομάδες μας και δημιουργήσουμε 8 clusters, οι 70 βιβλιοθήκες μας θα έχουν ομαδοποιηθεί σωστά.

Το αντίστοιχο διάγραμμα για το 4ο, 5ο και 6ο σετ παραμέτρων με 10 κλάσεις είναι το παρακάτω:
\begin{figure}
\centering
\includegraphics[width=0.7\linewidth]{../../../../../Dropbox/protypa-figs/pictures-2/kmeansCorBar10}
\caption{}
\label{fig:kmeansCorBar10}
\end{figure}

Από αυτό το διάγραμμα παίρνουμε το καλύτερο $SuccessRate$ που ισούται με 93.8\% στο dataset \url{join_duplicates/freq_8_50/gibberish_detector/join_similar/drop_fry_words/dataset.csv}.

Όμοια με πριν χρησιμοποιήσαμε το Weka για οπτικοποίηση των αποτελεσμάτων μας. Στην πρώτη περίπτωση έχουμε το μεγαλύτερο $Success Rate$ και στην δεύτερη το μεγαλύτερο $Silhouete$.

\noindent\begin{minipage}{\linewidth}
    \centering
    \captionsetup{type={figure}}
    \includegraphics[width=\linewidth]{images/kmeans10_result1.eps}
    \includegraphics[width=\linewidth]{images/kmeans10_result2.eps}
    \captionof{figure}{TODO}
    \label{fig:kmeans10_result}
\end{minipage}

Παρατηρούμε ότι ο αριθμός των κλάσεων είναι 10. Αυτό που αξίζει να παρατηρήσουμε είναι ότι αν προχωρούσαμε στο επόμενο στάδιο της μετ-επεξεργασίας, στην πρώτη περίπτωση, η 1η κλάση και η 8η κλάση θα ενώνονταν καθώς είναι και οι 2 καθαρά android βιβλιοθήκες. Όμοια θα μπορούσαμε να κάνουμε και άλλες συνενώσεις.

Τέλος υλοποιήσαμε ομαδοποίηση για 20 κλάσεις. Αρχικά πήραμε το 1ο, 2ο και 3ο σετ παραμέτρων:\\
\begin{figure}
\centering
\includegraphics[width=0.7\linewidth]{../../../../../Dropbox/protypa-figs/pictures-2/kmeansCosBar20}
\caption{}
\label{fig:kmeansCosBar20}
\end{figure}

Από το παραπάνω διάγραμμα πήραμε το καλύτερο $Success Rate$ να ισούται με 93.8\% στο dataset \url{join_duplicates/freq_8_50/gibberish_detector/join_similar/drop_fry_words/bool_it/dataset.csv}.

Τα αντίστοιχα αποτελέσματα στο 4ο, 5ο και 6ο dataset για 20 clusters:

\begin{figure}
\centering
\includegraphics[width=0.7\linewidth]{../../../../../Dropbox/protypa-figs/pictures-2/kmeansCorBar20}
\caption{}
\label{fig:kmeansCorBar20}
\end{figure}

Το παραπάνω διάγραμμα μας δίνει καλύτερη μετρική $Silhouette$ ίση με 0.278.

Η μετρική $Success Rate$ στις 4 παραπάνω ομαδοποιήσεις μετρήθηκε με τον 2ο τρόπο.  
%ΤΟ DO HYPERLINK TO DESCRIBE OF eval_clust(type2)
Οι 4 τελευταίες ομαδοποιήσεις, δηλαδή ο διαχώρισμος των ομάδων σε 10 και 20 αντίστοιχα έγιναν για να αντιμετωπίσουν το πρόβλημα που έχει ο k-means σε ανισομεγεθής ομάδες όπως αναφέρθηκε και στην παράγραφο της εισαγωγής για τον K-means. Ουσιαστικά δεν έγινε έπειτα η μείωση των ομάδων σε 8 όπως θέλουμε να είναι ο τελικός αριθμός. Περισσότερες πληροφορίες σχετικά με το θέμα της μετ-επεξεργασίας δίνονται στο κεφάλαιο με τα Ανοιχτά Θέματα.
%TO DO HYPERLINK TO OPENTHEMES
Παρακάτω βλέπουμε τα αποτελέσματα της βέλτιστης ομαδοποίησης ως προς $Success Rate$ kai $Silhouette$. Παρατηρούμε ότι πρόκειται όντως για 20 ομάδες. Πολλές ομάδες θα μπορούσαν να συνενωθούν σε εννιαίες όπως οι \url{android} (ομάδες 2,4,5,10,11), οι \url{http-clients} (ομάδες 17,18) και oi \url{command-line-parsers} (ομάδες 9,15). Οι υπόλοιπες συνενώσεις θα γίνουν για να διατηρηθεί ελάχιστο το $SSE$.
\\\noindent\begin{minipage}{\linewidth}
    \centering
    \captionsetup{type={figure}}
    \includegraphics[width=\linewidth]{images/kmeans20_result1.eps}
    \includegraphics[width=\linewidth]{images/kmeans20_result2.eps}
    \captionof{figure}{TODO}
    \label{fig:kmeans20_result}
\end{minipage}
% Please add the following required packages to your document preamble:
% \usepackage{graphicx}
Στο παρακάτω πινακάκι παρουσιάζονται μαζεμένες όλες οι βέλτιστες λύσεις για το $Silhouette$ και για το $Success Rate$ για τις ομαδοποιήσεις που έγιναν.

\begin{minipage}{\linewidth}
    \centering
    \resizebox{\textwidth}{!}{%
        \begin{tabular}{lllllllll}
            \cline{7-7}
            Dataset & Center Type & Distance Type & Number of Clusters & Cohesion & \multicolumn{1}{l|}{Separation} & \multicolumn{1}{l|}{Silhouette} & Success Rate & Max Type \\ \cline{7-7}
            Α & random multiple iterations & Cosine & 8 & 592 & 5080 & 0.231 & 0.925 & Best Success Rate \\
            B & random multiple iterations & Correlation & 8 & 660 & 5090 & 0.259 & 0.887 & Best Silhouette \\
            C & random multiple iterations & Correlation & 10 & 462 & 5300 & 0.222 & 0.938 & Best Success Rate \\
            B & random multiple iterations & Cosine & 10 & 544 & 2630 & 0.263 & 0.875 & Best Silhouette \\
            D & random multiple iterations & Cosine & 20 & 168 & 5500 & 0.213 & 0.938 & Best Success Rate \\
            B & random multiple iterations & Correlation & 20 & 228 & 5520 & 0.278 & 0.9 & Best Silhouette
        \end{tabular}
    }
    \captionof{table}{TODO}
    \label{K-means Best Results}
\end{minipage}

Στο παραπάνω πινακάκι χρησιμοποιήσαμε τους κωδικούς των dataset τα οποία ορίσαμε ως εξής:
\begin{enumerate}
    \item A=\url{join_duplicates/freq_8_70/dataset.csv}
    \item B=\url{join_duplicates/freq_3_70/gibberish_detector/join_similar/dataset.csv}
    \item C=\url{join_duplicates/freq_8_50/gibberish_detector/join_similar/drop_fry_words/dataset.csv}
    \item D=\url{join_duplicates/freq_8_50/gibberish_detector/join_similar/drop_fry_words/bool_it/dataset.csv}
\end{enumerate}

