\section{Περιγραφή εξερεύνησης των ιεραρχικών μοντέλων}

Τα μοντέλα μας δημιουργήθηκαν με την χρήση του Matlab και η διαδικασία που περιγράφτηκε στην εισαγωγή πραγματοποιείται στο αρχείο optimizer\_hier.m
για τις διάφορες παραμέτρους. Οι αποστάσεις ανάμεσα στα σημεία ή υπολογίζονται για τους ιεραρχικούς με την χρήση της συνάρτησης pdist , η οποία μπορεί να υπολογίσει τις εξής αποστάσεις:
\begin{itemize}
	\item euclidean
	\item seuclidean
	\item Mankowski
	\item chebychev
	\item mahalanobis
	\item cosine
	\item correlation
	\item spearman
	\item jaccard 
\end{itemize}


Περισσότερες πληροφορίες για την pdist \href{http://www.mathworks.com/help/stats/pdist.html}{εδώ}
Σαν αποστάσεις ικανοποιητικά αποτελέσματα έδιναν μόνο το correlation και το 
cosine  μία πιθανή εξήγηση βρίσκεται \textbf{(Εδώ τα reference})

Έπειτα χρησιμοποιήθηκε η συνάτηση linkage οποία δημιουργεί ουσιαστικά το δέντρο ιεραρχίας .Οι μετρική που χρησιθμοποιεί για να συνδέσει τα κοντινοερα cluster μπορεί να είναι μία από της εξής:
\begin{itemize}
  	\item single
  	\item complete
  	\item average
  	\item weighted
  	\item centroid
  	\item median
  	\item ward 
  \end{itemize}

Από άποψη χρόνου και αποτελεσμάτων ύστερα από δοκιμές για τα τελικά πειράματα επιλέχτηκαν οι weighted ,ward, complete,average.

Έπειτα με την χρήση της συνάρτησης cluster δημιουργήσαμε τα τελικά cluster που θέλαμε τα παραπάνω βήματα υλοποιούνται από τις παρακάτω γραμμές κώδικα 


\begin{lstlisting}[language=Matlab]
%simple example of hierarchical clustering
Y = pdist (X,distance); %X is an array containig the data
YY = squareform(Y); %convert Y in a square form
Z =linkage(YY,); %find  hierarchical cluster tree,
CDX = cluster(Z,'maxclust',8);
\end{lstlisting}


επίσης υπάρχει η συνάρτηση cophonet(Z,YY) η οποία είναι μια μετρική που υπολογίζει την συσχέτιση ανάμεσα στις συνδέσεις που έχει δημιουργήσει το Z και τις αντίστοιχες αποστάσεις στο YY .Όσο πιο μεγάλη είναι η συσχέτιση τόσο πιο καλά έχει αποτυπωθεί η διαφορετικότητα των αρχικών σημείων, σαν συνδέσεις μεταξύ clusters , περισσότερα
\href{https://en.wikipedia.org/wiki/Cophenetic\_correlation{εδώ}}

Στη συνέχεια παρουσιάζονται και σχολιάζονται τα πειραματικά αποτελέσματα.

\section{Πειραματικά αποτελέσματα}

Χρησιμοποιούνται οι γνωστές μετρικές silhouette ,cohesion,separation
αλλά και η success1 η οποία είναι μία από τις 2 μετρικές που  υπολογίζει το πόσα πετύχαμε από τα πραγματικά δεδομένα.Περισσότερα για το πως υπολογίστηκαν και τη αντιπροσωπεύει η success1 στο \textbf{TELOS} του κεφαλαίου.   	

Έγινε χρήση των προαναφερθένων  αλγορίθμων απόστασης και εφαρμόστηκαν στον καθένα 4 διαφορετικοί τρόποι σύνδεσης για όλα τα dataset


\begin{figure}
	\includegraphics{images/hierCosBar.pdf}
	\caption{Μετρικές για τον ιεραρχικό αλγόριθμο με την χρήση του Cosine}
	\label{fig:CosineHier}
\end{figure}



\begin{figure}
	\includegraphics{images/hierCorBar.pdf}
	\caption{Μετρικές για τον ιεραρχικό αλγόριθμο με την χρήση του Cosine}
	\label{fig:CorrelationHier}
\end{figure}




