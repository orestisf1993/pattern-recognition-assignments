\chapter{Εισαγωγή}
\section{Περιγραφή του προβλήματος}

Με την πάροδο των χρόνων η τεχνολογία εξελίσσεται συνεχώς και δημιουργούνται καινούργιες εφαρμογές οι οποίες απαιτούν υψηλό επίπεδο γνώσεων προγραμματισμού από την πλευρά του κατασκευαστή. Βασικό εργαλείο των προγραμμτιστών σε αυτήν την προσπάθεια αποτελεί η χρήση έτοιμων βιβλιοθηκών που δίνουν την δυνατότητα στον προγραμματιστή να κερδίσει πολύτιμο χρόνο καθώς αυτές περιέχουν έτοιμες συναρτήσεις,χρήσιμες για τον κάθε προγραμματιστή. Συνεπώς απαιτείται καλή γνώση των βιβλιοθηκών που υπάρχουν και γενικά των κατηγοριών.

Σε αυτήν την εργασία θα προσπαθήσουμε να ομαδοποιήσουμε τις βιβλιοθήκες με σκοπό να κατανοήσουμε τα χαρακτηριστικά που εξάγονται από τον κώδικα γνωστών βιβλιοθηκών προγραμματισμού Java. Στην ομαδοποίηση που θα κάνουμε οι λέξεις που έχουμε αφορούν τον κώδικα των βιβλιοθηκών και οι ομάδες που θα προκύψουν είναι οι κατηγορίες των βιβλιοθηκών. Στο σετ δεδομένων που μας δίνεται για κάθε λέξη που έχουμε (feature) η κάθε βιβλιοθήκη (sample) περιέχει μια τιμή που αντιστοιχεί στην συχνότητα εμφάνισης της λέξης στον κώδικα της βιβλιοθήκης. Στόχος της εργασίας μας είναι να ομαδοποιήσουμε τις βιβλιοθήκες σε clusters ανάλογα με τον σκοπό που εξυπηρετεί καθεμία από αυτές (π.χ. android, command-linel-parsers κτλ. ). Πιο συγκεκριμένα το σύνολο δεδομένων αποτελείται από 80 γνωστές βιβλιοθήκες της Java οι οποίες χωρίζονται σε 8 κατηγορίες :

\begin{enumerate}
	\item  android
	\item  command-line-parsers
	\item  csv-libraries
	\item  http-clients
	\item  json-libraries
	\item  swing-libraries
	\item  testing-frameworks
	\item  xml-processing
\end{enumerate} 

Η συγκεκριμένη ομαδοποίηση έγινε από τους προγραμματιστές βιβλιοθηκών και είναι διαθέσιμη στην αποθήκη βιβλιοθηκών του Maven.

Ακόμα είναι χρήσιμο να τονιστεί ότι ο διαχωρισμός σε κατηγορίες δίνεται στο dataset μας στην στήλη category. Αυτή η πληροφορία θα χρησιμοποιηθεί για την αξιολόγηση της ομαδοποίησης και όχι για την υλοποίηση της ομαδοποίησης. Εκτός από την πληροφορία που μας δίνει η στήλη category για την αξιολόγηση της ομαδοποίησης χρησιμοποιήσαμε 2 αριθμητικές μετρικές για να αποτιμήσουμε την ποιότητα της ομαδοποίησης. Χρησιμοποιήσαμε τους εσωτερικούς δείκτες για να μετρήσουμε την ποιότητα μιας ομαδοποίησης χωρίς αναφορά σε εξωτερικές πληροφορίες και την τους σχετικούς δείκτες για να συγκρίνουμε τις διαφορετικές ομαδοποιήσεις που κάναμε. Οι εσωτερικοί δείκτες που χρησιμοποιήσαμε είναι οι εξής:

\begin{enumerate}
	\item Sum of the Squared Error (SSE) : Υπολογίζει το άθροισμα των τετραγώνων του σφάλματος. Χρησιμοποιείται κυρίως για την μέτρηση της ευκλείδιας απόστασης μεταξύ των σημείων ενός cluster και του κέντρου της ομάδαος (cluster centroid). Σκοπός μας είναι να ελαχιστοποιήσουμε το SSE. Γενικά το SSE είναι μια καλή μετρική για την σύγκριση 2 ομαδοποιήσεων ή 2 ομάδων και μπορεί να χρησιμοποιηθεί για τον υπολογισμό του αριθμού των ομάδων. Ο τύπος για τον υπολογισμό του SSE δίνεται παρακάτω.
	\begin{equation}
	SSE=\sum_{i=1}^{N}{(x_i-\bar{x_i})^2}
	\end{equation}
	
	Όπου $x_i$ είναι τα σημεία των ομάδων μας και  $\bar{x_i}$ είναι τα κέντρα των ομάδων μας.
	\begin{figure}
\centering
\includegraphics[width=0.7\linewidth]{../../../../../Dropbox/protypa-figs/pictures-2/SSe}
\caption{}
\label{fig:SSe}
\end{figure}

	\item Cohesion (WSS) : Είναι το άθροισμα των βαρών όλων των συνδέσμων μέσα σε μία ομάδα. Ουσιαστικά είναι μια μετρική που μας δίνει το πόσο κάλα αποτελέσματα έχει επιφέρει η ομαδοποίηση εντός ομάδος καθώς μας δείχνει πόσο στενά συνδεδεμένα είναι τα αντικείμενα κάθε cluster. To Cohesion μετράται ώς το SSE μέσα σε κάθε cluster και ο τύπος για τον υπολογισμό του είναι:
	\begin{equation}
		WSS=\sum_{i}\sum_{x \in C_i}{(x_i-m_i)^2}
	\end{equation}
	\item Separation (BSS) : Είναι το άθροισμα 	των βαρών όλως των κόμβων μιας ομάδας με όλους τους κόμβους εκτός ομάδας. Στην πραγματικότητα είναι μια μετρική που μας δείχνει πόσο διακριτές ή καλά διαχωρισμένες είναι οι ομάδες μεταξύ τους. Το Seperation μετράται ως το άθροισμα των τετραγώνων ανάμεσα στις ομάδες και ο τύπος υπολογισμού του είναι ο εξής:
	\begin{equation}
	ΒSS=\sum_{i} \abs {C_i} {(m-m_i)^2}
	\end{equation}

Όπου  $\abs{C_i}$ είναι το μέγεθος της ομάδας i.

\begin{figure}
\centering
\includegraphics[width=0.7\linewidth]{../../../../../Dropbox/protypa-figs/pictures-2/cohesion-seperation}
\caption{}
\label{fig:cohesion-seperation}
\end{figure}
	
	Στο παραπάνω σχήμα βλέπουμε την διαφορά μεταξύ του Cohesion και του Seperation. Παρατηρούμε ότι όντως το Cohesion είναι το άθροισμα των βαρών των κόμβων εντός ενός Cluster ενώ το Seperation αφορά των άθροισμα των βαρών κόμβων που ανήκουν σε διαφορετικά cluster.

	\item Silhouete : Αυτή η μετρική συνδυάζει την λογική των 2 μετρικών που αναφέρθηκαν παραπάνω,δηλαδή του Cohesion και του Seperation αλλά για μεμονωμένα σημεία, καθώς και για ομάδες και ομαδοποιήσεις. Η λογική αυτής της μετρικής είναι η εξής:
	Για κάθε σημείο i:
	
	\begin{itemize}
		\item Υπολογίζουμε το internal που είναι η μέση απόσταση του i σημείου από όλα τα υπόλοιπα σημεία στην ομάδα.
		\item Υπολογίζουμε το external που είναι το ελάχιστο της μέσης απόστασης του i σημείου από τα σημεία στις άλλες ομάδες.
		\item Ο συντελεστής Silhouete για το σημείο δίνεται από τον τύπο
		 
		$\begin{displaystyle}
		s = \left.
		\begin{cases}
		1-  \frac{internal}{external}  , & \text{if }   \enspace a < b \\
		\frac{external}{internal}-1 ,&\text{if} \enspace b \leq a 
		
		\end{cases}
		\right\} 
		\end{displaystyle}$
		\item Τυπικά η τιμή αυτού του συντελεστή κυμαίνεται από 0 εώς 1.
		\item Όσο πιο κοντά στην τιμή 1 βρίσκεται τόσο καλύτερη είναι η ομαδοποίηση μας
	\end{itemize}
 
\end{enumerate}

Περισσότερες πληροφορίες τόσο για το dataset όσο και για την χρήση της κάθε μετρικής θα δοθούν παρακάτω στα κομμάτια της προεπεξεργασίας δεδομένων και ομαδοποίησης με τους αλγορίθμους αντίστοιχα.

\section{Τυπογραφικές Παραδοχές}

\section{Συνοπτική Περιγραφή της Διαδικασίας που ακολουθήθηκε}
Για την υλοποίηση της ομαδοποίησης ακολουθήσαμε τα παρακάτω βήματα:
\begin{enumerate}
	\item Αρχικά επεξεργαστήκαμε το dataset μας έτσι ώστε να έχουμε ένα πιο αντιπροσωπευτικό τμήμα και μικρότερο σε αριθμό δεδομένων. 
	\item Έπειτα επιλάξαμε τους αλγορίθμους τους οποίους χρησιμοποιήσαμε για την ομαδοποίηση εφαρμόζοντας διαφορετικά σετ παραμέτρων για κάθε αλγόριθμο.
	\item Αφού κάναμε την ομαδοποίηση προχωρήσαμε στην αξιολόγηση των αλγοριθμών βάσει των μετρικών που αναλύθηκαν παραπάνω.
	\item Τέλος κάναμε μια σύγκριση των ομαδοποιήσεων όπως αυτές προέκυψαν από την εφαρμογή των διαφόρων αλγορίθμων και βγάλαμε τα συμπεράσματα μας.
\end{enumerate}

\section{Αλγόριθμοι που χρησιμοποιήθηκαν}
Οι αλγόριθμοι ομαδοποίησης που χρησιμοποιήσαμε ήταν :

\begin{enumerate}
	\item \textbf{Διαχωριστικοί:}
	\begin{itemize}
		\item K-means
		\item K-medoids
	\end{itemize}  
	\item \textbf{Ιεραρχικοί}

	\item \textbf{Γενετικοί}  

\end{enumerate}

\section{Εργαλεία που χρησιμοποιήθηκαν}

\begin{enumerate}
	\item \textbf{Weka}
	Για το visualization και την εξερεύνηση των δεδομένων,
	και τις μετατροπές μεταξύ csv και arff.
	\item \textbf{Python}
	Για την αυτοματοποίηση μερικών διαδικασιών
	και την βελτιστοποίηση διάφορων παραμέτρων των μοντέλων μας. Η Python ήταν το βασικό μας εργαλείο για την προ-επεξεργασία των δεδομένων μας και την δημιουργία των τελικών dataset μας.
	\item \textbf{Matlab{}}
	Χρησιμοποιήσαμε το Matlab για την εκτέλεση των αλγορίθμων ομαδοποίησης μας, τον υπολογισμό των μετρικών και για να πάρουμε τα διαγράμματα που παρουσιάζονται στο παρών έγγραφο.
	\item \textbf{\LaTeX{}}
	Για την παραγωγή αυτού του αρχείου.
\end{enumerate}