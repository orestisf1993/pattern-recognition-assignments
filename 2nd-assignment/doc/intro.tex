\chapter{Εισαγωγή}
\section{Περιγραφή του προβλήματος}

Με την πάροδο των χρόνων η τεχνολογία εξελίσσεται συνεχώς και δημιουργούνται καινούργιες εφαρμογές οι οποίες απαιτούν υψηλό επίπεδο γνώσεων προγραμματισμού από την πλευρά του κατασκευαστή. Βασικό εργαλείο των προγραμμτιστών σε αυτήν την προσπάθεια αποτελεί η χρήση έτοιμων βιβλιοθηκών που δίνουν την δυνατότητα στον προγραμματιστή να κερδίσει πολύτιμο χρόνο καθώς αυτές περιέχουν έτοιμες συναρτήσεις,χρήσιμες για τον κάθε προγραμματιστή. Συνεπώς απαιτείται καλή γνώση των βιβλιοθηκών που υπάρχουν και γενικά των κατηγοριών.

Σε αυτήν την εργασία θα προσπαθήσουμε να ομαδοποιήσουμε τις βιβλιοθήκες με σκοπό να κατανοήσουμε τα χαρακτηριστικά που εξάγονται από τον κώδικα γνωστών βιβλιοθηκών προγραμματισμού Java. Στην ομαδοποίηση που θα κάνουμε οι λέξεις που έχουμε αφορούν τον κώδικα των βιβλιοθηκών και οι ομάδες που θα προκύψουν είναι οι κατηγορίες των βιβλιοθηκών. Στο σετ δεδομένων που μας δίνεται για κάθε λέξη που έχουμε (feature) η κάθε βιβλιοθήκη (sample) περιέχει μια τιμή που αντιστοιχεί στην συχνότητα εμφάνισης της λέξης στον κώδικα της βιβλιοθήκης. Στόχος της εργασίας μας είναι να ομαδοποιήσουμε τις βιβλιοθήκες σε clusters ανάλογα με τον σκοπό που εξυπηρετεί καθεμία από αυτές (π.χ. android, command-linel-parsers κτλ. ). Πιο συγκεκριμένα το σύνολο δεδομένων αποτελείται από 80 γνωστές βιβλιοθήκες της Java οι οποίες χωρίζονται σε 8 κατηγορίες :

\begin{enumerate}
	\item  android
	\item  command-line-parsers
	\item  csv-libraries
	\item  http-clients
	\item  json-libraries
	\item  swing-libraries
	\item  testing-frameworks
	\item  xml-processing
\end{enumerate} 

Η συγκεκριμένη ομαδοποίηση έγινε από τους προγραμματιστές βιβλιοθηκών και είναι διαθέσιμη στην αποθήκη βιβλιοθηκών του Maven.

Ακόμα είναι χρήσιμο να τονιστεί ότι ο διαχωρισμός σε κατηγορίες δίνεται στο dataset μας στην στήλη category. Αυτή η πληροφορία θα χρησιμοποιηθεί για την αξιολόγηση της ομαδοποίησης και όχι για την υλοποίηση της ομαδοποίησης. Εκτός από την πληροφορία που μας δίνει η στήλη category για την αξιολόγηση της ομαδοποίησης χρησιμοποιήσαμε 2 αριθμητικές μετρικές για να αποτιμήσουμε την ποιότητα της ομαδοποίησης. Χρησιμοποιήσαμε τους εσωτερικούς δείκτες για να μετρήσουμε την ποιότητα μιας ομαδοποίησης χωρίς αναφορά σε εξωτερικές πληροφορίες και την τους σχετικούς δείκτες για να συγκρίνουμε τις διαφορετικές ομαδοποιήσεις που κάναμε. Οι εσωτερικοί δείκτες που χρησιμοποιήσαμε είναι οι εξής:

\begin{enumerate}
	\item Sum of the Squared Error (SSE) : Υπολογίζει το άθροισμα των τετραγώνων του σφάλματος. Χρησιμοποιείται κυρίως για την μέτρηση της ευκλείδιας απόστασης μεταξύ των σημείων ενός cluster και του κέντρου της ομάδαος (cluster centroid). Σκοπός μας είναι να ελαχιστοποιήσουμε το SSE. Γενικά το SSE είναι μια καλή μετρική για την σύγκριση 2 ομαδοποιήσεων ή 2 ομάδων και μπορεί να χρησιμοποιηθεί για τον υπολογισμό του αριθμού των ομάδων. Ο τύπος για τον υπολογισμό του SSE δίνεται παρακάτω.
	\begin{equation}
	SSE=\sum_{i=1}^{N}{(x_i-\bar{x_i})^2}
	\end{equation}
	
	Όπου \bar{x_i}
\end{enumerate}

