\chapter{Προεπεξεργασία Δεδομένων}
\section{Μορφή των δεδομένων}
Τα δεδομένα μας προέρχονται από τους όρους-λέξεις 80 βιβλιοθηκών της γλώσσας προγραμματισμού Java που ανήκουν σε 8 κατηγορίες:
\begin{enumerate}\label{itemize:categories}
    \item android
    \item command-line-parsers
    \item csv-libraries
    \item http-clients
    \item json-libraries
    \item swing-libraries
    \item testing-frameworks
    \item xml-processing
\end{enumerate}

Το αρχικό μας dataset περιλαμβάνει $109706$ λέξεις οι οποίες παρουσιάζονται σαν στήλες σε ένα αρχείο \texttt{.csv}.
Κάτω από την καθεμία αναγράφονται οι απόλυτες συχνότητες εμφάνισης του κάθε όρου.
Επίσης, δίνονται οι στήλες \texttt{project}, κάτω από την οποία είναι τα ονόματα της κάθε βιβλιοθήκης, και \texttt{category}, κάτω από την οποία είναι η
\hyperref[itemize:categories]{κατηγορία} που ανήκει η κάθε βιβλιοθήκη.

\begin{center}
\captionsetup{type={table}}
\resizebox{\textwidth}{!}{%
\begin{tabular}{|c|c|c|c|c|c|c|c|c|c|}
\hline
project & category & ACONST & CompoundButtonCheckedChangeOnSubscribe & StatisticsComponent & getProjection & hashedSignature & propertyDescs & testGetBound & testReadPaths \\ \hline
com.appnexus.opensdk/appnexus-sdk & android & 0 & 0 & 0 & 0 & 0 & 0 & 0 & 0 \\
com.auth0.android/lock & android & 0 & 0 & 0 & 0 & 0 & 0 & 0 & 0 \\
com.bingzer.android.ads/adrunner & android & 0 & 0 & 0 & 0 & 0 & 0 & 0 & 0 \\
com.daimajia.easing/library & android & 0 & 0 & 0 & 0 & 0 & 0 & 0 & 0 \\
com.facebook.android/facebook-android-sdk & android & 0 & 0 & 0 & 0 & 2 & 0 & 0 & 0 \\
com.facebook.fresco/fbcore & android & 0 & 0 & 0 & 0 & 0 & 0 & 0 & 0 \\
com.facebook.fresco/fresco & android & 0 & 0 & 0 & 0 & 0 & 0 & 0 & 0 \\
com.github.asne/asne-core & android & 0 & 0 & 0 & 0 & 0 & 0 & 0 & 0 \\
com.github.castorflex.smoothprogressbar/library & android & 0 & 0 & 0 & 0 & 0 & 0 & 0 & 0 \\
com.hannesdorfmann.mosby/mvp & android & 0 & 0 & 1 & 0 & 0 & 0 & 0 & 0 \\
com.jakewharton.rxbinding/rxbinding & android & 0 & 3 & 0 & 0 & 0 & 0 & 0 & 0 \\
com.jakewharton.timber/timber & android & 0 & 0 & 0 & 0 & 0 & 0 & 0 & 0 \\
com.jdroidframework/jdroid-android & android & 0 & 0 & 0 & 3 & 0 & 0 & 0 & 0 \\
com.joanzapata.iconify/android-iconify & android & 0 & 0 & 0 & 0 & 0 & 0 & 0 & 0 \\
com.mixpanel.android/mixpanel-android & android & 0 & 0 & 0 & 0 & 0 & 3 & 0 & 1 \\
com.rengwuxian.materialedittext/library & android & 0 & 0 & 0 & 0 & 0 & 0 & 0 & 0 \\
com.shamanland/xdroid-core & android & 0 & 0 & 0 & 0 & 0 & 0 & 0 & 0 \\
com.stanfy.enroscar/enroscar-beans & android & 0 & 0 & 0 & 0 & 0 & 0 & 0 & 0 \\
com.stanfy.enroscar/enroscar-sdk-dep & android & 0 & 0 & 0 & 0 & 0 & 0 & 0 & 0 \\
com.stanfy.enroscar/enroscar-shared & android & 0 & 0 & 0 & 0 & 0 & 0 & 0 & 0 \\
de.quist.apps.maps/android-maps-abstraction & android & 0 & 0 & 0 & 6 & 0 & 0 & 0 & 0 \\
eu.inmite.android.lib/android-styled-dialogs & android & 0 & 0 & 0 & 0 & 0 & 0 & 0 & 0 \\
io.palaima.debugdrawer/debugdrawer & android & 0 & 0 & 0 & 0 & 0 & 0 & 0 & 0 \\
io.reactivex/rxandroid & android & 0 & 0 & 0 & 0 & 0 & 0 & 0 & 0 \\
args4j/args4j & command-line-parsers & 0 & 0 & 0 & 0 & 0 & 0 & 0 & 0 \\
com.beust/jcommander & command-line-parsers & 0 & 0 & 0 & 0 & 0 & 0 & 0 & 0 \\
com.github.spullara.cli-parser/cli-parser & command-line-parsers & 0 & 0 & 0 & 0 & 0 & 0 & 0 & 0 \\
com.lexicalscope.jewelcli/jewelcli & command-line-parsers & 0 & 0 & 0 & 0 & 0 & 0 & 0 & 0 \\
commons-cli/commons-cli & command-line-parsers & 0 & 0 & 0 & 0 & 0 & 0 & 0 & 0 \\
gnu.getopt/java-getopt & command-line-parsers & 0 & 0 & 0 & 0 & 0 & 0 & 0 & 0 \\
io.airlift/airline & command-line-parsers & 0 & 0 & 0 & 0 & 0 & 0 & 0 & 0 \\
net.sf.jopt-simple/jopt-simple & command-line-parsers & 0 & 0 & 0 & 0 & 0 & 0 & 0 & 0 \\
net.sourceforge.argparse4j/argparse4j & command-line-parsers & 0 & 0 & 0 & 0 & 0 & 0 & 0 & 0 \\
org.cyclopsgroup/jcli & command-line-parsers & 0 & 0 & 0 & 0 & 0 & 0 & 0 & 0 \\
org.realityforge.getopt4j/getopt4j & command-line-parsers & 0 & 0 & 0 & 0 & 0 & 0 & 0 & 0 \\
au.com.bytecode/opencsv & csv-libraries & 0 & 0 & 0 & 0 & 0 & 0 & 0 & 0 \\
com.fasterxml.jackson.dataformat/jackson-dataformat-csv & csv-libraries & 0 & 0 & 0 & 0 & 0 & 0 & 0 & 0 \\
com.opencsv/opencsv & csv-libraries & 0 & 0 & 0 & 0 & 0 & 0 & 0 & 0 \\
net.sf.flatpack/flatpack & csv-libraries & 0 & 0 & 0 & 0 & 0 & 0 & 0 & 0 \\
net.sf.supercsv/super-csv-dozer & csv-libraries & 0 & 0 & 0 & 0 & 0 & 0 & 0 & 0 \\
net.sf.supercsv/super-csv & csv-libraries & 0 & 0 & 0 & 0 & 0 & 0 & 0 & 0 \\
org.apache.commons/commons-csv & csv-libraries & 0 & 0 & 0 & 0 & 0 & 0 & 0 & 0 \\
org.beanio/beanio & csv-libraries & 0 & 0 & 0 & 0 & 0 & 0 & 0 & 0 \\
org.jdbi/jdbi & csv-libraries & 0 & 0 & 0 & 0 & 0 & 0 & 0 & 0 \\
org.onebusaway/onebusaway-csv-entities & csv-libraries & 0 & 0 & 0 & 0 & 0 & 0 & 0 & 0 \\
com.github.kevinsawicki/http-request & http-clients & 0 & 0 & 0 & 0 & 0 & 0 & 0 & 0 \\
com.google.http-client/google-http-client & http-clients & 0 & 0 & 0 & 0 & 0 & 0 & 1 & 0 \\
com.jcabi/jcabi-http & http-clients & 0 & 0 & 0 & 0 & 0 & 0 & 0 & 0 \\
com.m3/curly & http-clients & 0 & 0 & 0 & 0 & 0 & 0 & 0 & 0 \\
com.mashape.unirest/unirest-java & http-clients & 0 & 0 & 0 & 0 & 0 & 0 & 0 & 0 \\
com.metamx/http-client & http-clients & 0 & 0 & 0 & 0 & 0 & 0 & 0 & 0 \\
com.ning/async-http-client & http-clients & 0 & 0 & 0 & 0 & 0 & 0 & 0 & 0 \\
com.squareup.okhttp/okhttp & http-clients & 0 & 0 & 0 & 0 & 0 & 0 & 0 & 0 \\
org.apache.httpcomponents/httpasyncclient & http-clients & 0 & 0 & 0 & 0 & 0 & 0 & 0 & 0 \\
org.apache.httpcomponents/httpclient & http-clients & 0 & 0 & 0 & 0 & 0 & 0 & 0 & 0 \\
com.alibaba/fastjson & json-libraries & 5 & 0 & 0 & 0 & 0 & 0 & 0 & 0 \\
com.eclipsesource.minimal-json/minimal-json & json-libraries & 0 & 0 & 0 & 0 & 0 & 0 & 0 & 0 \\
com.fasterxml.jackson.core/jackson-core & json-libraries & 0 & 0 & 0 & 0 & 0 & 0 & 0 & 0 \\
com.google.code.gson/gson & json-libraries & 0 & 0 & 0 & 0 & 0 & 0 & 0 & 0 \\
com.googlecode.json-simple/json-simple & json-libraries & 0 & 0 & 0 & 0 & 0 & 0 & 0 & 0 \\
com.jayway.jsonpath/json-path & json-libraries & 0 & 0 & 0 & 0 & 0 & 0 & 0 & 0 \\
net.minidev/json-smart & json-libraries & 4 & 0 & 0 & 0 & 0 & 0 & 0 & 0 \\
org.codehaus.jettison/jettison & json-libraries & 0 & 0 & 0 & 0 & 0 & 0 & 0 & 0 \\
org.json/json & json-libraries & 0 & 0 & 0 & 0 & 0 & 0 & 0 & 0 \\
com.github.rickyclarkson/swingflow & swing-libraries & 0 & 0 & 0 & 0 & 0 & 0 & 0 & 0 \\
net.sf.cssbox/swingbox & swing-libraries & 0 & 0 & 0 & 0 & 0 & 0 & 0 & 0 \\
org.buildsomethingawesome.lib/awesome-java-swing & swing-libraries & 0 & 0 & 0 & 0 & 0 & 0 & 0 & 0 \\
org.fuin/utils4swing & swing-libraries & 0 & 0 & 0 & 0 & 0 & 0 & 0 & 0 \\
org.scijava/swing-checkbox-tree & swing-libraries & 0 & 0 & 0 & 0 & 0 & 0 & 0 & 0 \\
org.softsmithy.lib/softsmithy-lib-swing & swing-libraries & 0 & 0 & 0 & 0 & 0 & 0 & 0 & 0 \\
org.swinglabs/swingx & swing-libraries & 0 & 0 & 0 & 0 & 0 & 0 & 0 & 0 \\
com.novocode/junit-interface & testing-frameworks & 0 & 0 & 0 & 0 & 0 & 0 & 0 & 0 \\
junit/junit & testing-frameworks & 0 & 0 & 0 & 0 & 0 & 0 & 0 & 0 \\
org.testng/testng & testing-frameworks & 0 & 0 & 0 & 0 & 0 & 0 & 0 & 0 \\
xmlunit/xmlunit & testing-frameworks & 0 & 0 & 0 & 0 & 0 & 0 & 0 & 0 \\
com.fasterxml/aalto-xml & xml-processing & 0 & 0 & 0 & 0 & 0 & 0 & 0 & 0 \\
dom4j/dom4j & xml-processing & 0 & 0 & 0 & 0 & 0 & 0 & 0 & 0 \\
jdom/jdom & xml-processing & 0 & 0 & 0 & 0 & 0 & 0 & 0 & 0 \\
net.sf.kxml/kxml2 & xml-processing & 0 & 0 & 0 & 0 & 0 & 0 & 0 & 0 \\
xstream/xstream & xml-processing & 0 & 0 & 0 & 0 & 0 & 0 & 0 & 0 \\ \hline
\end{tabular}
}
\captionof{table}{Παράδειγμα μερικών στηλών για όλες τις γραμμές του αρχικού dataset}
\label{table:init}
\end{center}

\section{Εξερεύνηση Δεδομένων}
Πριν την διαδικασία της προεπεξεργασίας των δεδομένων, πραγματοποιούμε μία αρχική εξερεύνηση τους για την κατανόηση του προβλήματος ομαδοποίησης που καλούμαστε να αντιμετωπίσουμε.

Το αρχικό σύνολο δεδομένων, που έχουμε στη διάθεσή μας, παρατηρούμε ότι περιλαμβάνει έναν πολύ μεγάλο αριθμό λέξεων που αποτελούν ουσιαστικά τα attributes για τους αλγορίθμους μας.
Για αυτό το λόγο, στόχος μας ήταν η δραστική μείωση του αριθμού αυτών των λέξεων ενώ αποφασίσαμε ότι δεν ήταν σημαντικό να αφαιρέσουμε κάποια βιβλιοθήκη (γραμμή) καθώς ο αριθμός τους ήταν ήδη περιορισμένος.

Επίσης, παρατηρήσαμε ότι υπήρχαν διάφορα προβλήματα πάνω στην ποιότητα των λέξεων του dataset:
\begin{itemize}
    \item Πολλές λέξεις δεν είχαν κάποιο νόημα πχ
    % manual hyphenation because it didn't work with anything else.
    \texttt{ABCDE\-FGHIJKLMNO\-PQRSTU\-VWXYZ\-abc\-d\-e\-f\-ghij\-klmnopqrstuvwxyz}
    που λογικά χρησιμοποιείται ως ένα string που περιέχει όλους τους χαρακτήρες της Αγγλικής γλώσσας.

    \item Πολλές λέξεις υπήρχαν σε διάφορα σημεία αλλά λόγω της διάκρισης κεφαλαίων και πεζών θεωρούνταν ξεχωριστές λέξεις.
    πχ \texttt{ACCESS}, \texttt{access} και \texttt{Access}.

    \item Μερικές λέξεις ήταν παρόμοιες ή η μια ήταν στον ενικό ενώ η άλλη στον πληθυντικό ή υπήρχαν getters και setters.
    πχ \begin{itemize}
        \item \texttt{action} και \texttt{actions}.
        \item \texttt{accounts} και \texttt{getAccounts}.
    \end{itemize}
\end{itemize}

Τελικά, αποφασίσαμε ότι σε πολλές περιπτώσεις είναι πιο σημαντική η ύπαρξη ή όχι μιας λέξης σε μια βιβλιοθήκη παρά τον αριθμό των εμφανίσεών της.
Αυτό συμβαίνει γιατί γενικά το dataset μας είναι σχετικά αραιό.

\section{Η διαδικασία προεπεξεργασίας δεδομένων}
Η επεξεργασία έγινε στο Python
\footnote{Γραμμένο για να δουλεύει με python3 αλλά έγιναν διάφορες αλλαγές ώστε να τρέχει σωστά και σε python2.}
αρχείο \texttt{preprocess.py}.
Το dataset μας αρχικά διαβάζεται από ένα \texttt{.csv} αρχείο και αποθηκεύεται σε ένα αντικείμενο \lstinline!DataFrame! της βιβλιοθήκης \lstinline!pandas!.

Κατά την προεπεξεργασία των δεδομένων ακολουθήθηκε μια δεντρική δομή για την παραγωγή διάφορων τελικών dataset.
Αυτή η δομή προσδιορίζεται από τη συνάρτηση \lstinline!tree_init()!
και μπορεί να γίνει εύκολα η επεξεργασία της ώστε να αλλάξουν τα τελικά αποτελέσματα.

Ρίζα του δέντρου θεωρείται πάντα το αρχικό μας dataset και κάθε μεταβολή του αναπαριστάται σε ένα κόμβο παιδί.
Η κάθε διαδικασία επεξεργασίας αντιστοιχεί σε μία ακμή του δέντρου.
Τα διάφορα dataset που δημιουργούνται αποθηκεύονται σαν αρχεία και πάλι σε δεντρική δομή όπου οι φάκελοι είναι οι ακμές και τα \texttt{.csv} αρχεία οι κόμβοι.
Μια τέτοια δομή φαίνεται παρακάτω:
\begin{Verbatim}[frame=single]
datasets/root
├── join_duplicates
│   ├── frequency_based_selection
│   │   ├── gibberish_detector
│   │   │   ├── bool_it
│   │   │   │   ├── frequency_based_selection2
│   │   │   │   │   └── dataset.csv
│   │   │   │   └── dataset.csv
│   │   │   ├── join_similar
│   │   │   │   ├── drop_fry_words
│   │   │   │   │   ├── frequency_based_selection_df
│   │   │   │   │   │   └── dataset.csv
│   │   │   │   │   └── dataset.csv
│   │   │   │   └── dataset.csv
│   │   │   └── dataset.csv
│   │   └── dataset.csv
│   └── dataset.csv
└── dataset.csv
\end{Verbatim}

Στη συνέχεια περιγράφονται οι διάφορες διαδικασίες που αναπτύχθηκαν.

\subsection{Frequency Based Selection}
Αφαίρεση των λέξεων που εμφανίζονται πολύ συχνά ή πολύ σπάνια.
Η επιλογή βασίζεται μόνο στο αν μια λέξη εμφανίζεται ή όχι σε μια βιβλιοθήκη.
Ο απόλυτος αριθμός εμφανίσεων δεν έχει σημασία.
Έτσι, αν θέσουμε κατώτατο όριο $3$ και μία λέξη εμφανιστεί $500$ φορές σε μια βιβλιοθήκη αλλά πουθενά αλλού, τότε αυτή η λέξη θεωρείται ότι εμφανίζεται πολύ σπάνια και θα κοπεί.

Η υλοποίηση βρίσκεται στη συνάρτηση \lstinline!frequency_based_selection! και ακολουθούμε τα εξής βήματα:
\begin{enumerate}
\item Μετατροπή του dataset σε bool.
\begin{lstlisting}[numbers=none, aboveskip=\smallskipamount, belowskip=\smallskipamount, captionpos=none]
to_bool = dataset.applymap(lambda x: True if x else False)
\end{lstlisting}

\item Άθροισμα κατά γραμμή.
\begin{lstlisting}[numbers=none, aboveskip=\smallskipamount, belowskip=\smallskipamount, captionpos=none]
to_bool_sums = to_bool.sum(axis=0)
\end{lstlisting}

\item Εύρεση και αφαίρεση όσων στηλών είναι εκτός των ορίων από το αρχικό dataset \lstinline!low_bound! και \lstinline!upper_bound!
\begin{lstlisting}[numbers=none, aboveskip=\smallskipamount, belowskip=\smallskipamount, captionpos=none]
to_drop = [
    column
    for column, nonzeros in zip(columns, to_bool_sums)
    if nonzeros < low_bound or nonzeros > upper_bound
]
return dataset.drop(to_drop, axis=1)
\end{lstlisting}
\end{enumerate}

\subsection{Join Duplicates}
Συνάθροιση των λέξεων που είναι διπλές.
Αυτή η συνάρτηση συνήθως έχει νόημα αν μετατρέψουμε το αρχείο σε lower case καθώς το διαβάζουμε.
\begin{lstlisting}[captionpos=none, numbers=none]
def join_duplicates(dataset):
    """Join duplicate words."""
    return dataset.groupby(dataset.columns, axis=1).sum()
\end{lstlisting}
