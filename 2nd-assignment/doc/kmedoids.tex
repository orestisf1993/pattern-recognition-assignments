\section{Κ-medoids}
Ο K-medoids είναι μία παραλλαγή του k-means που χρησιμοποιείται σε κατηγορικά η διακριτά δεδομένα. Η βασική του διαφορά είναι ότι  χρησιμοποιεί σαν κέντρο του cluster ένα σημείο από αυτό το επονομαζόμενο medoid.

Γενικά θεωρείται πιο ανθεκτικός στο θόρυβο επειδή προσπαθεί να ελαχιστοποιήσει την ανομοιότητα ανάμεσα σε στοιχεία παρά το τετραγωνικό άθροισμα των αποστάσεων όπως κάνει συχνά ο k-means.Σαν medoid επιλέγεται συνήθως το σημείο το οποίο διαφέρει λιγότερα από όλα τα σημεία του cluster.


Ένας από του πιο συχνούς αλγορίθμους του k-medoids είναι o\textbf{ Partitioning Around Medoids (PAM)} Και ο τρόπος με τον οποίο δουλεύει 
περιγράφεται με τον ακόλουθο ψευδοκώδικα:
\begin{enumerate}
  \item Αρχικοποίηση$\:$διάλεξε στην τύχη $\kappa$ σημεία ως medoids
  \item Συσχέτισε το κάθε σημείο με το κοντινότερο medoid
  \item Ενώ το συνολικό κόστος  μειώνεται(συνολικό κόστος των cluster)
  \begin{enumerate}
  	\item Για κάθε σημείο medoid m,για κάθε μη-medoid σημείο o
  	\begin{enumerate}
  		\item Άλλαξε(swap) το m με το ο , ξαναυπολόγισε το κόστος του cluster(άθροισμα αποστάσεων από το medoid) 
  		\item αν το συνολικό κόστος αυξηθεί τότε αναίρεσε την αλλαγή(swap) 
  	\end{enumerate}
  \end{enumerate}
\end{enumerate}


\begin{minipage}{\linewidth}% to keep image and caption on one page
Ένα παράδειγμα του παραπάνω αλγορίθμου φάινεται στο παρακάτω σχήμα:
	\makebox[\linewidth]{%        to center the image
		\includegraphics[keepaspectratio=true,scale=0.5]{images/kmedoid}}
	\captionof{figure}{Παράδειγμα δεντρο-διαγράμματος}\label{fig:dentro1}%      only if needed  
\end{minipage}