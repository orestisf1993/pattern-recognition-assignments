\subsection{Κ-medoids}
\subsubsection{Εισαγωγή}
Ο K-medoids είναι μία παραλλαγή του k-means που χρησιμοποιείται σε κατηγορικά η διακριτά δεδομένα.
Η βασική του διαφορά είναι ότι χρησιμοποιεί σαν κέντρο του cluster ένα σημείο από αυτό το επονομαζόμενο medoid.

Γενικά θεωρείται πιο ανθεκτικός στο θόρυβο επειδή προσπαθεί να ελαχιστοποιήσει την ανομοιότητα ανάμεσα σε στοιχεία παρά το τετραγωνικό άθροισμα των αποστάσεων όπως κάνει συχνά ο k-means.
Σαν medoid επιλέγεται συνήθως το σημείο το οποίο διαφέρει λιγότερα από όλα τα σημεία του cluster.

Ένας από του πιο συχνούς αλγορίθμους του k-medoids είναι o \textbf{Partitioning Around Medoids (PAM)}.
Ο τρόπος με τον οποίο δουλεύει
περιγράφεται με τον ακόλουθο ψευδοκώδικα:\\
\begin{algorithm}[H]
    Διάλεξε στην τύχη $\kappa$ σημεία ως medoids\;
    Συσχέτισε το κάθε σημείο με το κοντινότερο medoid\;
    \While{το συνολικό κόστος μειώνεται(συνολικό κόστος των cluster)}{
        \For{κάθε σημείο medoid $m$, για κάθε μη-medoid σημείο $o$}{
            Άλλαξε (swap) το $m$ με το $o$ \;
            Ξαναυπολόγισε το κόστος του cluster(άθροισμα αποστάσεων από το medoid) \;
            \If{το συνολικό κόστος αυξηθεί}{
                Αναίρεσε την αλλαγή (swap)
            }
        }
    }
\end{algorithm}

Ένα παράδειγμα του παραπάνω αλγορίθμου φαίνεται στο 
\hyperref[fig:kmedoid-example]{\figurename{} \ref{fig:kmedoid-example}}:

\noindent\begin{minipage}{\linewidth}
    \centering
    \captionsetup{type={figure}}
	\includegraphics[width=\linewidth]{images/kmedoid}
	\captionof{figure}{Παράδειγμα k-medoid}\label{fig:kmedoid-example}
\end{minipage}

\subsubsection{Πειραματικά Αποτελέσματα}
Καθώς ο K-medoids αποτελεί μια παραλλαγή του K-means πιο ανθεκτική στον θόρυβο δεν περιμένουμε να έχουμε καλύτερα αποτελέσματα με την εφαρμογή του καθώς δεν έχουμε ανεπιθύμητο θόρυβο στα δεδομένα μας. Τρέξαμε όλα τα dataset μας για τον αλγόριθμο K-medoids στο Matlab χρησιμοποιώντας την εντολή
\lstinline[language=MATLAB, breaklines=true]!kmedoids(X, clnumber, 'Distance', correlation);!
Η μόνη παράμετρος που αλλάξαμε ήταν η απόσταση και χρησιμοποιήσαμε τον τύπο correlation. Τρέξαμε για είδος απόστασης cosine και τα αποτελέσματα ήταν ίδια. Παρατηρήσαμε ότι οι άλλες παράμετροι της εντολής \lstinline[language=MATLAB]!kmedoids! δεν δίνουν διαφορετικά αποτελέσματα για αυτό χρησιμοποιήθηκαν οι default τιμές τους. Για να πάρουμε τα διαγράμματα με τις μετρικές τρέξαμε το script  \lstinline[language=MATLAB]!scriptm!. Αυτό το script καλεί την συνάρτηση \lstinline[language=MATLAB]!optimizer_kmedoids! η οποία διαβάζει τα dataset μας, υλοποιεί τον αλγόριθμο K-medoids και μας βγάζει τα διαγράμματα στα οποία έχουμε τις μετρικές μας.

Το διάγραμμα που παίρνουμε είναι το \hyperref[fig:MedCorBar]{\figurename{} \ref{fig:MedCorBar}}:
Παρατηρούμε ότι τα αποτελέσματα μας είναι αρκετά ικανοποιητικά, δεν φτάνουν όμως τα επίπεδα των αποτελεσμάτων του K-means.
Στο
\hyperref[fig:kmedoid_result]{\figurename{} \ref{fig:kmedoid_result}}
βλέπουμε τα αποτελέσματα από τις βέλτιστες ομαδοποιήσεις ως προς τις μετρικές $SuccessRate$ και $Silhouette$ αντίστοιχα.
Παρατηρούμε ότι και στις 2 ομαδοποιήσεις ενώ πετυχαίνουμε υψηλά ποσοστά στην ομαδοποίηση των μεγάλων cluster χάνουμε πιο πολλές βιβλιοθήκες από πριν.
Τέλος, ο
\hyperref[table:k-medoids-best]{\tablename{} \ref{table:k-medoids-best}}
παρουσιάζει τις βέλτιστες λύσεις για το $Silhouette$ και για το $Success Rate$ για την ομαδοποίηση που έγινε.

\begin{figure}[htbp]
    \centering
    \captionsetup{type={figure}}
    \includegraphics[width=\linewidth]{images/MedCorBar}
    \captionof{figure}{Αποτελέσματα k-medoids}
    \label{fig:MedCorBar}
\end{figure}

\begin{figure}[htbp]
    \centering
    \captionsetup{type={figure}}
    \includegraphics[width=\linewidth]{images/kmedoid_result1}
    \includegraphics[width=\linewidth]{images/kmedoid_2}
    \captionof{figure}{Αποτελέσματα k-medoids}
    \label{fig:kmedoid_result}
\end{figure}

\begin{table}[htbp]
	\centering
	\resizebox{\textwidth}{!}{%
		\begin{tabular}{llllll}
			Dataset & Cohesion & Separation & Silhouette & Success Rate & Max Type \\
			&  &  &  &  & Best Success Rate \\
			&  &  &  &  & Best Silhouette
		\end{tabular}
	}
	\caption{Συνοπτικός πίνακας καλύτερων αποτελεσμάτων k-medoids}
	\label{table:k-medoids-best}
\end{table}
\FloatBarrier