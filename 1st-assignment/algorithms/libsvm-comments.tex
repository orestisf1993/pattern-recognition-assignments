Όπως αναφέραμε και παραπάνω ο LibSVM για να εκτελεστεί πρέπει τα δεδομένα εισόδου να είναι γραμμικά διαχωρίσιμα.
Το δικό μας αρχικό dataset δεν περιέχει αυτήν την ιδιότητα.
Για αυτόν τον λόγο εφαρμόζουμε PCA στο αρχικό μας dataset.
Το dataset που προέκυψε μετά το PCA είναι ένα ισοδύναμο και γραμμικά διαχωριζόμενο.

\begin{sloppypar}
Έπειτα, υπολογίσαμε τις τιμές των παραμέτρων για βελτιστοποίηση του f-measure.
Φορτώσαμε τον \lstinline[language=Java]!Cost.SensitiveClassifier!
και υπολογίσαμε την τιμή $cost(2,1)=5.47$.
Ακόμα, υπολογίσαμε την τιμή cost του LibSVM 2.06 καθώς επίσης επιλέξαμε radial basis function(RBF) πυρήνα.
\end{sloppypar}

Στόχος μας ήταν η διατήρηση του f-measure όσο το δυνατόν υψηλότερα μπορούσαμε για την κλάση 1 που υποδηλώνει την ύπαρξη bugs.
Επιτύχαμε ποσοστό περίπου $63\%$ διατηρώντας μεγαλύτερη έμφαση στο recall το οποίο βρίσκεται στο $78\%$.
Ακόμα, παρατηρούμε ότι και τα αντίστοιχα μεγέθη της κλάσης 0 παραμένουν υψηλά.
Τέλος, επιτύχαμε ποσοστό ακρίβειας $80.6\%$.