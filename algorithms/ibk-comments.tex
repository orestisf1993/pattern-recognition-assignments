Χρησιμοποιήσαμε το dataset όπως μας δίνεται έχοντας βγάλει μόνο το χαρακτηριστικό classid.
Ακόμα, κάνουμε κανονικοποίηση των δεδομένων μας.
Έπειτα, βελτιστοποιώντας τις παραμέτρους μας ως προς το f-measure επιλέγουμε k=18.
Η μετρική που χρησιμοποιούμε είναι η Ευκλείδια απόσταση και σαν distanceWeighting επιλέγουμε weight by 1/distance.
Ακόμα φορτώσαμε τον
\lstinline[language=Java]!Cost.SensitiveClassifier!
και υπολογίσαμε την τιμή $cost(2,1)=4.5$.

Παρατηρούμε ότι το f-measure για την κλάση 1 που μας ενδιαφέρει και υποδηλώνει την ύπαρξη bugs είναι $63\%$.
Αρκετά μεγάλο για το συγκεκριμένο dataset.
Ακόμα, παρατηρούμε ότι το recall είναι κοντά στο $80\%$.
Γενικά έχουμε καταφέρει να επιτύχουμε υψηλό ποσοστό recall χωρίς το precision και κατ'επέκταση το f-measure να πάρουν πολύ μικρές τιμές.
Τέλος, πετύχαμε αρκετά μεγάλο ποσοστό ακρίβειας $81\%$.
Όλα τα παραπάνω τα πετύχαμε χώρις να πέσουν τα αντίστοιχα ποσοστά της κλάσης 0.
Συμπεραίνουμε ότι πρόκειται για ένα μοντέλο το οποίο μπορεί να μας δώσει αρκετά ικανοποιητικά αποτελέσματα.