\chapter{Τελικά αποτελέσματα}

Σαν καλύτερο μοντέλο ως προς f-measure και recall επιλέξαμε αυτό του \hyperref[sssec:j48]{J48}.
Από τον explorer του weka εφαρμόσαμε το εκπαιδευμένο μοντέλο στο test dataset και αποθηκεύσαμε το αποτέλεσμα.
Στο \hyperref[fig:train-set-visualize]{\figurename{} \ref{fig:train-set-visualize}}
φαίνεται το visualization του train dataset από το weka 
ενώ στο \hyperref[fig:test-set-visualize]{\figurename{} \ref{fig:test-set-visualize}}
φαίνεται το visualization του test dataset.
Παρατηρούμε ότι η αναλογία των κλάσεων με bugs προς αυτή που δεν έχουν bugs είναι ελαφρώς αυξημένη στα 
test dataset.
Υποθέτοντας ότι η πραγματική αναλογία είναι παρόμοια, η αυξημένη αναλογία που παρατηρείται θεωρείται λογική καθώς δώσαμε βάση στο recall,
μειώνοντας έτσι το precision.

\begin{figure}[htb]
\centering
\includegraphics[width=1.0\textwidth]{bugs_train_visualize.png}
\caption{Visualization του train dataset}
\label{fig:train-set-visualize}
\end{figure}
\begin{figure}[htb]
\centering
\includegraphics[width=1.0\textwidth]{test_visualize.png}
\caption{Visualization του test dataset}
\label{fig:test-set-visualize}
\end{figure}